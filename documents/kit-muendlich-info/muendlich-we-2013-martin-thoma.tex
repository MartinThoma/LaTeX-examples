\documentclass[a4paper]{article}
\usepackage{myStyle}

%%%%%%%%%%%%%%%%%%%%%%%%%%%%%%%%%%%%%%%%%%%%%%%%%%%%%%%%%%%%%%%%%%%%%
% Hier eigene Daten einfügen                                        %
%%%%%%%%%%%%%%%%%%%%%%%%%%%%%%%%%%%%%%%%%%%%%%%%%%%%%%%%%%%%%%%%%%%%%
\newcommand{\Studiengang}{Informatik (BA)}
\newcommand{\Fach}{Web Engineering}
\newcommand{\Pruefungsdatum}{19.02.2013}    % DD.MM.YYYY
\newcommand{\Pruefer}{Dr. Nußbaumer}
\newcommand{\Beisitzer}{Matthias Keller}
% Nicht zwingend, aber es waere nett, wenn du zumindest die Zahl vor 
% dem Komma angeben koenntest:
\newcommand{\Note}{1,0}
\newcommand{\Dauer}{15} % in Minuten

%%% WEITER SCROLLEN %%%%%%%%%%%%%%%%%%%%%%%%%%%%%%%%%%%%%%%%%%%%%%%%%%%%%%%%%%%%%

\begin{document}
\begin{tabular}{p{2cm}p{15cm}}
\ifpdf\vspace{-0.8cm}\fi
\multirow{2}{2cm}{ \includegraphics[width=20mm]{FS-Eule}} & 

\Large Fragebogen der Fachschaft zu \\
& \Large {\bfseries mündlichen Prüfungen} \\
& \Large{im Informatikstudium}
\\
\end{tabular}

 \begin{tabular}{p{8cm}p{8cm}}
  \begin{flushleft}Dieser Fragebogen gibt den Studierenden,
   die nach Dir die Prüfung ablegen wollen, einen Einblick in Ablauf 
   und Inhalt der Prüfung. Das erleichtert die Vorbereitung.

   Bitte verwende zum Ausfüllen einen schwarzen Stift.
   Das erleichtert das Einscannen. \\[0.5cm]
%%% HIER GEHTS LOS! %%%%%%%%%%%%%%%%%%%%%%%%%%%%%%%%%%%%%%%%%%%%%%%%%%%%%%%%%%%%%

%%%%%%%%%%%%%%%%%%%%%%%%%%%%%%%%%%%%%%%%%%%%%%%%%%%%%%%%%%%%%%%%%%%%%
% Das Dokument                                                      %
%%%%%%%%%%%%%%%%%%%%%%%%%%%%%%%%%%%%%%%%%%%%%%%%%%%%%%%%%%%%%%%%%%%%%
   Dein Studiengang: \Studiengang \\[0.5cm]

   \textbf{Prüfungsart:}\\
%% entsprechende \boxempty bitte durch \boxtimes ersetzen.
   $\boxempty$ Wahlpflichtfach  \\
   $\boxempty$ Vertiefungsfach  \\
   $\boxempty$ Ergänzungsfach  \\[0.5cm]
%% Namen des Wahl/Vertiefungs/Ergaenzungsfachs hier bitte eintragen.
   Welches? \Fach
%% Jetzt kommt ein Barcode von uns.  Einfach weitergehen.  ;-)
  \end{flushleft}
  & 
  \begin{center}
   Barcode:
   \begin{tabular}{p{0.2cm}p{6.8cm}p{0.2cm}} 
   $\ulcorner$
   \vskip 2cm
   $\llcorner$ & & $\urcorner$
   \vskip 2cm
   $\lrcorner$ \\
   \end{tabular}
  \end{center}
  \vskip 0.5cm
%% Hier gehts weiter:
  \begin{flushright}
%% Pruefungsdatum, PrueferIn und BeisitzerIn bitte hier eintragen. Wichtig: Im Allgemeinen kann nur ein Professor der Pruefer gewesen sein.
  \begin{tabular}{ll}
   Prüfungsdatum:   & \Pruefungsdatum \\[0.5cm]
   Prüfer/-in:      & \Pruefer \\[0.5cm]
   Beisitzer/-in:   & \Beisitzer \\
  \end{tabular}
  \end{flushright} \\
 \end{tabular} 

 \begin{tabular}{|p{8.2cm}|p{3cm}|p{1cm}|p{3.5cm}|}
  \multicolumn{4}{l}{\bfseries Prüfungsfächer und Vorbereitung: } \\[0.2cm]
  \hline
  Veranstaltung & Dozent/-in  & Jahr & regelmäßig besucht? \\
  \hline
  \hline
%% Beispiel:
%% Interessante Vorlesung & Toller Prof & 2007 & Ich war immer 5 Minuten vorher da \\
  Web Engineering & Dr. Nußbaumer & 12/13 &  Ja \\[0.2cm]
  \hline
 \end{tabular} \\[0.5cm]

\begin{multicols}{2}
Note: \Note\\[0.5cm]
War diese Note angemessen?
%% Hier ist Platz fuer deinen Kommentar
Ja

\columnbreak
%% Bitte Pruefungsdauer eintragen
Prüfungsdauer: \Dauer{} Minuten \\[0.5cm]
\end{multicols}


 \textbf{\ding{46}} Wie war der \textbf{Prüfungsstil des Prüfers / der Prüferin?} \\
 \begin{footnotesize} (Prüfungsatmosphäre, (un)klare Fragestellungen, Frage nach Einzelheiten oder eher größeren Zusammenhängen, kamen häufiger Zwischenfragen oder ließ er/sie dich erzählen, wurde Dir weitergeholfen, wurde in Wissenslücken gebohrt?)\end{footnotesize}  \\
 \begin{minipage}[t][10cm]{\linewidth}
%% Hier ist Platz fuer deinen Kommentar
    Die Fragen waren klar gestellt. Die Atmosphäre war
    angenehm; er hat einen viel erzählen lassen.


 \end{minipage}

 \begin{flushright}$\hookrightarrow$\textbf{Rückseite bitte nicht vergessen!}\end{flushright}

 \newpage
 \columnseprule=.4pt

 \begin{multicols}{2}

  \ding{46} Hat sich der \textbf{Besuch / Nichtbesuch} der Veranstaltung für dich gelohnt? \\
  \begin{minipage}[t][6.8cm]{\linewidth}
%% Hier ist Platz fuer deinen Kommentar
    Ja. In der Vorlesung wurden interessante Diskussionen geführt.

  \end{minipage}

  \ding{46} Wie lange und wie hast du dich \textbf{alleine bzw. mit anderen vorbereitet}? \\ 
  \begin{minipage}[t][7cm]{\linewidth}
%% Hier ist Platz fuer deinen Kommentar
    5 Treffen à 1,5h mit einem Lernpartner sowie 30 Stunden alleine vorbereiten

  \end{minipage}

  \ding{46} Welche \textbf{Tips zur Vorbereitung} kannst du geben?
  \begin{footnotesize}(Wichtige / Unwichtige Teile des Stoffes, gute Bücher / Skripten, Lernstil)\end{footnotesize} \\
  \begin{minipage}[t][7cm]{\linewidth}
%% Hier ist Platz fuer deinen Kommentar
    Folien lesen und verstehen, Protokolle durchgehen und
    meinen Blog lesen:\\
    martin-thoma.com/web-engineering
  \end{minipage}

\columnbreak

  \ding{46} Kannst du ihn/sie \textbf{weiterempfehlen}? 
%% entsprechende \boxempty bitte durch \boxtimes ersetzen. 
  $\boxtimes$ Ja / $\boxempty$ Nein\newline Warum? \\
  \begin{minipage}[t][6.8cm]{\linewidth}
%% Hier ist Platz fuer deinen Kommentar
    Sehr nett, angenehme Athmosphäre.

  \end{minipage}

  \ding{46} Fanden vor der Prüfung \textbf{Absprachen} zu Form oder Inhalt statt? Wurden sie \textbf{eingehalten}? \\
  \begin{minipage}[t][7cm]{\linewidth}
%% Hier ist Platz fuer deinen Kommentar
    Nein, es gab keine Absprachen.

  \end{minipage}

  \ding{46} Kannst du Ratschläge für das \textbf{Verhalten in der Prüfung} geben? \\
  \begin{minipage}[t][6.8cm]{\linewidth}
%% Hier ist Platz fuer deinen Kommentar
    Mit den Antworten kann man etwas lenken, was als nächstes 
    gefragt wird.

  \end{minipage}
% 
\end{multicols}
\clearpage 

\section*{Inhalte der Prüfung:}
\fbox{\parbox{17cm}{
 \begin{itemize}
  \item Schreibe bitte möglichst viele Fragen und Antworten auf.
  \item Wo wurde nach Herleitungen oder Beweisen gefragt oder sonstwie nachgehakt?  
  \item Worauf wollte der Prüfer / die Prüferin hinaus?
  \item Welche Fragen gehörten nicht zum eigentlichen Stoff?
 \end{itemize}
}}


    \begin{itemize}
        \item Worum geht es im Web Engineering?
        \item[$\Rightarrow$] Software Engineering, Information Systems, Network Engineering und Hypermedia.
        \item Was ist Hypermedia?
        \item[$\Rightarrow$] Erst habe ich Hypertext erklärt (Text, der mit Links auf weitere Ressourcen delinearisiert werden kann).
             Hypermedia ist das gleiche wie Hypertext, nur zusätzlich mit anderen Medien wie z.B. Video. 
             Bei Hypermedia kann man im Information-Space frei navigieren.
        \item Was ist eine Ressource?
        \item[$\Rightarrow$] Addressierbare Einheiten, MIME-Types erklärt.
        \item Wie sieht ein HTTP-Request aus?
        \item[$\Rightarrow$] Habe ihm den Aufbau in Header / Content und Inhalte des
              Headers erklärt, aber das wollte er nicht hören. Er
              wollte auf die HTTP-Options (insbesondere GET) hinaus.
        \item Sind agile Methoden die Antwort auf alle Probleme?
        \item[$\Rightarrow$] Natürlich nicht, man muss das Prozess-Modell 
              nach den Anforderungen des Projekts wählen. Gäbe es ein
              Prozess-Modell für alle Projekt-Typen, würde man dieses
              immer nutzen.
        \item Was machen Agile Prozesse aus?
        \item[$\Rightarrow$] Werte des Agiles Manifesto.
        \item Will man in Agilen Methoden keinen Prozess?
        \item[$\Rightarrow$] Nein, aber "`Value individuals and interactions over processes and tools"'
    \end{itemize}
\end{document}
