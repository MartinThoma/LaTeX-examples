\documentclass[a4paper,9pt]{scrartcl}
\usepackage{amssymb, amsmath} % needed for math
\usepackage{} % needed for math
\usepackage[utf8]{inputenc} % this is needed for umlauts
\usepackage[ngerman]{babel} % this is needed for umlauts
\usepackage[T1]{fontenc}    % this is needed for correct output of umlauts in pdf
\usepackage[margin=2.5cm]{geometry} %layout
\usepackage{hyperref}  % links im text
\usepackage{color}
\usepackage{framed}
\usepackage{enumerate}  % for advanced numbering of lists
\clubpenalty  = 10000   % Schusterjungen verhindern
\widowpenalty = 10000   % Hurenkinder verhindern

\hypersetup{ 
  pdfauthor   = {Martin Thoma}, 
  pdfkeywords = {Lineare Algebra}, 
  pdftitle    = {Lineare Algebra - Definitionen} 
} 

%%%%%%%%%%%%%%%%%%%%%%%%%%%%%%%%%%%%%%%%%%%%%%%%%%%%%%%%%%%%%%%%%%%%%
% Custom definition style, by                                       %
% http://mathoverflow.net/questions/46583/what-is-a-satisfactory-way-to-format-definitions-in-latex/58164#58164
%%%%%%%%%%%%%%%%%%%%%%%%%%%%%%%%%%%%%%%%%%%%%%%%%%%%%%%%%%%%%%%%%%%%%
\makeatletter
\newdimen\errorsize \errorsize=0.2pt
% Frame with a label at top
\newcommand\LabFrame[2]{%
    \fboxrule=\FrameRule
    \fboxsep=-\errorsize
    \textcolor{FrameColor}{%
    \fbox{%
      \vbox{\nobreak
      \advance\FrameSep\errorsize
      \begingroup
        \advance\baselineskip\FrameSep
        \hrule height \baselineskip
        \nobreak
        \vskip-\baselineskip
      \endgroup
      \vskip 0.5\FrameSep
      \hbox{\hskip\FrameSep \strut
        \textcolor{TitleColor}{\textbf{#1}}}%
      \nobreak \nointerlineskip
      \vskip 1.3\FrameSep
      \hbox{\hskip\FrameSep
        {\normalcolor#2}%
        \hskip\FrameSep}%
      \vskip\FrameSep
    }}%
}}
\definecolor{FrameColor}{rgb}{0.25,0.25,1.0}
\definecolor{TitleColor}{rgb}{1.0,1.0,1.0}

\newenvironment{contlabelframe}[2][\Frame@Lab\ (cont.)]{% 
  % Optional continuation label defaults to the first label plus
  \def\Frame@Lab{#2}%
  \def\FrameCommand{\LabFrame{#2}}%
  \def\FirstFrameCommand{\LabFrame{#2}}%
  \def\MidFrameCommand{\LabFrame{#1}}%
  \def\LastFrameCommand{\LabFrame{#1}}%
  \MakeFramed{\advance\hsize-\width \FrameRestore} 
}{\endMakeFramed} 
\newcounter{definition}
\newenvironment{definition}[1]{%
  \par
  \refstepcounter{definition}%
  \begin{contlabelframe}{Definition \thedefinition:\quad #1}
 \noindent\ignorespaces}
{\end{contlabelframe}} 
\makeatother
%%%%%%%%%%%%%%%%%%%%%%%%%%%%%%%%%%%%%%%%%%%%%%%%%%%%%%%%%%%%%%%%%%%%%
% Begin document                                                    %
%%%%%%%%%%%%%%%%%%%%%%%%%%%%%%%%%%%%%%%%%%%%%%%%%%%%%%%%%%%%%%%%%%%%%
\begin{document}

\begin{definition}{injektiv, surjektiv und bijektiv}
Sei $f: A \rightarrow B$ eine Abbildung.
  \begin{enumerate}[(a)]
    \item $f$ heißt \textbf{surjektiv} $:\Leftrightarrow f(A) = B$
    \item $f$ heißt \textbf{injektiv}  $:\Leftrightarrow \forall x_1, x_2 \in A: x_1 \neq x_2 \Rightarrow f(x_1) \neq f(x_2)$
    \item $f$ heißt \textbf{bijektiv}  $:\Leftrightarrow f$ ist surjektiv und injektiv
  \end{enumerate}
\end{definition}

\begin{definition}{Relation}
Seien A und B Mengen. $R \subseteq A \times B$ heißt \textbf{Relation}.
\end{definition}

\begin{definition}{Ordnungsrelation}
Eine Relation $\leq$ heißt Ordnungsrelation in A und $(A, \leq)$ heißt
(partiell) geordnete Menge, wenn für alle $a, b, c  \in A$ gilt:

  \begin{description}
    \item[O1] $a \leq a$ (reflexiv)
    \item[O2] $a \leq b \land b \leq a \Rightarrow a = b$ (antisymmetrisch)
    \item[O3] $a \leq b \land b \leq c \Rightarrow a \leq c$ (transitiv)
  \end{description}

\noindent $(A, \leq)$ heißt total geordnet $:\Leftrightarrow \forall a, b, \in A: a \leq b \lor b \leq a$
\end{definition}

\begin{definition}{Äquivalenzrelation}
Sei $R \subseteq A \times A$ eine Relation. 
R heißt Äquivalenzrelation, wenn für alle $a, b, c \in A$ gilt:

  \begin{description}
    \item[Ä1] $a R a$ (reflexiv)
    \item[Ä2] $a R b \Rightarrow b R a$ (symmetrisch)
    \item[Ä3] $a R b \land b R c \Rightarrow a R c$ (transitiv)
  \end{description}
\end{definition}

\begin{definition}{Assoziativität}
Sei A eine Menge und $*$ eine Verknüpfung auf A.\\
A heißt \textbf{assoziativ} $:\Leftrightarrow \forall a, b, c \in A: (a * b) * c = a * (b*c)$
\end{definition}

\begin{definition}{Gruppe}
Sei G eine Menge und $*$ eine Verknüpfung auf G.\\
$(G, *)$ heißt \textbf{Gruppe} $: \Leftrightarrow$
  \begin{description}
    \item[G1] $\forall a, b, c \in G: (a * b)*c=a*(b*c)$ (assoziativ)
    \item[G2] $\exists e \in G \forall a \in G: e * a = a = a * e$ (neutrales Element)
    \item[G3] $\forall a \in G \exists a^{-1} \in G: a^{-1}*a=e=a*a^{-1}$ (inverses Element)
  \end{description}
\end{definition}

\begin{definition}{abelsche Gruppe}
Sei $(G, *)$ eine Gruppe.
$(G, *)$ heißt \textbf{abelsche Gruppe} $: \Leftrightarrow$
  \begin{description}
    \item[G4] $\forall a, b \in G: a * b = b * a$ (kommutativ)
  \end{description}
\end{definition}

\begin{definition}{Ring}
Sei R eine Menge und $+$ sowie $cdot$ Verknüpfungen auf R.\\
$(R, +, \cdot)$ heißt \textbf{Ring} $: \Leftrightarrow$
  \begin{description}
    \item[R1] $(R, +)$ ist abelsche Gruppe
    \item[R2] $\cdot$ ist assoziativ
    \item[R3] Distributivgesetze: $\forall a, b, c \in R: a \cdot (b+c) = a \cdot b + a \cdot c$ und $(b+c)\cdot a = b \cdot a + c \cdot a$
  \end{description}
\end{definition}

\begin{definition}{Nullteiler}
Sei $(R, +, \cdot)$ ein Ring.\\
$a \in R$ heißt (linker) \textbf{Nullteiler} $:\Leftrightarrow a \neq 0 \land \exists b: a \cdot b = 0$ 
\end{definition}

\begin{definition}{Ringhomomorphismus}
Seien $(R_1, +, \cdot)$ und $(R_2, +, \cdot)$ Ringe und $\Phi:R_1 \rightarrow R_2$ eine Abbildung.\\
$\Phi$ heißt \textbf{Ringhomomorphismus} $:\Leftrightarrow \forall x,y \in R_1: \Phi(x+y) = \Phi(x) + \Phi(y)$ und $\Phi(x \cdot y) = \Phi(x) \cdot \Phi(y)$ 
\end{definition}

\begin{definition}{Körper}
Sei $(\mathbb{K}, +, \cdot)$ ein Ring.\\
$(\mathbb{K}, +, \cdot)$ heißt \textbf{Körper} $:\Leftrightarrow (\mathbb{K} \setminus \{0\}, \cdot)$ ist eine abelsche Gruppe.
\end{definition}

\begin{definition}{Charakteristik}
Sei $(\mathbb{K}, +, \cdot)$ ein Körper.\\
Falls es ein $m \in N^+$ gibt, sodass
\[ \underbrace{1+1+ \dots + 1}_{m \text{ mal}} = 0 \]
gilt, so heißt die kleinste solche Zahl $p$ die Charakteristik ($\text{char } \mathbb{K}$) von $\mathbb{K}$.
Gibt es kein solches $m$, so habe $\mathbb{K}$ die Charaktersitik 0.
\end{definition}

\begin{definition}{Vektorraum}
Sei $(\mathbb{K}, +, \cdot)$ ein Körper und $V$ eine Menge mit einer Addition
\[ +: V \times V \rightarrow V, (x,y) \mapsto x  + y \]
und einer skalaren Multiplikation
\[ \cdot: \mathbb{K} \times V \rightarrow V, (\lambda, x) \mapsto \lambda \times x \]

heißt $\mathbb{K}$-Vektorraum, falls gilt:
  \begin{description}
    \item[V1] $(V, +)$ ist abelsche Gruppe
    \item[V2] für alle $\lambda, \mu \in \mathbb{K}$ und alle $x, y \in V$ gilt:
    \begin{enumerate}[(a)]
      \item $1 \cdot x = x$
      \item $\lambda \cdot (\mu \cdot x) = (\lambda \cdot \mu) \cdot x$
      \item $(\lambda + \mu) \cdot x = \lambda \cdot x + \mu \cdot x$
      \item $\lambda \cdot (x+y) = \lambda \cdot x + \lambda \cdot y$
    \end{enumerate}
  \end{description}
\end{definition}

\begin{definition}{Lineare Unabhängigkeit}
Sei V ein $\mathbb{K}$-Vektorraum. Endlich viele Vektoren $v_1, \dots, v_k \in V$
heißen \textbf{linear unabhängig}, wenn gilt:
\[ \displaystyle \sum_{i=1}^{k} \lambda_i v_i = 0 \Rightarrow \lambda_1 = \lambda_2 = \dots = \lambda_k = 0 \]
\end{definition}

\end{document}
