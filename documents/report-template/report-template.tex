\documentclass[11pt,a4paper,oneside]{report}
\usepackage{amssymb, amsmath} % needed for math
\usepackage{}               % needed for math
\usepackage[utf8]{inputenc} % this is needed for umlauts
\usepackage[ngerman]{babel} % this is needed for umlauts
\usepackage[T1]{fontenc}    % this is needed for correct output of umlauts in pdf
\usepackage[margin=2.5cm]{geometry} %layout
\usepackage{hyperref}   % links im text
\usepackage{color}
\usepackage{framed}
\usepackage{enumerate}  % for advanced numbering of lists
\clubpenalty  = 10000   % Schusterjungen verhindern
\widowpenalty = 10000   % Hurenkinder verhindern

%%%%%%%%%%%%%%%%%%%%%%%%%%%%%%%%%%%%%%%%%%%%%%%%%%%%%%%%%%%%%%%%%%%%%
% Change the following lines:                                       %
%%%%%%%%%%%%%%%%%%%%%%%%%%%%%%%%%%%%%%%%%%%%%%%%%%%%%%%%%%%%%%%%%%%%%
\newcommand\yourName{Martin Thoma}
\newcommand\yourTitle{Some Random Title}
\newcommand\yourKeywords{Lineare Algebra; LA; Mathematik; Universität}
%%%%%%%%%%%%%%%%%%%%%%%%%%%%%%%%%%%%%%%%%%%%%%%%%%%%%%%%%%%%%%%%%%%%%

\hypersetup { 
  pdfauthor   = {\yourName}, 
  pdfkeywords = {\yourKeywords}, 
  pdftitle    = {\yourTitle} 
}

\begin{document}
 \author{\yourName}
 \title{\yourTitle}

 \maketitle
 \tableofcontents

%%%%%%%%%%%%%%%%%%%%%%%%%%%%%%%%%%%%%%%%%%%%%%%%%%%%%%%%%%%%%%%%%%%%%
% Start editing your content here                                   %
%%%%%%%%%%%%%%%%%%%%%%%%%%%%%%%%%%%%%%%%%%%%%%%%%%%%%%%%%%%%%%%%%%%%%
\chapter{My first chapter}
\section{A first section}
Lorem ipsum dolor sit amet, consectetur adipiscing elit. Nulla quam 
elit, vestibulum nec facilisis at, condimentum id enim. Sed iaculis 
lacinia quam, vel accumsan eros tempor in. Integer ipsum metus, 
accumsan sit amet commodo a, egestas vitae sem. Mauris ut orci ut 
dolor viverra convallis nec a erat. Aenean consequat elit vel eros 
fermentum vestibulum id at ipsum. In vitae orci mauris, et rhoncus 
odio. Pellentesque habitant morbi tristique senectus et netus et 
malesuada fames ac turpis egestas.

\subsection{A subsection!}
Fusce libero nulla, euismod vel suscipit nec, elementum vel massa. 
Mauris ut sapien sed neque dignissim sodales. Proin accumsan, lectus 
non gravida dapibus, lorem leo tincidunt odio, in semper ligula 
libero bibendum lorem. Pellentesque venenatis massa a neque porttitor
congue. Maecenas ornare lacus ac orci mattis a placerat sapien 
euismod. In sed eros enim, non interdum nisi. Curabitur quis magna 
et tortor interdum pharetra. Donec sit amet turpis neque, quis congue
leo. Proin sit amet placerat dolor.

\subsection{A subsubsection}
You can go quite deep into details. You should use chapter, section, 
subsection and subsubsection in this order. No need for fancy bold or
underlined text.

One equation:
\begin{equation}
    \sin^2(x) + \cos^2(x) = 1
\end{equation}

Vivamus ante est, dictum at placerat id, semper auctor tellus. Donec 
sed ipsum enim, eget lacinia mi. Duis vulputate auctor ligula, sit 
amet suscipit lectus malesuada ut. Donec adipiscing rutrum dolor sit 
amet euismod. Aenean condimentum nibh vitae neque rhoncus ultrices. 
Vestibulum ultrices commodo mattis. Morbi aliquam elementum est, a 
pulvinar arcu viverra quis. Vivamus sed fermentum nisl. Cras 
bibendum, justo tincidunt dictum venenatis, sem turpis vestibulum 
nibh, ut dapibus nunc enim ut justo. 

\chapter{A new start}
Nullam iaculis vulputate 
aliquam. Sed nulla metus, facilisis eu porta sed, aliquet id mi. 
Duis congue blandit quam, a auctor turpis rhoncus non. Nam id dictum 
nibh. Integer vitae lacus sit amet ipsum semper interdum nec eget 
nisl. Vivamus aliquet, augue ut ultrices auctor, neque lacus 
vestibulum leo, vel dapibus ligula nisi ac nibh. Etiam accumsan, 
felis ut dapibus sagittis, velit turpis pellentesque nibh, nec cursus
libero erat at odio.

A Quadratic Equation is an equation in the form:
\begin{equation}
    ax^2+bx+c=0
\end{equation}
where \(a,b,c\in\mathbb{R}\).

Fusce commodo erat et eros tincidunt quis elementum dui tincidunt. 
Sed blandit elementum eros nec pulvinar. Aenean vitae dignissim est. 
Sed auctor porttitor tempor. Donec id euismod diam. Aenean vulputate 
hendrerit metus sit amet aliquet. Curabitur ac pretium lacus. Ut 
tempus, augue vitae tristique laoreet, lacus lorem ultrices libero, 
et pretium sapien leo facilisis massa. Nunc pretium lorem eget libero
hendrerit in pretium sapien convallis. Ut molestie, mi eget accumsan 
laoreet, lacus urna scelerisque metus, et luctus erat arcu non ante. 
Donec sit amet nisi felis, a interdum nibh.

Now: Aligned Equations:
\begin{align}
                    ax^2+bx+c &= 0\\
    \Leftrightarrow ax^2+bx   &= -c\\
    \Rightarrow x_{1,2} &= \frac{1}{2a} \cdot (-b \pm \sqrt{b^2 - 4ac})
\end{align}

Important is the ampersand \&. Spacing is irrelevant for rendering.
Spacing is only for the joy of reading through your \LaTeX{} code.

You can also make this equations without numberings. Add a asterisk (*):

\begin{align*}
                    ax^2+bx+c &= 0\\
    \Leftrightarrow ax^2+bx   &= -c\\
    \Rightarrow x_{1,2} &= \frac{1}{2a} \cdot (-b \pm \sqrt{b^2 - 4ac})
\end{align*}

Pellentesque commodo, nisi nec feugiat vehicula, augue erat convallis ipsum, adipiscing cursus purus dolor venenatis velit. Vivamus enim augue, lacinia in ultrices a, varius sit amet sapien. Aliquam enim velit, molestie vitae bibendum eget, laoreet at ante. Suspendisse pulvinar leo at nisi accumsan nec malesuada neque ullamcorper. Aenean quis mi lectus, quis porttitor nisi. Curabitur interdum luctus lectus et egestas. Aenean sapien ligula, aliquam sed fermentum id, blandit at orci. Integer a turpis ac tellus commodo suscipit. Vivamus massa orci, pharetra eu consequat eu, vulputate eget lacus. Suspendisse non justo arcu. Nullam lacus augue, dapibus vitae convallis a, consectetur at elit.
\end{document}
