Teilweise gelabelte Netzwerke sind allgegenwärtig. Publikationsdatenbanken
bzw. Wikipedia mit Kategorien, soziale Netzwerke mit Eigenschaften der 
Benutzer und Datenbanken zur Analyse des Kaufverhaltens mit 
Produkt oder Kundeneigenschaften als Labels. Da die Labels nur 
teilweise vorhanden sind, ist es wünschenswert die fehlenden Labels
zu ergänzen. Doch häufig sind diese Netzwerke viele $\num{10000}$ 
Knoten groß und ändern sich laufend. Außerdem stehen textuelle
Inhalte zu den Knoten bereit, die bei der Klassifikation genutzt 
werden können.

Der DYCOS-Algorithmus nutzt diese und kann auf große, dynamische
Netzwerken angewandt werden.

