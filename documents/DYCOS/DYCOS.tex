\documentclass[runningheads]{llncs}

%---- Sonderzeichen-------%
\usepackage {ngerman}
%---- Codierung----%
\usepackage[utf8]{inputenc}	% for Unix and Windows
\usepackage[T1]{fontenc}
\usepackage{graphicx}
\usepackage{url}
\usepackage{llncsdoc}
%----- Mathematischer Zeichenvorrat---%
\usepackage{amsmath}
\usepackage{amssymb}
\usepackage{enumerate}
% fuer die aktuelle Zeit
\usepackage{scrtime}
\usepackage{listings}
\usepackage{subfigure}
\usepackage{hyperref}
\usepackage{cite}
\usepackage{parskip}
\usepackage[framed,amsmath,thmmarks,hyperref]{ntheorem}
\usepackage{mystyle}

\setcounter{tocdepth}{3}
\setcounter{secnumdepth}{3}

\hypersetup{ 
  pdftitle    = {"Uber die Klassifizierung von Knoten in dynamischen Netzwerken mit Inhalt},
  pdfauthor   = {Martin Thoma}, 
  pdfkeywords = {DYCOS}
}

\begin{document}

\mainmatter
\title{\"Uber die Klassifizierung von Knoten in dynamischen Netzwerken mit Inhalt}
\titlerunning{\"Uber die Klassifizierung von Knoten in dynamischen Netzwerken mit Inhalt}
\author{Martin Thoma}
\authorrunning{Proseminar Netzwerkanalyse}
\institute{Betreuer: Christopher O\ss{}ner}
\date{17.01.2014}
\maketitle

\begin{abstract}%
Teilweise gelabelte Netzwerke sind allgegenwärtig. Publikationsdatenbanken
bzw. Wikipedia mit Kategorien, soziale Netzwerke mit Eigenschaften der 
Benutzer und Datenbanken zur Analyse des Kaufverhaltens mit 
Produkt oder Kundeneigenschaften als Labels. Da die Labels nur 
teilweise vorhanden sind, ist es wünschenswert die fehlenden Labels
zu ergänzen. Doch häufig sind diese Netzwerke viele $\num{10000}$ 
Knoten groß und ändern sich laufend. Außerdem stehen textuelle
Inhalte zu den Knoten bereit, die bei der Klassifikation genutzt 
werden können.

Der DYCOS-Algorithmus nutzt diese und kann auf große, dynamische
Netzwerken angewandt werden.


\end{abstract}
\clearpage

\section{Einleitung}
Der DYCOS-Algorithmus nutzt Random Walks im Graphen, startend 
von dem zu klassifizierenden Knoten $n$. Dabei wird pro Random Walk
gezählt, welche Klasse $K$ am häufigsten gesehen wird. Der Knoten $n$
wird dann als zu $K$ zugehörig klassifiziert.


\section{Abgrenzung}
Es gibt viele Knotenklassifizierungsalgorithmen. Im folgenden
werden einige von Ihnen mit dem DYCOS-Algorithmus verglichen und
unterschiedliche Eigenschaften der Algorithmen hervorgehoben.

Der MUCCA-Algorithmus aus \cite{zappella} sei der erste.


\section{DYCOS}
\subsection{Notation}
Im folgenden sei $\nodes$ die Menge der Knoten zum Zeitpunkt
$t$, $\labeledNodes \subseteq \nodes$ die Menge der Knoten
mit Label, $\edges$ die Kantenmenge.


\subsection{Inhalte}
\input{Inhalte}


\section{Ausblick}
Den DYCOS-Algorithmus kann in einigen Aspekten erweitert werden. So könnte man
vor der Auswahl des Vokabulars jedes Wort auf den Wortstamm zurückführen. Dafür
könnte zum Beispiel der in \cite{porter} vorgestellte Porter-Stemming-Algorithmus verwendet werden. Durch diese Maßnahme wird das Vokabular kleiner
gehalten wodurch mehr Artikel mit einander durch Vokabular verbunden werden
können. Außerdem könnte so der Gini-Koeffizient ein besseres Maß für die
Gleichheit von Texten werden.

Eine weitere Verbesserungsmöglichkeit besteht in der Textanalyse. Momentan ist
diese noch sehr einfach gestrickt und ignoriert die Reihenfolge von Wörtern
beziehungsweise Wertungen davon. So könnte man den DYCOS-Algorithmus in einem
sozialem Netzwerk verwenden wollen, in dem politische Parteiaffinität von
einigen Mitgliedern angegeben wird um die Parteiaffinität der restlichen
Mitglieder zu bestimmen. In diesem Fall macht es jedoch einen wichtigen
Unterschied, ob jemand über eine Partei gutes oder schlechtes schreibt.

Eine einfache Erweiterung des DYCOS-Algorithmus wäre der Umgang mit mehreren
Beschriftungen.

DYCOS beschränkt sich bei inhaltlichen Zweifachsprüngen auf die
Top-$q$-Wortknoten, also die $q$ ähnlichsten Knoten gemessen mit der
Aggregatanalyse, allerdings wurde bisher noch nicht untersucht, wie der
Einfluss von $q \in \mathbb{N}$ auf die Klassifikationsgüte ist.


% Normaler LNCS Zitierstil
%\bibliographystyle{splncs}
\bibliographystyle{itmalpha}
% TODO: �ndern der folgenden Zeile, damit die .bib-Datei gefunden wird
\bibliography{literatur}

\end{document}

