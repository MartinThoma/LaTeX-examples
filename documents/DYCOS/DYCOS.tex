\documentclass[runningheads]{llncs}

%---- Sonderzeichen-------%
\usepackage {ngerman}
%---- Codierung----%
\usepackage[utf8]{inputenc}	% for Unix and Windows
\usepackage[T1]{fontenc}
\usepackage{graphicx}
\usepackage{url}
\usepackage{llncsdoc}
%----- Mathematischer Zeichenvorrat---%
\usepackage{amsmath}
\usepackage{amssymb}
\usepackage{enumerate}
% fuer die aktuelle Zeit
\usepackage{scrtime}
\usepackage{listings}
\usepackage{subfigure}
\usepackage{hyperref}
\usepackage{mystyle}

\setcounter{tocdepth}{3}
\setcounter{secnumdepth}{3}

\hypersetup{ 
  pdftitle    = {�ber die Klassifizierung von Knoten in einem dynamischen Netzwerk mit Inhalt},
  pdfauthor   = {Martin Thoma}, 
  pdfkeywords = {DYCOS}
}

\begin{document}

\mainmatter
\title{\"Uber die Klassifizierung von Knoten in einem dynamischen Netzwerk mit Inhalt}
\titlerunning{Titel der Seminararbeit}
\author{Martin Thoma}
\authorrunning{Titel des Seminars}
\institute{Betreuer: Christopher O\ss{}ner}
\date{17.01.2014}
\maketitle

\begin{abstract}%
Teilweise gelabelte Netzwerke sind allgegenwärtig. Publikationsdatenbanken
bzw. Wikipedia mit Kategorien, soziale Netzwerke mit Eigenschaften der 
Benutzer und Datenbanken zur Analyse des Kaufverhaltens mit 
Produkt oder Kundeneigenschaften als Labels. Da die Labels nur 
teilweise vorhanden sind, ist es wünschenswert die fehlenden Labels
zu ergänzen. Doch häufig sind diese Netzwerke viele $\num{10000}$ 
Knoten groß und ändern sich laufend. Außerdem stehen textuelle
Inhalte zu den Knoten bereit, die bei der Klassifikation genutzt 
werden können.

Der DYCOS-Algorithmus nutzt diese und kann auf große, dynamische
Netzwerken angewandt werden.


\end{abstract}

\section{Einleitung}
Der DYCOS-Algorithmus nutzt Random Walks im Graphen, startend 
von dem zu klassifizierenden Knoten $n$. Dabei wird pro Random Walk
gezählt, welche Klasse $K$ am häufigsten gesehen wird. Der Knoten $n$
wird dann als zu $K$ zugehörig klassifiziert.


\section{Notation}
Im folgenden sei $\nodes$ die Menge der Knoten zum Zeitpunkt
$t$, $\labeledNodes \subseteq \nodes$ die Menge der Knoten
mit Label, $\edges$ die Kantenmenge.


\section{Inhalte}
\input{Inhalte}

% Normaler LNCS Zitierstil
%\bibliographystyle{splncs}
\bibliographystyle{itmalpha}
% TODO: �ndern der folgenden Zeile, damit die .bib-Datei gefunden wird
\bibliography{literatur}

\end{document}

