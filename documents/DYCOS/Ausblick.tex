Den sehr einfach aufgebauten DYCOS-Algorithmus kann man noch an
vielen Punkten verbessern. So könnte man vor der Auswahl des
Vokabulars jedes Wort auf den Wortstamm zurückführen.
Dafür könnte zum Beispiel der \todo{Wo steht was über den?}{Porter-Stemming-Algorithmus} verwendet 
werden. Durch diese Maßnahme wird das
Vokabular kleiner gehalten, mehr Artikel können mit einander
durch Vokabular verbunden werden und der Gini-Koeffizient wird ein
besseres Maß für die Gleichheit von Texten.

Eine weitere Verbesserungsmöglichkeit besteht in der textanalyse.
Momentan ist diese noch sehr einfach gestrickt und ignoriert die
Reihenfolge von Wortern beziehungsweise Wertungen davon. So könnte
man den DYCOS-Algorithmus in einem sozialem Netzwerk verwenden wollen,
in dem politische Parteiaffinität von einigen Mitgliedern angegeben
wird um die Parteiaffinität der restlichen Mitglieder zu bestimmen.
In diesem Fall macht es jedoch einen wichtigen Unterschied, ob jemand
über eine Partei gutes oder schlechtes schreibt.

Eine einfache Erweiterung des DYCOS-Algorithmus wäre der Umgang mit 
mehreren Labels.
