\documentclass[a4paper,10pt]{report}

% Definitions
\usepackage{lscape}
\usepackage[height=25cm]{geometry}
\usepackage{timetable}

\geometry{top=2cm,left=1cm,right=1cm,bottom=3cm}

\begin{document}
\thispagestyle{empty}
\begin{landscape}
\noindent\printheading{Stundenplan von Martin Thoma - 3. Semester}

% Define the layout of your time tables
\setslotsize{2.8cm}{0.3cm}
\setslotcount {5} {44}
\settopheight{3}
\settextframe{0.8mm}

% Retro
\setframetype[t]{1}
\seteventcornerradius{0pt}

% Print timestamps into event blocks
%\setprinttimestamps{2}

% Define event types
\defineevent{lecture}{0.0} {0.28}{1.0} {1.0}{1.0}{1.0}
\defineevent{exercise-course}    {1.0} {0.4} {0.2} {1.0}{1.0}{1.0}
\defineevent{tutorial}   {0.6} {0.8} {1.0} {1.0}{1.0}{1.0}
\defineevent{langcourse} {1.0} {0.4} {0.2} {1.0}{1.0}{1.0}
\defineevent{icpc}       {0.21}{0.5} {0.16}{1.0}{1.0}{1.0}

% Start the time table
\begin{timetable}
  \hours{8}{15}{1}
  \germandays{1}
  \event 1 {0800} {0930} {Analysis III}                           {Schmöger}             {10.21 Benz}      {lecture}
  \event 1 {0945} {1115} {Betriebssysteme}                        {Beigl}                {10.32 Nusselt}   {lecture}
  \event 1 {1400} {1530} {Digitaltechnik}                         {Asfour}               {50.35 HS a. F.}  {lecture}

  \event 2 {0945} {1115} {Betriebssysteme}                        {Beigl}                {20.40 HS 37}     {lecture}
  \event 2 {1545} {1715} {Algorithmen II}                         {Wagner}               {30.46 Neue Chemie} {lecture}

  \event 3 {1400} {1530} {Digitaltechnik}                         {Asfour}               {50.35 HS a. F.}  {lecture}

  \event 4 {1545} {1715} {Algorithmen II}                         {Wagner}               {30.21 Gerthsen}  {lecture}

  \event 5 {1130} {1300} {Analysis III}                           {Schmoeger}            {10.11 Hertz}     {lecture}
  \event 5 {1545} {1715} {Analysis III}                           {Bolleyer}             {10.21 Daimler}   {exercise-course}
\end{timetable}

\end{landscape}
\end{document}          
