%!TEX root = Programmierparadigmen.tex
\chapter{Java Bytecode}
\index{Java Bytecode|(}
\begin{definition}[Bytecode]\xindex{Bytecode}%
    Der Bytecode ist eine Sammlung von Befehlen für eine virtuelle Maschine.
\end{definition}

Bytecode ist unabhängig von realer Hardware.

\begin{definition}[Heap]\xindex{Heap}\xindex{Speicher!dynamischer}%
    Der dynamische Speicher, auch Heap genannt, ist ein Speicherbereich, aus dem
    zur Laufzeit eines Programms zusammenhängende Speicherabschnitte angefordert
    und in beliebiger Reihenfolge wieder freigegeben werden können.
\end{definition}

\textit{Activation Record} ist ein \textit{Stackframe}.\index{Activation Record|see{Stackframe}}
\section{Instruktionen}
\begin{table}[h]
    \begin{tabular}{p{6cm}|ll}
    \textbf{Beschreibung}                               & \textbf{int} & \textbf{float} \\ \hline
    Addition                                            & iadd         & fadd           \\
    Element aus Array auf Stack packen                  & iaload       & faload         \\
    Element aus Stack in Array speichern                & iastore      & fastore        \\
    Konstante auf Stack legen                           & iconst\_<i>  & fconst\_<f> \\
    Divide second-from top by top                       & idiv         & fdiv           \\
    Multipliziere die obersten beiden Zahlen des Stacks & imul         & fmul           \\
    \end{tabular}
\end{table}

\section{Polnische Notation}
\begin{definition}[Schreibweise von Rechenoperationen]
    Sei $f: A \times B \rightarrow C$ eine Funktion, $a \in A$ und $b \in B$.

    \begin{defenum}
        \item Die Schreibweise $a\ f\ b$ heißt \textbf{Infix-Notation}\xindex{Infix-Notation}.
        \item Die Schreibweise $f\ a\ b$ heißt \textbf{Präfixnotation}\xindex{Präfixnotation}
        \item Die Schreibweise $a\ b\ f$ heißt \textbf{Postfixnotation}\xindex{Postfixnotation}.
    \end{defenum}
\end{definition}

\textit{Polnische Notation}\index{Notation!polnische|see{Präfixnotation}} ist ein Synonym für die Präfixnotation.

\textit{Umgekehrte polnische Notation}\index{Notation!umgekehrte polnische|see{Postfixnotation}} ist ein Synonym für die Postfixnotation.

\begin{beispiel}[Schreibweise von Rechenoperationen]
    \begin{bspenum}
        \item $1 + 2$ nutzt die Infix-Notation.
        \item $f\ a\ b$ nutzt die polnische Notation.
        \item Wir der Ausdruck $1 + 2 \cdot 3$ in Infix-Notation ohne Operatoren-Präzedenz
              ausgewertet, so gilt: 
              \[1 + 2 \cdot 3 = 9\]
              Wird er mit Operatoren-Präzendenz ausgewertet, so gilt:
              \[1 + 2 \cdot 3 = 7\]
        \item Der Ausdruck
              \[1 + 2 \cdot 3 = 7\]
              entspricht
              \[+\ 1\ \cdot\ 2\ 3\]
              in der polnischen Notation und
              \[1\ 2\ 3\ \cdot\ +\]
              in der umgekehrten polnischen Notation.
    \end{bspenum}
\end{beispiel}

\begin{bemerkung}[Eigenschaften der Prä- und Postfixnotation]
    \begin{bemenum}
        \item Die Reihenfolge der Operanden kann beibehalten und gleichzeitig
              auf Klammern verzichtet werden, ohne dass sich das Ergebnis 
              verändert.
        \item Die Infix-Notation kann in einer Worst-Case Laufzeit von $\mathcal{O}(n)$,
              wobei $n$ die Anzahl der Tokens ist mittels des
              \textit{Shunting-yard-Algorithmus}\xindex{Shunting-yard-Algorithmus} in
              die umgekehrte Polnische Notation überführt werden.
    \end{bemenum}
\end{bemerkung}

\section{Weitere Informationen}
\begin{itemize}
    \item \url{http://cs.au.dk/~mis/dOvs/jvmspec/ref-Java.html}
    \item \href{http://scanftree.com/Data_Structure/prefix-postfix-infix-online-converter}{scanftree.com}:
          Infix $\leftrightarrow$ Postfix Konverter
\end{itemize}
\index{Java Bytecode|)}