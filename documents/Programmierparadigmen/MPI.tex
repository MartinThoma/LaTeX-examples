%!TEX root = Programmierparadigmen.tex
\chapter{MPI}
\index{MPI|(}

Message Passing Interface (kurz: MPI) ist ein Standard, 
der den Nachrichtenaustausch bei parallelen Berechnungen auf 
verteilten Computersystemen beschreibt.

\section{Erste Schritte}
\inputminted[numbersep=5pt, tabsize=4, frame=lines, label=hello-world.c]{c}{scripts/mpi/hello-world.c}

Das wird \texttt{mpicc hello-world.c} kompiliert.\\
Mit \texttt{mpirun -np 14 scripts/mpi/a.out} werden 14 Kopien des Programms
gestartet.

\section{Syntax}
\section{Beispiele}
\section{Weitere Informationen}
\begin{itemize}
    \item \url{http://www.open-mpi.org/}
\end{itemize}

\index{MPI|)}