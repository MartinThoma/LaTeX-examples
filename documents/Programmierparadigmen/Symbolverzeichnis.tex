%!TEX root = Programmierparadigmen.tex
\markboth{Symbolverzeichnis}{Symbolverzeichnis}
\chapter*{Symbolverzeichnis}
\addcontentsline{toc}{chapter}{Symbolverzeichnis}
%%%%%%%%%%%%%%%%%%%%%%%%%%%%%%%%%%%%%%%%%%%%%%%%%%%%%%%%%%%%%%%%%%%%%
% Reguläre Ausdrücke                                                %
%%%%%%%%%%%%%%%%%%%%%%%%%%%%%%%%%%%%%%%%%%%%%%%%%%%%%%%%%%%%%%%%%%%%%
\section*{Reguläre Ausdrücke}
$\emptyset\;\;\;$ Leere Menge\\
$\epsilon\;\;\;$ Das leere Wort\\
$\alpha, \beta\;\;\;$ Reguläre Ausdrücke\\
$L(\alpha)\;\;\;$ Die durch $\alpha$ beschriebene Sprache\\
$\begin{aligned}[t]
    L(\alpha | \beta)    &= L(\alpha) \cup L(\beta)\\
    L(\alpha \cdot \beta)&= L(\alpha) \cdot L(\beta)
\end{aligned}$\\
$L^0 := \Set{\varepsilon}\;\;\;$ Die leere Sprache\\
$L^{n+1} := L^n \circ L \text{ für } n \in \mdn_0\;\;\;$ Potenz einer Sprache\\
$\begin{aligned}[t]
    \alpha^+ &=& L(\alpha)^+ &=& \bigcup_{i \in \mdn} L(\alpha)^i\\
    \alpha^* &=& L(\alpha)^* &=& \bigcup_{i \in \mdn_0} L(\alpha)^i
\end{aligned}$

%%%%%%%%%%%%%%%%%%%%%%%%%%%%%%%%%%%%%%%%%%%%%%%%%%%%%%%%%%%%%%%%%%%%%
% Logik                                                             %
%%%%%%%%%%%%%%%%%%%%%%%%%%%%%%%%%%%%%%%%%%%%%%%%%%%%%%%%%%%%%%%%%%%%%
\section*{Logik}
$\mathcal{M} \models \varphi\;\;\;$ Im Modell $\mathcal{M}$ gilt das Prädikat $\varphi$.\\
$\psi \vdash \varphi\;\;\;$ Die Formel $\varphi$ kann aus der Menge der Formeln $\psi$ hergeleitet werden.\\
%%%%%%%%%%%%%%%%%%%%%%%%%%%%%%%%%%%%%%%%%%%%%%%%%%%%%%%%%%%%%%%%%%%%%
% Weiteres                                                          %
%%%%%%%%%%%%%%%%%%%%%%%%%%%%%%%%%%%%%%%%%%%%%%%%%%%%%%%%%%%%%%%%%%%%%
\section*{Weiteres}
$\bot\;\;\;$ Bottom\\