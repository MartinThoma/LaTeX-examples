\documentclass[runningheads]{llncs}

%---- Sonderzeichen-------%
\usepackage[utf8]{inputenc} % this is needed for umlauts
\usepackage[ngerman]{babel} % this is needed for umlauts
\usepackage[T1]{fontenc}    % this is needed for correct output of umlauts in pdf
%---- Codierung----%
\usepackage{graphicx}
\usepackage{url}
\usepackage{llncsdoc}
%----- Mathematischer Zeichenvorrat---%
\usepackage{amsmath}
\usepackage{amssymb}
\usepackage{enumerate}
% fuer die aktuelle Zeit
\usepackage{scrtime}
\usepackage{listings}
\usepackage{hyperref}
\usepackage{cite}
\usepackage{parskip}
\usepackage[framed,amsmath,thmmarks,hyperref]{ntheorem}
\usepackage{algorithm}
\usepackage[noend]{algpseudocode}
\usepackage{csquotes}
\usepackage[colorinlistoftodos]{todonotes}
\usepackage{subfig}         % multiple figures in one
\usepackage{caption}
\usepackage{tikz}
\usepackage{enumitem}
\usepackage[german,nameinlink]{cleveref}
\usepackage{braket}
\allowdisplaybreaks
\usetikzlibrary{backgrounds}
\usepackage[binary-units=true]{siunitx}
\usepackage{mystyle}

\setcounter{tocdepth}{3}
\setcounter{secnumdepth}{3}

\hypersetup{ 
  pdftitle    = {Über die Klassifizierung von Knoten in dynamischen Netzwerken mit textuellen Inhalten},
  pdfauthor   = {Martin Thoma}, 
  pdfkeywords = {DYCOS}
}

\begin{document}

\mainmatter
\title{Über die Klassifizierung von Knoten in dynamischen Netzwerken mit Inhalt}
\titlerunning{Über die Klassifizierung von Knoten in dynamischen Netzwerken mit Inhalt}
\author{Martin Thoma}
\authorrunning{Proseminar Netzwerkanalyse}
\institute{Betreuer: Christopher Oßner}
\date{17.01.2014}
\maketitle

\begin{abstract}%
Teilweise gelabelte Netzwerke sind allgegenwärtig. Publikationsdatenbanken
bzw. Wikipedia mit Kategorien, soziale Netzwerke mit Eigenschaften der 
Benutzer und Datenbanken zur Analyse des Kaufverhaltens mit 
Produkt oder Kundeneigenschaften als Labels. Da die Labels nur 
teilweise vorhanden sind, ist es wünschenswert die fehlenden Labels
zu ergänzen. Doch häufig sind diese Netzwerke viele $\num{10000}$ 
Knoten groß und ändern sich laufend. Außerdem stehen textuelle
Inhalte zu den Knoten bereit, die bei der Klassifikation genutzt 
werden können.

Der DYCOS-Algorithmus nutzt diese und kann auf große, dynamische
Netzwerken angewandt werden.


\end{abstract}

\section{Einleitung}
Der DYCOS-Algorithmus nutzt Random Walks im Graphen, startend 
von dem zu klassifizierenden Knoten $n$. Dabei wird pro Random Walk
gezählt, welche Klasse $K$ am häufigsten gesehen wird. Der Knoten $n$
wird dann als zu $K$ zugehörig klassifiziert.


\section{Related Work}
%!TEX root = Ausarbeitung-Thoma.tex
Sowohl das Problem der Knotenklassifikation, als auch das der
Textklassifikation, wurde bereits in verschiedenen Kontexten analysiert. Jedoch
scheinen bisher entweder nur die Struktur des zugrundeliegenden Graphen oder
nur Eigenschaften der Texte verwendet worden zu sein.

So werden in \cite{bhagat,szummer} unter anderem Verfahren zur
Knotenklassifikation beschrieben, die wie der in \cite{aggarwal2011}
vorgestellte DYCOS-Algorithmus, um den es in dieser Ausarbeitung geht, auch auf
Random Walks basieren.

Obwohl es auch zur Textklassifikation einige Paper gibt
\cite{Zhu02learningfrom,Jiang2010302}, geht doch keines davon auf den
Spezialfall der Textklassifikation mit einem zugrundeliegenden Graphen ein.

Die vorgestellten Methoden zur Textklassifikation variieren außerdem sehr
stark. Es gibt Verfahren, die auf dem bag-of-words-Modell basieren
\cite{Ko:2012:STW:2348283.2348453} wie es auch im DYCOS-Algorithmus verwendet
wird. Aber es gibt auch Verfahren, die auf dem
Expectation-Maximization-Algorithmus basieren \cite{Nigam99textclassification}
oder Support Vector
Machines nutzen \cite{Joachims98textcategorization}.

Es wäre also gut Vorstellbar, die Art und Weise wie die Texte in die
Klassifikation des DYCOS-Algorithmus einfließen zu variieren. Allerdings ist
dabei darauf hinzuweisen, dass die im Folgenden vorgestellte Verwendung der
Texte sowohl einfach zu implementieren ist und nur lineare Vorverarbeitungszeit
in Anzahl der Wörter des Textes hat, als auch es erlaubt einzelne Knoten zu
klassifizieren, wobei der Graph nur lokal um den zu klassifizierenden Knoten
betrachten werden muss.


\section{DYCOS}
\subsection{Notation}
Im folgenden sei $\nodes$ die Menge der Knoten zum Zeitpunkt
$t$, $\labeledNodes \subseteq \nodes$ die Menge der Knoten
mit Label, $\edges$ die Kantenmenge.


\subsection{Inhalte}
\input{Inhalte}


\section{Analyse des DYCOS-Algorithmus}
Für den DYCOS-Algorithmus wurde in \cite{aggarwal2011} bewiesen,
dass sich nach Ausführung von DYCOS für einen unbeschrifteten
Knoten mit einer Wahrscheinlichkeit von höchstens
$(1-k)\cdot e^{-l \cdot b^2 / 2}$ eine Knotenbeschriftung ergibt, deren
relative Häufigkeit weniger als $b$ der häufigsten Beschriftung ist.
Dabei ist $k$ die Anzahl der Klassen und $l$ die Länge der 
Random-Walks.

Außerdem wurde experimentell anhand des DBLP-Datensatzes\footnote{http://dblp.uni-trier.de/}
und des CORA-Datensatzes\footnote{http://people.cs.umass.edu/~mccallum/data/cora-classify.tar.gz}
gezeigt, dass die Klassifikationsgüte nicht wesentlich von der Anzahl der Wörter mit
höchstem Gini-Koeffizient $m$ abhängt.  Obwohl es sich nicht sagen lässt,
wie genau die Ergebnisse aus \cite{aggarwal2011} zustande gekommen sind,
eignet sich das Kreuzvalidierungsverfahren zur Bestimmung der Klassifikationsgüte
wie es in \cite{Lavesson,Stone1974} vorgestellt wird:
\begin{enumerate}
    \item Betrachte nur $V_{L,T}$.
    \item Unterteile $V_{L,T}$ zufällig in $k$ disjunkte Mengen $M_1, \dots, M_k$.
    \item \label{schritt3} Teste die Klassifikationsgüte, wenn die Knotenbeschriftungen
          aller Knoten in $M_i$ für DYCOS verborgen werden für $i=1,\dots, k$.
    \item Bilde den Durchschnitt der Klassifikationsgüten aus \cref{schritt3}.
\end{enumerate}

Es wird $k=10$ vorgeschlagen.




\section{Probleme des DYCOS-Algorithmus}
Bei der Anwendung des in \cite{aggarwal2011} vorgestellten Algorithmus auf
reale Datensätze könnten zwei Probleme auftreten, die im Folgenden erläutert
werden. Außerdem werden Verbesserungen vorgeschlagen, die es allerdings noch zu
untersuchen gilt.

\subsection{Anzahl der Knotenbeschriftungen}
So, wie der DYCOS-Algorithmus vorgestellt wurde, können nur Graphen bearbeitet
werden, deren Knoten jeweils höchstens eine Beschriftung haben. In vielen
Fällen, wie z.~B. Wikipedia mit Kategorien als Knotenbeschriftungen haben
Knoten jedoch viele Beschriftungen.

Auf einen ersten Blick ist diese Schwäche einfach zu beheben, indem man beim
zählen der Knotenbeschriftungen für jeden Knoten jedes Beschriftung zählt. Dann
wäre noch die Frage zu klären, mit wie vielen Beschriftungen der betrachtete
Knoten beschriftet werden soll.

Jedoch ist z.~B. bei Wikipedia-Artikeln auf den Knoten eine Hierarchie
definiert. So ist die Kategorie \enquote{Klassifikationsverfahren} eine
Unterkategorie von \enquote{Klassifikation}. Bei dem Kategorisieren von
Artikeln sind möglichst spezifische Kategorien vorzuziehen, also kann man nicht
einfach bei dem Auftreten der Kategorie \enquote{Klassifikationsverfahren}
sowohl für diese Kategorie als auch für die Kategorie \enquote{Klassifikation}
zählen.


\subsection{Überanpassung und Reklassifizierung}
Aggarwal und Li beschreiben in \cite{aggarwal2011} nicht, auf welche Knoten der
Klassifizierungsalgorithmus angewendet werden soll. Jedoch ist die Reihenfolge
der Klassifizierung relevant. Dazu folgendes Minimalbeispiel:

Gegeben sei ein dynamischer Graph ohne textuelle Inhalte. Zum Zeitpunkt $t=1$
habe dieser Graph genau einen Knoten $v_1$ und $v_1$  sei mit dem $A$
beschriftet. Zum Zeitpunkt $t=2$ komme ein nicht beschrifteter Knoten $v_2$
sowie die Kante $(v_2, v_1)$ hinzu.\\
Nun wird der DYCOS-Algorithmus auf diesen Knoten angewendet und $v_2$ mit $A$
beschriftet.\\
Zum Zeitpunkt $t=3$ komme ein Knoten $v_3$, der mit $B$ beschriftet ist, und
die Kante $(v_2, v_3)$ hinzu. \Cref{fig:Formen} visualisiert diesen Vorgang.

\begin{figure}[ht]
    \centering
    \subfloat[$t=1$]{
        \tikzstyle{vertex}=[draw,black,circle,minimum size=10pt,inner sep=0pt]
\tikzstyle{edge}=[very thick]
\begin{tikzpicture}[scale=1,framed]
    \node (a)[vertex,label=$A$] at (0,0) {$v_1$};
    \node (b)[vertex, white] at (1,0) {$v_2$};
    \node (struktur)[label={[label distance=-0.2cm]0:$t=1$}] at (-2,1) {};
\end{tikzpicture}

        \label{fig:graph-t1}
    }%
    \subfloat[$t=2$]{
        \tikzstyle{vertex}=[draw,black,circle,minimum size=10pt,inner sep=0pt]
\tikzstyle{edge}=[very thick]
\begin{tikzpicture}[scale=1,framed]
    \node (a)[vertex,label=$A$] at (0,0) {$v_1$};
    \node (b)[vertex,label={\color{blue}$A$}] at (1,0) {$v_2$};
    \draw[->] (b) -- (a);
    \node (struktur)[label={[label distance=-0.2cm]0:$t=2$}] at (-2,1) {};
\end{tikzpicture}

        \label{fig:graph-t2}
    }

    \subfloat[$t=3$]{
        \tikzstyle{vertex}=[draw,black,circle,minimum size=10pt,inner sep=0pt]
\tikzstyle{edge}=[very thick]
\begin{tikzpicture}[scale=1,framed]
    \node (a)[vertex,label=$A$] at (0,0) {$v_1$};
    \node (b)[vertex,label={\color{blue}$A$}] at (1,0) {$v_2$};
    \node (c)[vertex,label=$B$] at (2,0) {$v_3$};
    \draw[->] (b) -- (a);
    \draw[->] (b) -- (c);
    \node (struktur)[label={[label distance=-0.2cm]0:$t=3$}] at (-1,1) {};
\end{tikzpicture}

        \label{fig:graph-t3}
    }%
    \subfloat[$t=4$]{
        \tikzstyle{vertex}=[draw,black,circle,minimum size=10pt,inner sep=0pt]
\tikzstyle{edge}=[very thick]
\begin{tikzpicture}[scale=1,framed]
    \node (a)[vertex,label=$A$] at (0,0) {$v_1$};
    \node (b)[vertex,label=45:{\color{blue}$A$}] at (1,0) {$v_2$};
    \node (c)[vertex,label=$B$] at (2,0) {$v_3$};
    \node (d)[vertex] at (1,1) {$v_4$};
    \draw[->] (b) -- (a);
    \draw[->] (b) -- (c);

    \draw[->] (d) -- (a);
    \draw[->] (d) -- (b);
    \draw[->] (d) -- (c);
    \node (struktur)[label={[label distance=-0.2cm]0:$t=3$}] at (-1,1) {};
\end{tikzpicture}

        \label{fig:graph-t4}
    }%
    \caption{Minimalbeispiel für den Einfluss früherer DYCOS-Anwendungen}
    \label{fig:Formen}
\end{figure}

Würde man nun den DYCOS-Algorithmus erst jetzt, also anstelle von Zeitpunkt
$t=2$ zum Zeitpunkt $t=3$ auf den Knoten $v_2$ anwenden, so würde eine
\SI{50}{\percent}-Wahrscheinlichkeit bestehen, dass dieser mit $B$ beschriftet
wird. Aber in diesem Beispiel wurde der Knoten schon zum Zeitpunkt $t=2$
beschriftet. Obwohl es in diesem kleinem Beispiel noch keine Rolle spielt, wird
das Problem klar, wenn man weitere Knoten einfügt:

Wird zum Zeitpunkt $t=4$ ein unbeschrifteter Knoten $v_4$ und die Kanten
$(v_1, v_4)$, $(v_2, v_4)$, $(v_3, v_4)$ hinzugefügt, so ist die
Wahrscheinlichkeit, dass $v_4$ mit $A$ beschriftet wird bei $\frac{2}{3}$.
Werden die unbeschrifteten Knoten jedoch erst jetzt und alle gemeinsam
beschriftet, so ist die Wahrscheinlichkeit für $A$ als Beschriftung bei nur $50\%$.
Bei dem DYCOS-Algorithmus findet also eine Überanpassung an vergangene
Beschriftungen statt.

Das Reklassifizieren von Knoten könnte eine mögliche Lösung für dieses
Problem sein. Knoten, die durch den DYCOS-Algorithmus beschriftet wurden
könnten eine Lebenszeit bekommen (TTL, Time to Live). Ist diese
abgelaufen, wird der DYCOS-Algorithmus erneut auf den Knoten angewendet.


\section{Ausblick}
Den DYCOS-Algorithmus kann in einigen Aspekten erweitert werden. So könnte man
vor der Auswahl des Vokabulars jedes Wort auf den Wortstamm zurückführen. Dafür
könnte zum Beispiel der in \cite{porter} vorgestellte Porter-Stemming-Algorithmus verwendet werden. Durch diese Maßnahme wird das Vokabular kleiner
gehalten wodurch mehr Artikel mit einander durch Vokabular verbunden werden
können. Außerdem könnte so der Gini-Koeffizient ein besseres Maß für die
Gleichheit von Texten werden.

Eine weitere Verbesserungsmöglichkeit besteht in der Textanalyse. Momentan ist
diese noch sehr einfach gestrickt und ignoriert die Reihenfolge von Wörtern
beziehungsweise Wertungen davon. So könnte man den DYCOS-Algorithmus in einem
sozialem Netzwerk verwenden wollen, in dem politische Parteiaffinität von
einigen Mitgliedern angegeben wird um die Parteiaffinität der restlichen
Mitglieder zu bestimmen. In diesem Fall macht es jedoch einen wichtigen
Unterschied, ob jemand über eine Partei gutes oder schlechtes schreibt.

Eine einfache Erweiterung des DYCOS-Algorithmus wäre der Umgang mit mehreren
Beschriftungen.

DYCOS beschränkt sich bei inhaltlichen Zweifachsprüngen auf die
Top-$q$-Wortknoten, also die $q$ ähnlichsten Knoten gemessen mit der
Aggregatanalyse, allerdings wurde bisher noch nicht untersucht, wie der
Einfluss von $q \in \mathbb{N}$ auf die Klassifikationsgüte ist.


% Normaler LNCS Zitierstil
%\bibliographystyle{splncs}
\bibliographystyle{itmalpha}
% TODO: Ändern der folgenden Zeile, damit die .bib-Datei gefunden wird
\bibliography{literatur}

\end{document}

