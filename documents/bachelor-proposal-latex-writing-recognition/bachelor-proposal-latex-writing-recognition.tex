\documentclass[a4paper]{scrartcl}
\usepackage{amssymb, amsmath} % needed for math
\usepackage[utf8]{inputenc} % this is needed for umlauts
\usepackage[english]{babel} % this is needed for umlauts
\usepackage[T1]{fontenc}    % this is needed for correct output of umlauts in pdf
\usepackage[margin=2.5cm]{geometry} %layout
\usepackage{hyperref}   % links im text
\usepackage{color}
\usepackage{framed}
\usepackage{enumerate}  % for advanced numbering of lists
\usepackage{csquotes}
\usepackage{ifxetex,ifluatex}
\usepackage{etoolbox}
\usepackage[svgnames]{xcolor}
\usepackage{tikz}
\usepackage{framed}
\usepackage{parskip}
\usepackage{cite}
\usepackage{fancyref}
\usepackage{mystyle}
\clubpenalty  = 10000   % Schusterjungen verhindern
\widowpenalty = 10000   % Hurenkinder verhindern

\hypersetup{ 
  pdfauthor   = {Martin Thoma}, 
  pdfkeywords = {Bachelor proposal, LaTeX, handwriting recognition}, 
  pdftitle    = {Proposal for a Bachelor of Science Thesis:\\Interactive on-line handwriting recognition of mathematical formulae} 
} 

%%%%%%%%%%%%%%%%%%%%%%%%%%%%%%%%%%%%%%%%%%%%%%%%%%%%%%%%%%%%%%%%%%%%%

\begin{document}
    \title{Proposal for a Bachelor of Science Thesis:\\Interactive on-line handwriting recognition of mathematical formulae}
    \author{Martin Thoma}
    \maketitle
\section{The problem backgound}
    There are people who don't know how to write even
    simple mathematical formulae with \LaTeX{} like
    \[\pi/\alpha=\sum_{n=-\infty}^\infty \frac{\sin^2 (c+n)\alpha}{(c+n)^2}=\int_{-\infty}^\infty \frac{\sin^2 (c+n)\alpha}{(c+n)^2}\, \text{d}n\]
    or who need much time to do so. Currently, there are several online
    services, programms and apps that help to write mathematical 
    formulae, but all programms I know have serious disadvantages:
    \begin{itemize}
        \item \href{http://detexify.kirelabs.org/classify.html}{detexify.kirelabs.org}
              recognizes \textbf{only symbols},
        \item the formel editor of LibreOffice Writer 3.6 as showen 
              in \Fref{fig:libre-office-3.6} offers some
              guidiance by grouping common operations while showing
              a WYSIWYG editor, but it has \textbf{no handwriting recognition}.
              Another drawback is the fact that it is \textbf{not available 
              as an online service}, so you have to install LibreOffice
              which might not be possible on all devices.
        \item The \enquote{Daum Equation Editor} (see \Fref{fig:daum-editor}) is available online
              and offers guidiance through the creation of equations,
              but does not offer handwriting recognition. Although
              it might be OpenSource, the \textbf{source code is difficult to
              find}. This means if you want to improve the recognition,
              it is not possible. It also makes use of Adobe Flash 
              which is not available on many smartphones and tablet
              computers.
        \item Maple seems to offer handwritten symbol recognition (\href{http://www.maplesoft.com/products/maple/features/handwritten.aspx}{source}),
              but on the one hand I was not able to test that, because
              it is \textbf{not available for free}. On the other hand you 
              have to install additional software, it seems not to be
              available for tablet computers and it does only recognize
              single symbols.
        \item Wolfram Mathematica seems to be able to do complete
              formula recognition at least for simple formulae (\href{http://reference.wolfram.com/mathematica/tutorial/HandwrittenMathRecognition.html}{source})
              by using Microsofts \href{http://windows.microsoft.com/en-ph/windows7/use-math-input-panel-to-write-and-correct-math-equations}{Math Input Panel},
              but this is neither OpenSource nor available as an
              online service. Additionally it is not
              available for Linux systems, so I can't test it.
    \end{itemize}

    A more comprehensive list can be found at \href{https://en.wikipedia.org/wiki/Formula_editor}{https://en.wikipedia.org/wiki/Formula\_editor}.
    A problem of some of the projects presented there is that they
    require the client to execute Java Applets which is a security 
    risk.

    \begin{figure}[h]
        \centering
        \includegraphics*[width=5cm, keepaspectratio]{figures/libreoffice-writer.png}
        \caption{LibreOffice Writer 3.6 - Formel Editor}
        \label{fig:libre-office-3.6}
    \end{figure}

    \begin{figure}[h]
        \centering
        \includegraphics*[width=15cm, keepaspectratio]{figures/daum-editor.png}
        \caption{Daum Equation editor}
        \label{fig:daum-editor}
    \end{figure}
\break
\section{The problem statement}
    What I would like to have is an interactive on-line handwriting
    recognition service, that is available as a web service which makes
    use of touchscreens. Additionally, it should be for free and 
    OpenSource, the source code should be easy to find and documented.
    This means:
    \begin{itemize}
        \item \textbf{Service}: The program can be accessed over the web, so
              that the user does only have to have a modern browser. 
              As a consequence, the software could be used with any
              device that has a touch screen.
        \item \textbf{On-line handwriting recognition}: The service
              starts recognizing while the user enters a formula.
        \item \textbf{Interactive}: The service offers symbols and constructs
              to the user before the user starts typing. These suggestions
              might chage depending on what the user has typed before.
        \item \textbf{OpenSource}: Any license in this list: \href{http://opensource.org/licenses}{http://opensource.org/licenses}
        \item \textbf{Easy to find}: Ideally, the project should have
              an own domain that contains the source code, the service
              and documentation. But it might be enough to provide
              an email address to a developer within the top of
              of the source code of the delivered HTML document.
    \end{itemize}

    This service should also encourage the users by techniques
    of \enquote{gamification} to give as much
    meta information about their formulae as possible:
    \begin{itemize}
        \item Which problem domain does the formula belong to, e.~g. \enquote{Euclidean geometry}, \enquote{analysis} or \enquote{calculus}?
        \item Does the formula itself have a name, e.~g. \enquote{Pythagorean theorem}, \enquote{Fibonacci numbers} or \enquote{geometric series}?
    \end{itemize}

    This information should be used to create a formula database.

\section{Significance}
For me as a Linux user, there no software that I can test and which 
offers on-line, interactive math handwriting recognition. But the
need of such a software is there.

But there are more reasons why this bachelor's thesis matters:
Projects like \LaTeX{}, Linux, Apache or FireFox have shown that
OpenSoure software can enrich the develpment in specific areas. The
\enquote{Browser Wars} might be the most famous result of an active
OpenSource community. Internet Explorer 6 had
a market share of over 80\% in 2003. Prequels of Firefox and the Mozilla 
foundation already existed, but Firefox 1.0 was released not until
November 2004. After that, Firefox and other open browsers added many
features that Internet Explorer had to compete with, like tabbed browsing,
HTML4 standard conformance, support of the \texttt{<canvas>} tag and
speed of HTML rendering and JavaScript execution.\footnote{\href{http://www.evolutionoftheweb.com/}{www.evolutionoftheweb.com} offers a graphical overview. Although supporting standards like HTML4 or CSS~2 is not done with one version, but rather an incremental process.} Some of these
questions are interesting for science such as many problems related
to layouts and just-in-time compilation (JIT). With OpenSource software
that makes it easy to find its source and offers good documentation,
researchers can simply try their ideas without being blocked by 
having to try to access the source code.

Additionally, such a project might give researchers more time to
concentrate on the tasks they really want to do rather than spending
hours by learning \LaTeX{}.

One last reason why this thesis matters is the formula database that
gets created by users. This database might be used in follow-up work,
e.~g. a formula spotter for presentations or a math detector for speech.

\section{Time schedule}
\begin{itemize}
    \item[70h] Literature research about on-line handwriting recognition 
               techniques and gamification.
    \item[5h]  Defining browsers and devices that should get supported
               and required client side software like HTML5, CSS 3
               and ECMAScript (better known as JavaScript). Also,
               required input methods like touchscreens and stylus
               should be mentioned.
    \item[20h] Writing use cases. This is includes writing example 
               formula that the user shoud type and the system should
               be able to recognize; finding people with different 
               knowledge of \LaTeX{} and from different fields who 
               want to participate in user tests.
    \item[60h] Implementing the core of the application: Handwriting
               recognition of digits and symbols by using only
               HTML, CSS and  on the client side. This includes implementing
               a way for the user to enter new symbols and to correct the
               symbol that was suggested by the recognition system.
    \item[20h] Introduce testers that already know \LaTeX{} to the 
               current system. At this point, the system does only do
               symbol recognition. The testers should train it, 
               insert symbols like $a-z, A-Z, 0-9, \alpha-\omega, A-\Omega, \cdot, \circ, \dots$
    \item[10h] Get feedback by the users. This feedback will not be included
               in the thesis, but the improvements will get documented.
    \item[60h] Finding structures and ways how to enter them. Examples
               of structures that can be nested are sums:
               \begin{verbatim}\sum_{<some structure>}^{<another strcuture>} <a third structure>\end{verbatim}
               Implement the recognition of those strucutres.
    \item[30h] Observe \enquote{fresh} testers while they try to use
               the system. 
    \item[70h] Improving the software to fix problems that were found
               with user tests
    \item[50h] Fix bugs, improve code quality and readability as well
               as documentation.
    \item[45h] Usability testing: Try Hallway testing. The results
               of these tests get documented and will be part of the
               bachelor's thesis. If possible, I would like
               to let the testers use their own devices.
    \item[10h] Mentioning open questions and ideas how they could be
               analyzed with the service that was created.
\end{itemize}

\section{Outline}
I have described in which steps I would like to write the software, 
but almost all points include writing the bachelor's thesis document.
A first draft of the outline could be like this:

\begin{enumerate}
    \item Introduction
    \item Definitions
    \begin{enumerate}
        \item Hardware: What is available and what is the distribution?
        \item Software: What is available and what is the distribution?
        \item Support of standards like HTML, CSS, ECMA-Script, Flash, Cookies, ...
        \item Choice of hardware, software and standards that should get supported as well as the choice of Libraries and the required server-side software
        \item Application to the domain of math recognition
    \end{enumerate}
    \item On-line handwriting techniques
    \begin{enumerate}
        \item Description of techniques in general
        \item Application to the domain of math recognition
    \end{enumerate}
    \item Gamification techniques
    \begin{enumerate}
        \item Description of techniques in general
        \item Application to the domain of math recognition in the web
    \end{enumerate}
    \item Software Project
    \begin{enumerate}
        \item Structure of the code
        \item Availability of documentation
        \item Availability of the service
    \end{enumerate}
    \item Summary
    \begin{enumerate}
        \item Future Work
    \end{enumerate}
\end{enumerate}
\break

\renewcommand\refname{Related Literature}
\nocite{*}
\bibliographystyle{itmalpha}
\bibliography{literatur}

This literature list is only a list that seems to make sense to me
by now. As I proceed I might find more usefull sources for the different
topics. So I might add, but also remove elements from this list.
Especially for gamification I might read documents from
\href{http://gamification-research.org/}{gamification-research.org}.
\end{document}
