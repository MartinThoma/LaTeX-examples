\documentclass[a4paper]{scrreprt}
\usepackage[top=2.5cm,bottom=2.5cm,left=2.5cm,right=2.5cm]{geometry}
\usepackage{amssymb, amsmath} % needed for math
\usepackage[utf8]{inputenc} % this is needed for umlauts
\usepackage[ngerman]{babel} % this is needed for umlauts
\usepackage[T1]{fontenc}    % this is needed for correct output of umlauts in pdf
\usepackage{parskip}
\usepackage{paperandpencil}

\title{Minimal distance to a cubic function}
\author{Martin Thoma}

%%%%%%%%%%%%%%%%%%%%%%%%%%%%%%%%%%%%%%%%%%%%%%%%%%%%%%%%%%%%%%%%%%%%%
% Begin document                                                    %
%%%%%%%%%%%%%%%%%%%%%%%%%%%%%%%%%%%%%%%%%%%%%%%%%%%%%%%%%%%%%%%%%%%%%
\begin{document}

\question{\bf Wie lautet der Name ihres (ehemaligen) Studienganges?}
\linetext{Studiengang}

\question{\bf In welchen Jahr wurden Sie geboren?}
\linetext{Geburtsjahr}

\question{\bf Wie viele wissenschaftliche Arbeiten (Abschlussarbeiten, Artikel für Konferenzen und ähnliche) haben Sie geschrieben?}
\begin{answersC}
\item 0 Arbeit
\item 1 Arbeit
\item 2 Arbeiten
\item 3--5 Arbeiten
\end{answersC}

\question{\bf Mussten Sie schon mal Formeln in den Computer eingeben?}
\begin{linkanswers}
\framedlink{Ja}{\em Bitte weiter mit Frage 5}
\framedlink{Nein}{\em Fragebogen komplett beantwortet.Vielen Dank!}
\end{linkanswers}

\question{\bf Schreiben Sie drei beliebige Formeln in folgende Felder:}
\fbox{\parbox[b][4em][t]{\textwidth}{1} }
\fbox{\parbox[b][4em][t]{\textwidth}{2} }
\fbox{\parbox[b][4em][t]{\textwidth}{3} }

\question{\bf Wie häufig müssen Sie in einem typischen Monat Formeln in den Computer eingeben?}
\begin{answersC}
\item nahezu täglich
\item etwa jeden zweiten Tag
\item etwa einmal in der Woche
\item seltener
\end{answersC}
\clearpage

\question{\bf Wie haben Sie gelernt, wie man Formeln eingibt?}
(Mehrfachantworten möglich)
\begin{longanswersC}
\item Mit einem oder mehreren Lehrern / Dozenten
\item Im Selbststudium über das Internet, und zwar \linetext{Website}
\item Mit einem oder mehreren Büchern, und zwar \linetext{Buchtitel}
\item Mit Freunden
\end{longanswersC}

\question{\bf Welche Strategien haben Sie, um herauszufinden, wie Formeln / Sybmole
          in den Computer eingegeben werden?}
(Mehrfachantworten möglich)
\begin{longanswersC}
\item Mit einer Website, und zwar \linetext{Website} \hfill\ebigbox{Keine}
\item Mit einem Buch, und zwar \linetext{Buchtitel}
\item Mit den \LaTeX-Hilfsdateien
\item Mit einer Sybmol-Tabelle
\item Ich frage einen Freund / eine Freundin
\end{longanswersC}

\question{\bf Welches Programm nutzen Sie, wenn Sie Formeln eingeben?}
(Mehrfachantworten möglich)
\begin{longanswersC}
\item Microsoft Word
\item OpenOffice
\item LibreOffice
\item LaTeX
\item Sonstiges, und zwar \linetext{Programm}
\end{longanswersC}

\question{\bf Wie geben Sie Ihre Formeln ein?}

\vertikalblockfive{so gut wie nie}{selten}{manchmal}{häufig}{so gut wie immer}{
\blocktextfive{Tastatur}
\blocktextfive{Maus}
\blocktextfive{Trackball}
\blocktextfive{Smartphone}}

\question{\bf Wie zufrieden sind sie mit der Art Ihrer Eingabe?}
\htextlinesix{sehr unzufrieden}{eher unzufrieden}{unentschieden}{eher zufrieden}{sehr zufrieden}{keine Angabe}


\question{\bf Was stört Sie bei der Eingabe von Formeln bisher?}
\vspace{3cm}

\question{\bf Haben Sie eine Idee, wie man Probleme bei der Eingabe von Formeln beheben könnte bzw. was bei der Formeleingabe verbessert werden könnte?}
\vspace{3cm}

\question{\bf Haben Sie bereits Formeln über ein Touch-Gerät eingegeben?}
\begin{answersC}
\item Ja, und zwar mit \linetext{Name des Toch-Gerätes}
\item Nein
\end{answersC}

\question{\bf Mit welchen Geräten können Sie sich vorstellen Formeln einzugeben?}

\textit{Hinweis}: Ein Stylus ist ein stiftförmiges Eingabegerät für Touch-Geräte,
das nur mit dem Touch-Gerät verwendbar ist. Ein Smartpen hingegen wird ohne
Touch-Gerät und mit Papier verwendet.

\vertikalblockfive{überhaupt nicht}{eher nicht}{unent\-schieden}{eher schon}{auf jeden Fall}{
\blocktextfive{Tablet (Touch)}
\blocktextfive{Smartphone (Touch)}
\blocktextfive{Touch-Bildschirm}\hline
\blocktextfive{Tablet (Stylus)}
\blocktextfive{Smartphone (Stylus)}\hline
\blocktextfive{Smartpen}
\blocktextfive{Grafiktablett}}

\end{document}
