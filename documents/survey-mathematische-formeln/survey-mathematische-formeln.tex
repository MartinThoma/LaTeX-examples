\documentclass[a4paper]{scrreprt}
\usepackage[top=2.5cm,bottom=2.5cm,left=2.5cm,right=2.5cm]{geometry}
\usepackage{amssymb, amsmath} % needed for math
\usepackage[utf8]{inputenc} % this is needed for umlauts
\usepackage[ngerman]{babel} % this is needed for umlauts
\usepackage[T1]{fontenc}    % this is needed for correct output of umlauts in pdf
\usepackage{parskip}
\usepackage{paperandpencil}

\title{Minimal distance to a cubic function}
\author{Martin Thoma}

%%%%%%%%%%%%%%%%%%%%%%%%%%%%%%%%%%%%%%%%%%%%%%%%%%%%%%%%%%%%%%%%%%%%%
% Begin document                                                    %
%%%%%%%%%%%%%%%%%%%%%%%%%%%%%%%%%%%%%%%%%%%%%%%%%%%%%%%%%%%%%%%%%%%%%
\begin{document}

\question{\bf Wie lautet der Name ihres (ehemaligen) Studienganges?}
\linetext{Studiengang}

\question{\bf In welchen Jahr wurden Sie geboren?}
\linetext{Geburtsjahr}

\question{\bf Wie viele wissenschaftliche Arbeiten (Abschlussarbeiten, Artikel
für Konferenzen und ähnliches) haben Sie geschrieben?}
\begin{answersC}
\item 0 Arbeiten
\item 1 Arbeit
\item 2 Arbeiten
\item 3--5 Arbeiten
\item 6 oder mehr Arbeiten
\end{answersC}

\question{\bf Mussten Sie schon mal Formeln in den Computer eingeben?}
\begin{linkanswers}
\framedlink{Ja}{\em Bitte weiter mit Frage 5}
\framedlink{Nein}{\em Fragebogen komplett beantwortet.Vielen Dank!}
\end{linkanswers}

\question{\bf Wie häufig müssen Sie in einem typischen Monat Formeln in den Computer eingeben?}
\begin{answersC}
\item nahezu täglich
\item etwa jeden zweiten Tag
\item etwa einmal in der Woche
\item seltener
\end{answersC}
\clearpage

\question{\bf Wie haben Sie gelernt, wie man Formeln eingibt?}
\begin{longanswersC}
\item Mit Lehrer(n) / Dozent(en)
\item Mit Freund(in/en)
\item Im Selbststudium über das Internet, und zwar \linetext{Website}
\item Mit einem oder mehreren Büchern, und zwar \linetext{Buchtitel}
\end{longanswersC}

\question{\bf Welche Strategien haben Sie, um herauszufinden, wie Formeln / Sybmole
          in den Computer eingegeben werden?}
\begin{longanswersC}
\item Keine
\item Mit einer Website, und zwar \linetext{Website}
\item Mit einem Buch, und zwar \linetext{Buchtitel}
\item Mit den \LaTeX-Hilfsdateien
\item Mit einer Sybmol-Tabelle
\item Ich frage einen Freund / eine Freundin
\end{longanswersC}

\question{\bf Welches Programm nutzen Sie, wenn Sie Formeln eingeben?}
\begin{longanswersC}
\item Microsoft Word
\item OpenOffice
\item LibreOffice
\item \LaTeX
\item Sonstiges, und zwar \linetext{Programm}
\end{longanswersC}

\question{\bf Wie geben Sie Ihre Formeln ein?}

\vertikalblockfive{so gut wie nie}{selten}{manchmal}{häufig}{so gut wie immer}{
\blocktextfive{Tastatur}
\blocktextfive{Maus}
\blocktextfive{Trackball}
\blocktextfive{Smartphone}}

\question{\bf Wie zufrieden sind sie mit der Art Ihrer Eingabe?}
\htextlinesix{sehr unzufrieden}{eher unzufrieden}{unentschieden}{eher zufrieden}{sehr zufrieden}{keine Angabe}


\question{\bf Was stört Sie bei der Eingabe von Formeln?}
\vspace{3cm}

\question{\bf Haben Sie eine Idee, wie man Probleme bei der Eingabe von Formeln beheben könnte bzw. was bei der Formeleingabe verbessert werden könnte?}
\vspace{3cm}

\question{\bf Haben Sie bereits Formeln über ein Touch-Gerät eingegeben?}
\begin{answersC}
\item Ja, und zwar mit \linetext{Name des Touch-Gerätes}
\item Nein
\end{answersC}

\clearpage
\question{\bf Mit welchen Geräten können Sie sich vorstellen Formeln einzugeben?}

\textit{Hinweis}: Ein \textit{Stylus} ist ein stiftförmiges Eingabegerät für Touch-Geräte,
das nur mit dem Touch-Gerät verwendbar ist. Ein \textit{Smartpen} hingegen wird
ohne Touch-Gerät und mit Papier verwendet. Ein \textit{Trackball} ist eine Art
Maus, bei der der Ball mit dem Daumen bewegt wird.

\vertikalblockfive{überhaupt nicht}{eher nicht}{unent\-schieden}{eher schon}{auf jeden Fall}{
\blocktextfive{Tablet (Touch)}
\blocktextfive{Smartphone (Touch)}
\blocktextfive{Touch-Bildschirm}\hline
\blocktextfive{Tablet (Stylus)}
\blocktextfive{Smartphone (Stylus)}\hline
\blocktextfive{Smartpen}
\blocktextfive{Grafiktablett}
\blocktextfive{Maus}
\blocktextfive{Trackball}}

\question{\bf Schreiben Sie drei beliebige Formeln, die Sie gerne automatisch erkennen lassen würden, in folgende Felder:}
\textit{Hinweis}: Es besteht die Möglichkeit, dass Programme Formeln
automatisch erkennen, die Sie z.B. mit einem Stylus auf ein Tablet geschrieben
haben. Wenn Sie ein solches Programm hätten, was wären typische Formeln, die
Sie nicht mehr auf die übliche Art eingeben würden, sondern erkennen lassen
würden?

\fbox{\parbox[b][4em][t]{\textwidth}{1} }
\fbox{\parbox[b][4em][t]{\textwidth}{2} }
\fbox{\parbox[b][4em][t]{\textwidth}{3} }

\end{document}
