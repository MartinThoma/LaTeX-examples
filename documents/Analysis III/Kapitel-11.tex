
Die Sätze in diesem Kapitel geben wir \textbf{ohne} Beweis an. Es seien
\(X,Y\subseteq\mdr^d\) nichtleer und offen. 

\begin{definition}
\index{Diffeomorphismus}
Sei \(\Phi\colon X\to Y\) eine Abbildung. \(\Phi\) heißt 
\textbf{Diffeomorphismus} genau dann wenn \(\Phi\in C^1(X,\mdr^d)\), \(\Phi\)
ist bijektiv und \(\Phi^{-1}\in C^{1}(Y,\mdr^d)\).\\
Es gilt \[x=\Phi^{-1}(\Phi(x))\text{ für jedes } x\in X\]
Kettenregel: \[I=\left(\Phi^{-1}\right)^\prime(\Phi(x))\cdot\Phi^\prime(x)
\text{ für jedes } x\in X\] Das heißt \(\Phi^\prime(x)\) ist invertierbar für
alle \(x\in X\) und somit ist \(\det\left(\Phi^\prime(x)\right)\neq 0\)
für alle \(x\in X\).
\end{definition}

\begin{satz}[Transformationssatz (Version I)]
\label{Satz 11.1}
\(\Phi\colon X\to Y\) sei ein Diffeomorphismus.
\begin{enumerate}
\item	\(f\colon Y\to[0,+\infty]\) sei messbar und für \(x\in X\) sei
	\(g(x):=f\left(\Phi(x)\right)\cdot\lvert\det\Phi^\prime(x)\rvert\).\\
	Dann ist \(g\) messbar und es gilt:
	\begin{align*}\tag{$*$} \int_Yf(y)\,dy=\int_Xg(x)\,dx=\int_Xf\left(\Phi(x)\right)
	\cdot\lvert\det\Phi^\prime(x)\rvert\,dx\end{align*}
\item	\(f\colon Y\to\imdr\) sei integrierbar und $g$ sei definiert wie in (1).
	Dann ist $g$ integrierbar und es gilt die Formel \((\ast)\).
\end{enumerate}
\end{satz}

\begin{erinnerung}
\index{Inneres}
Sei \(A\subseteq\mdr^d\) und \(A^\circ:=\{x\in A :\text{ es existiert ein } r=r(x)>0
\text{ mit } U_r(x)\subseteq A\}\) das \textbf{Innere} von $A$. $A^\circ$ ist offen!
\end{erinnerung}

\begin{beispiel}
Sei \(A=\mdr\setminus\mdq\). Es ist \(A^\circ=\emptyset\) und 
\(A\setminus A^\circ=A\). Aus \(\mdr=A\dot\cup\mdq\) folgt
\[\infty=\lambda_1(\mdr)=\lambda_1(A)+\lambda_1(\mdq)=\lambda_1(A)\]
Das heißt \(A\setminus A^\circ\) ist keine Nullmenge.
\end{beispiel}


\begin{satz}[Transformationssatz (Version II)]
\label{Satz 11.2}
Es sei $\emptyset \neq U \subseteq \MdR^d$ offen, $\Phi \in C^1(U, \MdR^d)$, $A \subseteq U$, $A \in \fb_d$,
$X := A^{\circ}$ und $A \setminus A^{\circ}$ eine Nullmenge.
Weiter sei $\Phi$ injektiv auf $X$, $\det\Phi' \neq 0$ für alle $x \in X$, $B:=\Phi(A) \in \fb_d$ und
$g(x) = f(\Phi(x)) \cdot \lvert\det\Phi'(x)\rvert$ für $x \in A$.
%% BILD: von Phi und Mengen
Dann gilt:
\begin{enumerate}
\item	$Y := \Phi(X)$ ist offen und $\Phi: X\to Y$ ist ein Diffeomorphismus.
\item	Ist $f\colon B \to [0, \infty]$ messbar, so ist $g\colon A \to [0, \infty]$ messbar und
\[ \int_B f(y) \, dy = \int_A g(x) \, dx= \int_A f(\Phi(x)) \cdot\lvert\det(\Phi'(x))\rvert \, dx \qquad (\ast\ast)\]
\item	Ist $f\colon B \to \imdr$ messbar, so gilt:\\
\[ f \in \fl^{1}(B) \gdw g \in \fl^{1}(A) \]
Ist $f \in \fl^{1}(B)$ so gilt $(\ast\ast)$
\end{enumerate}
\end{satz}

\begin{folgerungen}
\label{Folgerung 11.3}
\begin{enumerate}
\item	Sei $T\colon \MdR^d \to \MdR^d$ linear und $\det T \neq 0$. Weiter sei $A \in \fb_d$ und $v \in \MdR^d$.
Dann ist $T(A) \in \fb_d$ und es gilt:
\[\lambda_d(T(A)+v) = \lvert\det T\rvert \cdot\lambda_d(A)\]
\item	$\Phi\colon X \to Y$ sei ein Diffeomorphismus und $A \in \fb(X)$. 
Dann ist $\Phi(A) \in \fb_d$ und es gilt:
\[\lambda_d(\Phi(A)) = \int_A |\det \Phi'(X)| \, dx\]
\item	Sei $F \in C^1(X, \MdR^d)$ und $N \subseteq X$ eine Nullmenge.
Dann ist $F(N)$ enthalten in einer Nullmenge.
\end{enumerate}
\end{folgerungen}

\begin{beispiel}
Seien $a,b > 0$ und $T:=\begin{pmatrix} a &  0 \\ 0 & b \end{pmatrix}$, $\det T = a b > 0$. Definiere:
\[A:=\{(x,y)\in \MdR^2: x^2 + y^2 \leq 1\}\]
Dann ist $A \in \fb_2$ und $\lambda_2(A) = \pi$.
\begin{align*}
(u,v) \in T(A) &\gdw \exists (x,y)\in A: (u,v) = (a x, b y)\\
&\gdw \exists (x,y) \in A: (x = \frac{u}{a})\wedge (y = \frac{v}{b})\\
&\gdw \frac{u^2}{a^2} + \frac{v^2}{b^2} \leq 1
\end{align*}
%% BILD: einer Ellipse
Aus \ref{Folgerung 11.3} folgt $T(A) \in \fb_2$ und $\lambda(T(A)) = a b \pi$.
\end{beispiel}

\setcounter{section}{3}
\section{Polarkoordinaten}
\index{Polarkoordinaten}
%% BILD: von PK neben Formeln
%% Tabellarisches Layout?
Jeder Vektor im $\mdr^2$ lässt sich nicht nur durch seine Projektionen auf die Koordinatenachsen $(x,y)$, sondern auch eindeutig durch seine Länge $r$ und den (kleinsten positiven) Winkel $\varphi$ zur $x$-Achse darstellen. Diese Darstellung $(r,\varphi)$ heißen die \textbf{Polarkoordinaten} des Vektors. Dabei gilt:
\[r = \|(x,y)\| = \sqrt{x^2 + y^2}\]
und
\[\begin{cases}
x = r \cos(\varphi)\\
y = r \sin(\varphi)
\end{cases}\]
Definiere nun für $(r,\varphi) \in [0,\infty)\times[0,2\pi]$:
\[\Phi(r,\varphi) := (r \cos(\varphi), r \sin(\varphi))\]
Dann ist $\Phi \in C^1(\MdR^2, \MdR^2)$ und es gilt: 
\[\Phi'(r,\varphi) = \begin{pmatrix}
\cos(\varphi) & -r \sin(\varphi) \\ 
\sin(\varphi) & r \cos(\varphi)
\end{pmatrix}\]
d.h. falls $r > 0$ ist gilt:
\[\det\Phi'(r,\varphi) = r \cos^2(\varphi) + r \sin^2(\varphi) = r > 0\]


\begin{bemerkung}[Faustregel für Polarkoordinaten]
Ist ein Integral der Form $\int_B f(x,y) d(x,y)$ zu berechnen, so lässt sich oft eine Menge $A$ finden, sodass $\Phi(A) = B$ ist.
%% BILD: Kreissektor <=> Rechteck
Mit \ref{Satz 11.2} folgt dann:
\[\int_B f(x,y) \text{ d}(x,y) = \int_A f(r \cos \varphi, r \sin \varphi) \cdot r \text{ d}(r,\varphi)\]
\end{bemerkung}

\begin{beispiel}
\begin{enumerate}
\item	Sei $0 \le \rho < R$. Definiere 
\[B := \{(x,y) \in \MdR^2 : \rho^2 \le x^2 + y^2 \le R^2\} \]
Dann gilt: 
%% BILD: der Kreisfläche und Trafo
\begin{align*}
\lambda_2(B) &= \int_B 1 \text{ d}(x,y)\\ 
&= \int_A 1 \cdot r \text{ d}(r,\varphi)\\ 
&\overset{\text{§\ref{Kapitel 10}}}= \int_{\rho}^{R} \left( \int_0^{2\pi} r \text{ d}\varphi \right) \text{ d}r\\
&= \left[ 2\pi \frac{1}{2} r^2 \right]_\rho^R\\
&= \pi (R^2 - \rho^2)
\end{align*}
		
\item	Definiere 
\[B := \{ (x,y) \in \MdR^2 : x^2 + y^2 \le 1, y \ge 0 \}\]
%% BILD: der (Halb)Kreisfläche und Trafo
Dann gilt:
\begin{align*}
\int_B y \sqrt{x^2+y^2} \text{ d}(x,y) &= \int_A r \sin(\varphi) r \cdot r \text{ d}(r,\varphi)\\
&= \int_A r^3 \sin\varphi \text{ d}(r,\varphi)	\\
&\overset{\text{§\ref{Kapitel 10}}}= \int_0^\pi \left( \int_0^1 r^3 \sin\varphi \text{ d}r \right) \text{ d}\varphi\\
&= \frac{1}{4} \int_0^\pi \sin\varphi \text{ d}\varphi\\
&= \left[ \frac{1}{4}(-\cos\varphi) \right]_0^\pi\\
&= \frac{1}{4}(1+1) = \frac{1}{2}
\end{align*}
\item	\textbf{Behauptung:} \[\int_{-\infty}^\infty e^{-x^2} \, dx = \sqrt{\pi}\]
\textbf{Beweis:}
%% BILD: Bilder von Kreis und Rechtecktrafos/näherungen
Für $\rho > 0$ sei
\[B_\rho := \{(x,y) \in \MdR^2 \mid x,y\ge 0, x^2+ y^2 \le \rho^2\}\]
Weiterhin sei $Q_\rho := [0,\rho] \times [0,\frac{\pi}2]$ und $f(x,y) = e^{-(x^2 + y^2)}$. Dann gilt:
\begin{align*}
\int_{ B_\rho } f(x,y) \text{ d}(x,y) &= \int_{Q_\rho} e^{-r^2} r\text{ d}(r,\varphi)\\
&\overset{\text{§\ref{Kapitel 10}}}= \int_0^{\frac{\pi}{2}} \left( \int_0^\rho r e^{-r^2} \text{ d}r \right) \text{ d}\varphi	\\
&= \frac{\pi}{2} \left[ -\frac{1}{2} e^{-r^2} \right]_{0}^{\rho}\\
&= \frac{\pi}{2} \left( -\frac{1}{2} e^{-\rho^2} +\frac{1}{2} \right)	\\
& =: h(\rho) \stackrel{\rho \to \infty}\to \frac\pi4
\end{align*}
Außerdem gilt:
\begin{align*}
\int_{Q_\rho} f(x,y) \text{ d}(x,y) &= \int_{Q_\rho} e^{-x^2} e^{-y^2}\text{ d}(x,y)	\\
&= \int_0^\rho \left( \int_0^\rho e^{-x^2} e^{-y^2} \text{ d}y \right) \text{ d}x \\
&= \left( \int_0^\rho e^{-x^2} \text{ d}x \right)^2
\end{align*}
		
Wegen $ B_\rho \subseteq Q_\rho \subseteq B_{\sqrt{2} \rho} $ und $f \ge 0$ folgt:
\begin{center}
\begin{tabular}{cccccc}
&$\int_{B_\rho} f \text{ d}(x,y)$ &$\le$ &$\int_{Q_\rho} f \text{ d}(x,y)$ &$\le$ &$\int_{B_{\sqrt{2} \rho}} f \text{ d}(x,y)$\\
$\implies$ &$h(\rho)$ &$\le$ &$\int_{Q_\rho} f \text{ d}(x,y)$	&$\le$ &$h(\sqrt{2} \rho)$ \\
$\implies$ &$h(\rho)$ &$\le$ &$\left( \int_0^\rho e^{-x^2} \text{ d}x \right)^2$ &$\le$ &$h(\sqrt{2} \rho)$ \\
$\implies$ &$\sqrt{h(\rho)}$ &$\le$ &$\int_0^\rho e^{-x^2} \text{ d}x$ &$\le$ &$\sqrt{h(\sqrt{2} \rho)}$\\
\end{tabular}
\end{center}
Mit $\rho \to \infty$ folgt daraus 
\[\int_0^\infty e^{-x^2} \text{ d}x = \frac{\sqrt{\pi}}{2}\]
und damit die Behauptung.
\end{enumerate}
\end{beispiel}

\section{Zylinderkoordinaten}
\index{Zylinderkoordinaten}
Definiere für $(r,\varphi,z)\in[0,\infty)\times[0,2\pi]\times\mdr$:
\[\Phi(r,\varphi,z):=(r\cos(\varphi),r\sin(\varphi),z)\]
Dann gilt:
\[|\det\Phi'(r,\varphi,z)|=\left|\det
\begin{pmatrix}
\cos(\varphi)&-r\sin(\varphi)&0\\
\sin(\varphi)&r\cos(\varphi)&0\\
0&0&1\end{pmatrix}\right|=r
\]

\begin{bemerkung}[Faustregel für Zylinderkoordinaten]
Ist ein Integral der Form $\int_B f(x,y,z) d(x,y,z)$ zu berechnen, so lässt sich eine Menge $A$ finden, sodass $\Phi(A) = B$ ist.
Mit \ref{Satz 11.2} folgt dann:
\[\int_B f(x,y,z) \text{ d}(x,y,z) = \int_A f(r \cos \varphi, r \sin \varphi, z) \cdot r \text{ d}(r,\varphi,z)\]
\end{bemerkung}

\begin{beispiel}
Definiere
\[B:=\{(x,y,z)\in\mdr^3\mid x^2+y^2\le 1, x,y\ge 0,z\in[0,1]\}\]
Dann gilt:
\begin{align*}
\int_B z+y\sqrt{x^2+y^2}\text{ d}(x,y,z)&=\int_A(z+r\sin(\varphi)\cdot r)\cdot r\text{ d}(r,\varphi,z)\\
&=\int_A rz+r^3\sin(\varphi)\text{ d}(r,\varphi,z)\\
&=\int_0^1(\int_0^{\frac\pi 2}(\int_0^1 rz+r^3\sin(\varphi)\text{ d}r)\text{ d}\varphi)\text{ d}z\\
&=(\int_0^1 r\text{ d}r)\cdot(\int_0^1 z\text{ d}z)\cdot(\int_0^{\frac\pi 2} \text{ d}\varphi)+ (\int_0^1 r^3\text{ d}r)\cdot(\int_0^{\frac\pi 2} \sin(\varphi)\text{ d}\varphi)\cdot(\int_0^1 \text{ d}z)\\
&= \frac\pi 8+\frac14
\end{align*}
\end{beispiel}

\section{Kugelkoordinaten}
\index{Kugelkoordinaten}
Definiere für $(r,\varphi,\theta)\in [0,\infty)\times[0,2\pi]\times[0,\pi]$:
\[\Phi(r,\varphi,\theta):=(r\cos(\varphi)\sin(\theta),r\sin(\varphi)\sin(\theta),r\cos(\theta))\]
Dann gilt (nachrechnen!):
\[\det\Phi'(r,\varphi,\theta)= -r^2\sin(\theta)\]

\begin{bemerkung}[Faustregel für Kugelkoordinaten]
Ist ein Integral der Form $\int_B f(x,y,z) d(x,y,z)$ zu berechnen, so lässt sich eine Menge $A$ finden, sodass $\Phi(A) = B$ ist.
Mit \ref{Satz 11.2} folgt dann:
\[\int_B f(x,y,z) \text{ d}(x,y,z) = \int_A f(r\cos(\varphi)\sin(\theta),r\sin(\varphi)\sin(\theta),r\cos(\theta)) \cdot r^2\sin(\theta) \text{ d}(r,\varphi,\theta)\]
\end{bemerkung}

\begin{beispiel}
Definiere
\[B:=\{(x,y,z)\in\mdr^3\mid 1\le \|(x,y,z)\|\le 2, x,y,z\ge 0\}\]
Dann gilt:
\begin{align*}
\int_B \frac1{x^2+y^2+z^2}\text{ d}(x,y,z)&=\int_A \frac1{r^2}\cdot r^2\cdot\sin(\theta)\text{ d}(r,\varphi,\theta)\\
&=\int_A \sin(\theta)\text{ d}(r,\varphi,\theta)\\
&=\frac\pi2
\end{align*}
\end{beispiel}

\begin{beispiel}[Zugabe von Herrn Dr. Ullmann]
Wir wollen das Kugelvolumen $\lambda_3(K)$ mit $K:=\{(x,y,z)\in\mdr^3\mid\|(x,y,z)\|\le 1\}$ berechnen. Dann ist $K=\Phi(A)$ mit $A:= [0,1]\times[0,2\pi]\times [0,\pi]$. Und es gilt:
\begin{align*}
\lambda_3(K)&=\int_K 1\text{ d}(x,y,z)\\
&=\int_A r^2\sin(\theta)\text{ d}(r,\varphi,\theta)\\
&=\int_0^1(\int_0^{2\pi}(\int_0^\pi r^2\sin(\theta) \text{ d}\theta)\text{ d}\varphi)\text{ d}r\\
&=(\int_0^1 r^2 \text{ d}r)\cdot(\int_0^{2\pi} \text{ d}\varphi)\cdot(\int_0^\pi \sin(\theta) \text{ d}\theta)\\
&=\frac{4\pi}3
\end{align*}
\end{beispiel}
