\chapter*{Introduction}
When you want to develop a selfdriving car, you have to plan which path 
it should take. A reasonable choice for the representation of
paths are cubic splines. You also have to be able to calculate
how to steer to get or to remain on a path. A way to do this
is by applying the \href{https://en.wikipedia.org/wiki/PID_algorithm}{PID algorithm}.
This algorithm needs to know the signed current error. So you need to 
be able to get the minimal distance of a point (the position of the car)
to a cubic spline (the prefered path)
combined with sign (which represents the steering direction).
As one steering direction might be prefered, it is not only necessary to
get the minimal absolute distance, but might also help to get all points
on the spline with minimal distance.

In this paper, I want to discuss how to find all points on a cubic 
function with minimal distance to a given point.
As other representations of paths might be easier to understand and
to implement, I will also cover the problem of finding the minimal
distance of a point to a polynomial of degree 0, 1 and 2.

While I analyzed this problem, I've got interested in variations
of the underlying PID-related problem. So I will try to give
robust and easy-to-implement algorithms to calculate the distance
of a point to a (piecewise or global) defined polynomial function
of degree $\leq 3$.

When you're able to calculate the distance to a polynomial which is
defined on a closed invervall, you can calculate the distance from
a point to a spline by calculating the distance to the pieces of the
spline.
