\documentclass[a4paper,12pt]{scrartcl}
\usepackage{amssymb, amsmath} % needed for math
\usepackage[utf8]{inputenc} % this is needed for umlauts
\usepackage[ngerman]{babel} % this is needed for umlauts
\usepackage[T1]{fontenc}    % this is needed for correct output of umlauts in pdf
\usepackage[top=3cm, bottom=3cm, left=4cm, right=2cm]{geometry} %layout
\usepackage{hyperref}   % links im text
\usepackage{color}
\usepackage{framed}
\usepackage{enumerate}  % for advanced numbering of lists
\usepackage{pdfpages}  % Signatureinbingung und includepdf
\usepackage{parskip}   % spaces instead of intendation between paragraphs
\usepackage{cite}
\linespread{1.45}     % 1,45-Facher Zeilenabstand

\usepackage{titlesec}
%\titlespacing{command}{left spacing}{before spacing}{after spacing}[right]
\titlespacing\section{0pt}{12pt plus 3pt minus 2pt}{0pt plus 2pt minus 1pt}
\usepackage[framemethod=tikz,xcolor=true]{mdframed}

\usepackage{enumitem}
\usepackage{braket} % needed for nice printing of sets

\usepackage{fancyhdr}  % needed for the footer
\usepackage{lastpage}  % needed for the footer

\clubpenalty  = 10000   % Schusterjungen verhindern
\widowpenalty = 10000   % Hurenkinder verhindern

\hypersetup{ 
  pdfauthor   = {Martin Thoma}, 
  pdfkeywords = {Asymmetrische Verschlüsselungsverfahren; RSA-Kryptosystems}, 
  pdftitle    = {Asymmetrische Verschlüsselungsverfahren am Beispiel des RSA-Kryptosystems} 
} 

\pagestyle{fancy}
\fancyhf{}
\renewcommand{\headrulewidth}{0pt}
\renewcommand{\footrulewidth}{0pt}
\fancyfoot[R]{Seite~\thepage~von \pageref{LastPage}}

% From http://www.matthewflickinger.com/blog/archives/2005/02/20/latex_mod_spacing.asp
% Thanks!
\makeatletter
\def\imod#1{\allowbreak\mkern10mu({\operator@font mod}\,\,#1)}
\makeatother

\usepackage{minted} % needed for the inclusion of source code

\begin{document}
\setcounter{page}{0}
\pagenumbering{roman} 

%\thispagestyle{empty}

\begin{center}
{\Huge Paul-Klee-Gymnasium}


Facharbeit aus der Mathematik



Thema:

Asymmetrische Verschlüsselungsverfahren
am Beispiel des RSA-Kryptosystems



\begin{tabular}{lll}
Verfasser       &:& Martin Andreas Thoma\\
Kursleiter      &:& Claudia Wenninger\\
Abgegeben am    &:& 20.01.2010 (verändert am 06.04.2010)\\
\\
\\
Erzielte Note   &:& \line(1,0){120}\\
Erzielte Punkteanzahl   &:& \line(1,0){120}\\
\end{tabular} 
\end{center}


\includepdf[pages=1]{Titelseite.pdf}
\clearpage

\thispagestyle{empty}
\tableofcontents 
\clearpage


\pagenumbering{arabic} 
\setcounter{page}{2}

% Start der eigentlichen Arbeit
\section{Prinzip der asymmetrischen Verschlüsselung}
Bereits im 5. Jahrhundert v. Chr. war ein Verfahren zur geheimen Weitergabe an Informationen bekannt: Die Skytale\footnote{[Wrixon], S. 21f}. Man wickelte Papier spiralförmig um einen Stab, die Skytale,  und schrieb die Nachricht längs der Skytale auf das Papier. Dann wurde das Papier dem Empfänger gebracht, der mit einer gleichen Skytale diese Nachricht entschlüsseln konnte.

Verfahren, die den gleichen Schlüssel zur Ver- als auch zur Entschlüsselung verwenden nennt man symmetrisch\footnote{[Birthälmer], S. 4}. In dem Beispiel ist die Skytale der Schlüssel. Es ist unüblich die Skytale als symmetrische Verschlüsselung zu bezeichnen, normalerweise sind Block- oder Stromchiffren damit gemeint. Diese sind jedoch schwerer zu beschreiben.

Nach Kerckhoffs' Prinzip darf ein Verschlüsselungssystem keine Geheimhaltung erfordern\footnote{[Petitcolas]}, also muss der Schlüssel für die Sicherheit sorgen. Wollen allerdings zwei Personen miteinander geheim kommunizieren, so muss dieser Schlüssel übertragen werden. Bei der Übertragung könnte er abgefangen werden.
Asymmetrische Verschlüsselungsverfahren benutzen einen öffentlichen Schlüssel zum verschlüsseln und einen privaten Schlüssel zum entschlüsseln. Will Alice eine geheime Nachricht von Bob empfangen, so schickt sie Bob ihren öffentlichen Schlüssel. Bob verschlüsselt seine Nachricht mit diesem Schlüssel und schickt die Nachricht an Alice. Der private Schlüssel wird nicht übertragen. In dieser Hinsicht sind asymmetrische Verschlüsselungsverfahren sicherer als symmetrische.

Mithilfe von asymmetrischen Verschlüsselungen kann man auch digitale Signaturen erstellen und sich damit authentifizieren. Im RSA-Verfahren sind privater und öffentlicher Schlüssel austauschbar. Das heißt, wenn eine Nachricht mit dem privaten Schlüssel verschlüsselt wird, kann sie mit dem öffentlichen Schlüssel entschlüsselt werden. Da jedoch nur der Besitzer des privaten Schlüssels eine Nachricht erstellen kann, die man mit dem öffentlichen Schlüssel entschlüsseln kann, ist es so möglich den Absender einer Nachricht zu authentifizieren.

\clearpage
\index{Lebesgue-Maß}

In diesem Kapitel sei \(X\) eine Menge, \(X\neq\emptyset\).
\begin{definition}
    \index{Ring}
    Sei \(\emptyset\neq \fr \subseteq \cp(X)\).
    $\fr$ heißt ein \textbf{Ring} auf \(X\), genau dann wenn gilt:
    \begin{enumerate}
        \item[(R1)] \(\emptyset \in \fr\)
        \item[(R2)] \(A,B \in \fr \, \implies \; A\cup B, \, B \setminus A \in \fr\)
    \end{enumerate}
\end{definition}

\textbf{Hinweis}: $(\fr, \cup, \setminus)$ ist kein Ring im Sinne
der linearen Algebra, $(\fr, \cup)$ kein Inverses Element hat und
$(\fr, \cup)$ nicht kommutativ ist.

\begin{definition}
    \index{Elementarvolumen}
    \index{Figuren}
    Sei \(d\in\MdN\).
    \begin{enumerate}
        \item \(\ci_d :=\Set{(a,b] | a,b \in \MdR^{d}, \, a \leq b} (\emptyset \in \ci_d)\).
              Seien \(a=(a_{1},\dots,a_{d}),\,b=(b_{1},\dots,b_{d})\in\MdR^d\)
              und \(I:=(a,b] \in \ci_{d}\)
              \[
              \lambda_{d}(I)= \begin{cases}
                0                                             & \text{falls }I=\emptyset\\
                (b_{1}-a_{1})(b_{2}-a_{2})\dots(b_{d}-a_{d}) & \text{falls }I\neq\emptyset\end{cases}\quad\text{(\textbf{Elementarvolumen})}
              \]
        \item \(\cf_d:=\Set{\bigcup_{j=1}^{n}I_{j} | n\in\MdN,\,I_{1},\dots,I_{n}\in \ci_d}\) (\textbf{Menge der Figuren})
    \end{enumerate}
\end{definition}
Ziel dieses Kapitels: Fortsetzung von \(\lambda_{d}\) auf \(\cf_{d}\)
und dann auf \(\fb_d\) (\(\leadsto\) Lebesgue-Maß)

Beachte: \(\ci_{d}\subseteq\cf_{d}\subseteq\fb_{d}\overset{1.4}{\implies}\fb_{d}=\sigma(\ci_{d})=\sigma(\cf_{d})\)
\begin{lemma}
    \label{Lemma 2.1}
    Seien \(I,I'\in\ci_{d}\) und \(A\in\cf_{d}\). Dann:
    \begin{enumerate}
        \item \(I\cap I'\in\ci_{d}\)
        \item \(I\setminus I'\in\cf_{d}.\)
              Genauer: \(\exists\left\{I_{1}',\dots,I_{l}'\right\}\subseteq\ci_{d}\) disjunkt:
              \(I\setminus I'=\bigcup_{j=1}^{l}{I_{j}'}\) % \bigcupdot
        \item \(\exists\left\{I_{1}',\dots,I_{l}'\right\}\subseteq\ci_{d}\) disjunkt: \(A=\bigcup_{j=1}^{l}{I_{j}'}\)
        \item \(\cf_d\) ist ein Ring.
    \end{enumerate}
\end{lemma}

\begin{beweis}
\begin{enumerate}
    \item Sei \(I=\prod_{k=1}^{d}{(a_{k},b_{k}]},
             \,I'=\prod_{k=1}^{d}{(\alpha_{k},\beta_{k}]};
   \,\alpha_{k}':=\max\{\alpha_{k},a_{k}\},
    \,\beta_{k}':=\min\{\beta_{k},b_{k}\}\)

          \(\exists k\in\Set{1,\dots,d} : \alpha_{k}'\geq\beta_{k}'
            \implies I\cap I'=\emptyset\in\ci_{d}\).\\
          Sei \(\alpha_{k}'<\beta_{k}'\forall k\in\{1,\dots,d\}\), so
          ist \(I\cap I'=\prod_{k=1}^{d}{(\alpha_{k}',\beta_{k}']\in\ci_{d}}\)
    \item Induktion nach \(d\):
          \begin{itemize}
            \item[I.A.] Klar \checkmark % hier fehlt noch eine Graphik
            \item[I.V.] Die Behauptung gelte für ein \(d\geq 1\)
            \item[I.S.] Seien \(I,I'\in\ci_{d+1}\). Es existieren \(I_{1},I_{1}'\in\ci_{1}\) und \(I_{2},I_{2}'\in\ci_{d}\) mit:
                        \(I=I_{1}\times I_{2},\,I'=I_{1}'\times I_{2}'\)
                        % Graphik einfuegen!

                        Nachrechnen:
                        \[
                        I\setminus I'=(I_{1}\setminus I_{1}')\times I_{2}\dot \cup(I_{1}\cap I_{1}')\times(I_{2}\setminus I_{2}')
                        \]
                        I.A.\(\implies\,I_{1}\setminus I_{1}'=\) endliche disjunkte Vereinigung von Elementen aus \(\ci_{1}\)\\
                        I.V.\(\implies\,I_{2}\setminus I_{2}'=\) endliche disjunkte Vereinigung von Elementen aus \(\ci_{d}\)\\
                        Daraus folgt die Behauptung für \(d+1\)
          \end{itemize}
    \item \begin{itemize}
            \item[\underline{Vor.:}] Sei $n \in \mdn$ und
                \(A=\bigcup_{j=1}^{n}{I_{j}}\) mit
                \(I_{1},\dots,I_{d}\in\ci_{d}\)
            \item[\underline{Beh.:}] Es existiert
                \(\{I_{1}',\dots,I_{l}'\}\subseteq\ci_{d}\) disjunkt:
                \(A=\bigcup_{j=1}^{l}{I_{j}'}\)
          \item[\underline{Bew.:}] mit Induktion nach $n$:
          \begin{itemize}
            \item[I.A.] \(n=1:\,A=I_{1}\)\checkmark
            \item[I.V.] Die Behauptung gelte für ein \(n\geq 1\)
            \item[I.S.] Sei \(A=\bigcup_{j=1}^{n+1}{I_{j}}\quad(I_{1},\dots,I_{n+1}\in\ci_{d})\)

                        IV\(\,\implies\,\exists\{I_{1}',\dots,I_{l}'\}\subseteq\ci_{d}\) disjunkt:
                        \(\bigcup_{j=1}^{n}{I_{j}}=\bigcup_{j=1}^{l}{I_{j}'}\)	% \bigcupdot

                        Dann: \(A=I_{n+1}\cup\bigcup_{j=1}^{l}{I_{j}'}=I_{n+1}\cup\bigcup_{j=1}^{l}{(I_{j}'\setminus I_{n+1})}\) % \cupdot

                        Wende (2) auf jedes \(I_{j}'\setminus I_{n+1}\) an \((j=1,\dots,l)\):
                        \(I_{j}'\setminus I_{n+1}=\bigcup_{j=1}^{l_{j}}{I_{j}''}\quad(I_{j}''\in\ci_{d})\)

                        Damit folgt:
                        \[
                        A=I_{n+1}\cup\bigcup_{j=1}^{l}{\left(\bigcup_{j=1}^{l_{j}}{I_{j}''}\right)}
                        \]
                        Daraus folgt die Behauptung für \(n+1\).
            \end{itemize}
        \end{itemize}
    \item \((a,a]=\emptyset\implies\emptyset\in\cf_{d}\)

          Seien \(A,B\in\cf_{d}\). Klar: \(A\cup B\in\cf_{d}\)

          Sei \(A=\bigcup_{j=1}^{n}{I_{j}},\,B=\bigcup_{j=1}^{n}{I_{j}'}\quad(I_{j},I_{j}'\in\ci_{d})\). Zu zeigen: \(B\setminus A\in\cf_{d}\)
          \begin{itemize}
            \item[I.A.] \(n=1:\,A=I_{1}\implies B\setminus A=\bigcup_{j=1}^{n}(\underbrace{I_{j}'\setminus I_{j}}_{\in\cf_{d}})\). Wende
                        (2) auf jedes \(I_{j}'\setminus I_{1}\) an. Aus (2) folgt dann \(B\setminus A\in\cf_{d}\).
            \item[I.V.] Die Behauptung gelte für ein \(n\in\MdN\)
            \item[I.S.] Sei \(A'=A\cup I_{n+1}\quad(I_{n+1}\in\ci_{d})\). Dann:
                        \[
                        B\setminus A'=\underbrace{(B\setminus A)}_{\in\cf_{d}}\setminus\underbrace{I_{n+1}}_{\in\cf_{d}}\in\cf_{d}
                        \text{ (siehe I.A.)}
                        \]
          \end{itemize}
  \end{enumerate}
\end{beweis}
ohne Beweis:
\begin{lemma}[Unabhängigkeit von der Darstellung]
    \label{Lemma 2.2}
    Sei \(A\in\cf_{d}\) und \(\{I_{1},\dots,I_{n}\}\subseteq\ci_{d}\) disjunkt und
    \(\{I_{1}',\dots,I_{m}'\}\subseteq\ci_{d}\) disjunkt mit
    \(\bigcup_{j=1}^{n}{I_{j}}=A=\bigcup_{j=1}^{m}{I_{j}'}\). Dann:
    \[
    \sum_{j=1}^{n}{\lambda_{d}(I_{j})}=\sum_{j=1}^{m}{\lambda_{d}(I_{j}')}
    \]
\end{lemma}
\begin{definition}
    Sei \(A\in\cf_{d}\) und \(A=\bigcup_{j=1}^{n}{I_{j}}\) mit
    \(\{I_{1},\dots,I_{n}\}\subseteq\ci_{d}\)
    disjunkt (beachte Lemma \ref{Lemma 2.1}, Punkt 3).
    \[
    \lambda_{d}(A):=\sum_{j=1}^{n}{\lambda_{d}(I_{j})}
    \]
    \folgtnach{\ref{Lemma 2.2}} \(\lambda_{d}:\cf_{d}\to[0,\infty)\)
    ist wohldefiniert.
\end{definition}
\begin{satz}
    \label{Satz 2.3}
    Seien \(A,B\in\cf_{d}\) und \((B_{n})\) sei eine Folge in \(\cf_{d}\).
    \begin{enumerate}
        \item \(A\cap B=\emptyset\implies\lambda_{d}(A\cup B)=\lambda_{d}(A)+\lambda_{d}(B)\)
        \item \(A\subseteq B\implies\lambda_{d}(A)\leq\lambda_{d}(B)\)
        \item \(\lambda_{d}(A\cup B)\leq\lambda_{d}(A)+\lambda_{d}(B)\)
        \item Sei \(\delta>0\). Es existiert \(C\in\cf_{d}:\overline{C}\subseteq B\)
              und \(\lambda_{d}(B\setminus C)\leq\delta\).
        \item Ist \(B_{n+1}\subseteq B_{n}\forall n\in\mdn\) und
              \(\bigcap B_{n}=\emptyset\), so gilt:
              \(\lambda_{d}(B_{n})\to 0\,(n\to \infty)\)
    \end{enumerate}
\end{satz}

\begin{beweis}
\begin{enumerate}
\item Aus Lemma \ref{Lemma 2.1} folgt: Es existiert
\(\{I_{1},\dots,I_{n}\}\subseteq\ci_{d}\)
disjunkt und es existiert \(\{I_{1}',\dots,I_{m}'\}\subseteq\ci_{d}\) disjunkt:
\(A=\bigcup_{j=1}^{n}{I_{j}},\,B=\bigcup_{j=1}^{m}{I_{j}'}\).

\(J:=\{I_{1},\dots,I_{n},I_{1}',\dots,I_{m}'\}\subseteq\ci_{d}\). Aus
\(A\cap B=\emptyset\) folgt: \(J\) ist disjunkt. Dann:
\(A\cup B=\bigcup_{I\in J}{I}\)	% Hier auch wieder: \bigcupdot

Also:
\begin{align*}
\lambda_{d}(A\cup B)&=\sum_{I\in J}{\lambda_{d}(I)}\\
    &=\sum_{j=1}^{n}{\lambda_{d}(I_{j})}+\sum_{j=1}^{m}{\lambda_{d}(I_{j}')}\\
    &=\lambda_{d}(A)+\lambda_{d}(B)
\end{align*}
\item wie bei Satz \ref{Satz 1.7}
\item \(\lambda_{d}(A\cup B)=\lambda(A \dot{\cup} (B\setminus A))\overset{(1)}{=}\lambda_{d}(A)+\lambda_{d}(B\setminus A)\overset{(2)}{\leq}\lambda_{d}(A)+\lambda_{d}(B)\) % \cupdot
\item Übung (es genügt \(B\in\ci_{d}\) zu betrachten).
\item Sei \(\varepsilon>0\). Aus (4) folgt: Zu jedem \(B_{n}\) existiert ein
\(C_{n}\in\cf_{d}:\overline{C}_{n}\subseteq B_{n}\) und
\begin{equation}
\label{eq: Abschaetzung Mass -- Beweis Satz 2.3.(5)}
\lambda_{d}(B_{n}\setminus C_{n})\leq\frac{\varepsilon}{2^{n}}
\end{equation}
Dann:
\(\bigcap{\overline{C}_{n}}\subseteq\bigcap{B_{n}}=\emptyset\implies\bigcup{\overline{C}_{n}^{c}}=\mdr^{d}\implies\underbrace{\overline{B}_{1}}_{\text{kompakt}}\subseteq\bigcup{\underbrace{\overline{C}_{n}^{c}}_{\text{offen}}}\)

Aus der Definition von Kompaktheit (Analysis II, \S 2) folgt:
\(\exists m\in\mdn:\,\bigcup_{j=1}^{m}{\overline{C}_{j}^{c}}\supseteq\overline{B}_{1}\)
Dann: \(\bigcap_{j=1}^{m}{\overline{C}_{j}}\subseteq\overline{B}_{1}^{c}\).
Andererseits: \(\bigcap_{j=1}^{m}{\overline{C}_{j}}\subseteq\bigcap_{j=1}^{m}{B_{j}}\subseteq B_{1}\subseteq\overline{B}_{1}\).

Also: \(\bigcap_{j=1}^{m}{\overline{C}_{j}}=\emptyset\). Das heißt:
\(\bigcap_{j=1}^{n}{\overline{C}_{j}}=\emptyset \quad \forall n\geq m\)

\(D_{n}:=\bigcap_{j=1}^{n}{C_{j}}\). Dann: \(D_{n}=\emptyset \quad \forall n\geq m\)

\textbf{Behauptung:} \(\lambda_{d}(B_{n}\setminus D_{n})\leq\left(1-\frac{1}{2^{n}}\right)\ep \quad \forall n\in\mdn\)
\begin{beweis} (induktiv)
\begin{itemize}
\item[I.A.] \(\lambda_{d}(B_{1}\setminus D_{1})=\lambda_{d}(B_{1}\setminus C_{1})\overset{\eqref{eq: Abschaetzung Mass -- Beweis Satz 2.3.(5)}}{\leq}\frac{\ep}{2}=\left(1-\frac{1}{2}\right)\ep\) \checkmark
\item[I.V.] Sei \(n\in\mdn\) und es gelte
            $\lambda_{d}(B_{n}\setminus D_{n})\leq\left(1-\frac{1}{2^{n}}\right)\ep$
\item[I.S.] \begin{align*}
    \lambda_{d}(B_{n+1}\setminus D_{n+1})&=\lambda_{d}\left((B_{n+1}\setminus D_{n})\cup(B_{n+1}\setminus C_{n+1})\right)\\
    &\overset{(3)}{\leq}\lambda_{d}(\underbrace{B_{n+1}\setminus D_n}_{\subseteq B_{n}\setminus D_{n}})+\underbrace{\lambda_{d}(B_{n+1}\setminus C_{n+1})}_{\overset{\eqref{eq: Abschaetzung Mass -- Beweis Satz 2.3.(5)}}{\leq}\frac{\ep}{2^{n+1}}}\\
    &\overset{(2)}{\leq}\lambda_{d}(B_{n}\setminus D_{n})+\frac{\ep}{2^{n+1}}\\
    &\overset{\text{I.V.}}{\leq}\left(1-\frac{1}{2^{n}}\right)+\frac{\ep}{2^{n+1}}\\
&=\left(1-\frac{1}{2^{n+1}}\right)\ep
    \end{align*}
\end{itemize}
\end{beweis}

Für \(n\geq m:\,D_{n}=\emptyset\,\implies\,\lambda_{d}(B_{n})=\lambda_{d}(B_{n}\setminus D_{n})\leq\left(1-\frac{1}{2^{n}}\right)\varepsilon\leq\varepsilon\)
\end{enumerate}
\end{beweis}

\begin{definition}
\index{Prämaß}
Es sei \(\fr\) ein Ring auf \(X\). Eine Abbildung \(\mu:\fr\to[0,\infty]\)
heißt ein \textbf{Prämaß} \ auf \(\fr\), wenn gilt:
\begin{enumerate}
\item \(\mu(\emptyset)=0\)
\item Ist \(A_{j}\) eine disjunkte Folge in \(\fr\) und \(\bigcup{A_{j}}\in\fr\), so ist \(\mu\left(\bigcup{A_{j}}\right)=\sum{\mu(A_{j})}\).
\end{enumerate}
\end{definition}

\begin{satz}
\label{Satz 2.4}
\(\lambda_{d}:\cf_{d}\to[0,\infty]\) ist ein Prämaß auf $\cf_{d}$.
\end{satz}
\begin{beweis}
\begin{enumerate}
\item Klar: \(\lambda_{d}(\emptyset)=0\)
\item Sei \(A_{j}\) eine disjunkte Folge in \(\cf_{d}\) und \(A:=\bigcup{A_{j}}\in\cf_{d}\).

\(B_{n}:=\bigcup_{j=n}^{\infty}{A_{j}}\,(n\in\mdn)\); \((B_{n})\) hat die
Eigenschaften aus \ref{Satz 2.3}, Punkt 5. Also: \(\lambda_{d}(B_{n})\to 0\).

Für \(n\geq 2\):
\[
\lambda_{d}(A)=\lambda_{d}(A_{1}\cup\dots\cup A_{n-1}\cup B_{n})\overset{\ref{Satz 2.3}.(1)}{=}\sum_{j=1}^{n-1}{\lambda_{d}(A_{j})}+\lambda_{d}(B_{n})
\]
Daraus folgt:
\[
\sum_{j=1}^{n-1}{\lambda_{d}(A_{j})}=\lambda_{d}(A)-\lambda_{d}(B_{n})\quad\forall n\geq 2
\]
Mit \(n\to\infty\) folgt die Behauptung.
\end{enumerate}
\end{beweis}

Ohne Beweis:
\begin{satz}[Fortsetzungssatz von Carath\'eodory]
\label{Satz 2.5}
Sei \(\fr\) ein Ring auf \(X\) und \(\mu:\fr\to[0,\infty]\) ein Prämaß. Dann
existiert ein Maßraum \((X,\fa(\mu),\overline{\mu})\) mit
\begin{enumerate}
\item \(\sigma(\fr)\subseteq\fa(\mu)\)
\item \(\overline{\mu}(A)=\mu(A) \quad \forall A\in\fr\)
\end{enumerate}
Insbesondere: \(\overline{\mu}\) ist ein Maß\ auf \(\sigma(\fr)\).
\end{satz}

\begin{satz}[Eindeutigkeitssatz]
\label{Satz 2.6}
Sei \(\emptyset\neq\ce\subseteq\cp(X)\), es seien \(\nu,\,\mu\) Maße auf
\(\sigma(\ce)\).

Es gelte:
\begin{enumerate}
    \item \(E,F\in\ce\implies E\cap F\in\ce\quad\text{(durchschnittstabil)}\)
    \item $\exists$ eine Folge \((E_{n})\) in \(\ce\): \(\bigcup{E_{n}}=X\)
          und \(\mu(E_{n})<\infty \quad \forall n\in\mdn\).
    \item \(\mu(E)=\nu(E) \quad \forall E\in\ce\)
\end{enumerate}
Dann: \(\mu=\nu\) auf \(\sigma(\ce)\).
\end{satz}

\begin{satz}
\label{Satz 2.7}
\index{Lebesgue-Maß}
Es gibt genau eine Fortsetzung von \(\lambda_{d}:\cf_{d}\to[0,\infty]\) auf
\(\fb_{d}\) zu einem Maß. Diese Fortsetzung heißt \textbf{Lebesgue-Maß} \ (L-Maß)
und wird ebenfalls mit \(\lambda_{d}\) bezeichnet.
\end{satz}
\begin{beweis}
\folgtnach{(\ref{Lemma 2.1}) und (\ref{Satz 2.4})}: \(\lambda_{d}\) ist ein
Prämaß\ auf \(\fr:=\cf_{d}\); es ist \(\sigma(\fr)=\fb_{d}\).

\folgtnach{\ref{Satz 2.5}}: \(\lambda_{d}\) kann zu einem Maß auf
\(\sigma(\cf_{d}) = \fb_{d}\) fortgesetzt werden. Für diese
Fortsetzung schreiben wir wieder $\lambda_d$, also
$\lambda_d: \fb_{d} \rightarrow [0, +\infty]$

Sei \(\nu\) ein weiteres Maß\ auf \(\fb_{d}\) mit:
\(\nu(A)=\lambda_{d}(A)\,\forall A\in\cf_{d}\). \(\ce:=\ci_{d}\). Dann:
\(\sigma(\ce)\overset{\ref{Satz 1.4}}{=}\fb_{d}\).
\begin{enumerate}
    \item \(E,F\in\ce\overset{\ref{Lemma 2.1}}{\implies}E\cap F\in\ce\)
    \item \(E_{n}:=(-n,n]^{d}\)
          Klar:
          \begin{align*}
            \bigcup E_{n}&=\mdr^{d}\\
            \lambda_{d}(E_{n})&=(2n)^{d}<\infty
          \end{align*}
\end{enumerate}
Klar: \(\nu(E)=\lambda_{d}(E)\,\forall E\in\ce\). Mit Satz \ref{Satz 2.6} folgt
dann: \(\nu=\lambda_{d}\) auf \(\fb_{d}\).
\end{beweis}

\begin{bemerkung}
Sei \(X\in\fb_{d}\). Aus 1.6 folgt: \(\fb(X)=\Set{A\in\fb_{d} | A\subseteq X}\).
Die Einschränkung von \(\lambda_{d}\) auf \(\fb(X)\) heißt ebenfalls
L-Maß\ und wird mit \(\lambda_{d}\) bezeichnet.
\end{bemerkung}

\begin{beispieleX}
\begin{enumerate}
\item Seien \(a=(a_{1},\dots,a_{d}),\,b=(b_{1},\dots,b_{d})\in\mdr^{d},\,a\leq b\) und \(I=[a,b]\).\\
\textbf{Behauptung}\\\(\lambda_{d}([a,b])=(b_{1}-a_{1})\dots(b_{d}-a_{d})\) (Entsprechendes gilt für \((a,b)\) und \([a,b)\))
\begin{beweis}
\(I_{n}:=(a_{1}-\frac{1}{n},b_{1}]\times\dots\times(a_{d}-\frac{1}{n},b_{d}];\,I_{1}\supset I_{2}\supset\dots;\,\bigcap I_{n}=I,\,\lambda_{d}(I_{1})<\infty\)

Aus Satz \ref{Satz 1.7}, Punkt 5, folgt:
\begin{align*}
\lambda_{d}(I)&=\lim_{n\to\infty}{\lambda_{d}(I_{n})}\\
&=\lim_{n\to\infty}{(b_{1}-a_{1}+\frac{1}{n})\dots(b_{d}-a_{d}+\frac{1}{n})}\\
&=(b_{1}-a_{1})\dots(b_{d}-a_{d})
\end{align*}
\end{beweis}
\item Sei \(a\in\mdr^{d},\,\{a\}=[a,a]\in\fb_{d}\). \folgtnach{Bsp (1)} \(\lambda_{d}(\{a\})=0\).
\item \(\mdq^{d}\) ist abzählbar, also: \(\mdq^{d}=\{a_{1},a_{2},\dots\}\)
mit \(a_{j}\neq a_{i}\,(i\neq j)\). Dann: \(\mdq^{d}=\bigcup\{a_{j}\}\) %\bigcupdot

Dann gilt: \(\mdq^{d}\in\fb_{d}\) und \(\lambda_{d}(\mdq^{d})=\sum{\lambda_{d}(\{a_{j}\})}=0\).
\item Wie in Beispiel (3): Ist \(A\subseteq\mdr^{d}\) abzählbar, so ist
\(A\in\fb_{d}\) und \(\lambda_{d}(A)=0\).
\item Sei \(j\in\{1,\dots,d\}\) und \(H_{j}:=\Set{(x_{1},\dots,x_{d})\in\mdr^{d} | x_{j}=0}\). \(H_{j}\) ist abgeschlossen, damit folgt: \(H_{j}\in\fb_{d}\).

Ohne Beschränkung der Allgemeinheit sei \(j=d\). Dann:
\(I_{n}:=\underbrace{[-n,n]\times\dots\times[-n,n]}_{(d-1)-\text{mal}}\times\{0\}\).
% Hier fehlt noch eine Graphik
Aus Beispiel (1) folgt: \(\lambda_{d}(I_{n})=0\).

Aus \(H_{d}=\bigcup{I_{n}}\) folgt: \(\lambda_{d}(H_{d})\leq\sum{\lambda_{d}(I_{n})}=0\). Also: \(\lambda_{d}(H_{j})=0\).
\end{enumerate}
\end{beispieleX}

\begin{definition}
    Sei $x\in\mdr^d, \emptyset \neq A\subseteq\mdr^d$. Definiere:
    \begin{align*}
        x+A          &:= \Set{x+a | a \in A}\\
        x+ \emptyset &:= \emptyset
    \end{align*}
\end{definition}

\begin{beispiel}
Ist $I\in\ci_d$, so gilt $x+I\in\ci_d$ und $\lambda_d(x+I)=\lambda_d(I)$.
\end{beispiel}

\begin{satz}
\label{Satz 2.8}
Sei $x\in\mdr^d, \fa:=\{B\in\fb_d:x+B\in\fb_d\}$ und $\mu:\fa\to[0,\infty]$ sei definiert durch $\mu(A):=\lambda_d(x+A)$. Dann gilt:
\begin{enumerate}
\item $(\mdr^d,\fa,\mu)$ ist ein Maßraum.
\item Es ist $\fa=\fb_d$ und $\mu=\lambda_d$ auf $\fb_d$. D.h. für alle $A\in\fb_d$ ist $x+A\in\fb_d$ und $\lambda_d(x+A)=\lambda_d(A)$ (Translationsinvarianz des Lebesgue-Maßes).
\end{enumerate}
\end{satz}

\begin{beweis}
\begin{enumerate}
\item Leichte Übung!
\item Es ist klar, dass $\fb_d\supseteq\fa$. Nach dem Beispiel von oben gilt:
\[\ci_d\subseteq\fa\subseteq\fb_d=\sigma(\ci_d)\subseteq\sigma(\fa)=\fa\]
Setze $\ce:=\ci_d$, dann ist $\sigma(\ce)=\fb_d$ und es gilt nach dem Beispiel von oben:
\[\forall E\in\ce:\mu(E)=\lambda_d(E)\]
$\ce$ hat die Eigenschaften (1) und (2) aus Satz \ref{Satz 2.6}, daraus folgt dann, dass $\mu=\lambda_d$ auf $\fb_d$ ist.
\end{enumerate}
\end{beweis}

Ohne Beweis:
\begin{satz}
    \label{Satz 2.9}
    Sei $\mu$ ein Maß auf $\fb_d$ mit der Eigenschaft:
    \[\forall x\in\mdr^d, A\in\fb_d:\mu(A)=\mu(x+A)\]
    Weiter sei $c:=\mu((0,1]^d)<\infty$. Dann gilt:
    \[\mu=c\cdot\lambda_d\]
    Falls $c=1$, so ist $\mu$ das Lebesgue-Maß.
\end{satz}

\begin{satz}[Regularität des Lebesgue-Maßes]
\label{Satz 2.10}
Sei $A \in\fb_d$, dann gilt:
\begin{enumerate}
\item
$\lambda_d(A)
 =\inf\Set{\lambda_d(G) | G\subseteq\mdr^d\text{ offen und }A \subseteq G}\\
 =\inf\Set{\lambda_d(V) | V=\bigcup_{j=1}^\infty I_j, I_j\subseteq\mdr^d\text{ offenes Intervall }, A\subseteq V}$
\item $\lambda_d(A)=\sup\Set{\lambda_d(K) | K\subseteq\mdr^d\text{ kompakt }, K\subseteq A}$
\end{enumerate}
\end{satz}

\begin{beweis}
\begin{enumerate}
\item Ohne Beweis.
\item Setze $\beta:=\sup\Set{\lambda_d(K) | K\subseteq\mdr^d\text{ kompakt }, K\subseteq A}$.
      Sei $K$ kompakt und $K\subseteq A$, dann gilt $\lambda_d(K)\le\lambda_d(A)$, also ist auch $\beta\le\lambda_d(A)$.

\textbf{Fall 1:} Sei $A$ zusätzlich beschränkt.\\
Sei $\ep>0$. Es existiert ein $r>0$, sodass $A\subseteq B:=\overline{U_r(0)}\subseteq[-r,r]^d$ ist, dann gilt:
\[\lambda_d(A)\le\lambda_d([-r,r]^d)=(2r)^d<\infty\]
Aus (1) folgt, dass eine offene Menge $G\supseteq B\setminus A$ existiert mit $\lambda_d(G)\le\lambda_d(B\setminus A)+\ep$. Dann gilt nach \ref{Satz 1.7}:
\[\lambda_d(B\setminus A)=\lambda_d(B)-\lambda_d(A)\]
Setze nun $K:=B\setminus G=B\cap G^c$, dann ist $K$ kompakt und $K\subseteq B\setminus(B\setminus A)=A$. Da $B\subseteq G\cup K$ ist, gilt:
\[\lambda_d(B)\le\lambda_d(G\cup K)\le \lambda_d(B)-\lambda_d(A)+\ep+\lambda_d(K)\]
Woraus folgt:
\[\lambda_d(A)\le\lambda_d(K)+\ep\]

\textbf{Fall 2:} Sei $A\in\fb_d$ beliebig.\\
Setze $A_n:=A\cap\overline{U_n(0)}$. Dann ist $A_n$ für alle $n\in\mdn$ beschränkt, $A_n\subseteq A_{n+1}$ und $A=\bigcup_{n\in\mdn} A_n$. Nach \ref{Satz 1.7} gilt:
\[\lambda_d(A)=\lim\lambda_d(A_n)\]
Aus Fall 1 folgt, dass für alle $n\in\mdn$ ein kompaktes $K_n\subseteq A_n$ mit $\lambda_d(A_n)\le\lambda_d(K_n)+\frac1n$ existiert. Dann gilt:
\[\lambda_d(A_n)\le\lambda_d(K_n)+\frac1n\le\lambda_d(A)+\frac1n\]
Also auch:
\[\lambda_d(A)=\lim\lambda(K_n)\le\beta\]
\end{enumerate}
\end{beweis}

\textbf{Auswahlaxiom:}\\
Sei $\emptyset\ne\Omega$ Indexmenge, es sei $\Set{X_\omega | \omega\in\Omega}$
ein disjunktes System von nichtleeren Mengen $X_\omega$. Dann
existiert ein $C\subseteq\bigcup_{\omega\in\Omega}X_\omega$, sodass
$C$ mit jedem $X_j$ genau ein Element gemeinsam hat.

\begin{satz}[Satz von Vitali]
\label{Satz 2.11}
Es existiert ein $C\subseteq\mdr^d$ sodass $C\not\in\fb_d$.
\end{satz}

\begin{beweis}
Wir definieren auf $[0,1]^d$ eine Äquivalenzrelation $\sim$, durch:
\begin{align*}
\forall x,y\in[0,1]^d: x \sim y\iff x-y\in\mdq^d\\
\forall x\in[0,1]^d:[x]:=\Set{y\in[0,1]^d | x\sim y}
\end{align*}
Nach dem Auswahlaxiom existiert ein $C\subseteq[0,1]^d$, sodass $C$ mit jedem $[x]$ genau ein Element gemeinsam hat.
Es ist $\mdq^d\cap[-1,1]^d=\{q_1,q_2,\dots\}$ mit $q_i\ne q_j$ für $(i\ne j)$. Dann gilt:
\begin{align*}
\tag{1} \bigcup_{n=1}^\infty(q_n+C)\subseteq[-1,2]^d\\
\tag{2} [0,1]^d\subseteq\bigcup_{n=1}^\infty(q_n+C)
\end{align*}
\begin{beweis}
Sei $x\in[0,1]^d$. Wähle $y\in C$ mit $y\in[x]$, dann ist $x\sim y$, also $x-y\in\mdq^d\cap[-1,1]^d$. D.h.:
\[\exists n\in\mdn: x-y=q_n\implies x=q_n+y\in q_n+C\]
\end{beweis}
Außerdem ist $\Set{q_n+C | n\in\mdn}$ disjunkt.
\begin{beweis}
Sei $z\in(q_n+C)\cap(q_m+C)$, dann existieren $a,b\in\mdq^d$, sodass gilt:
\begin{align*}
(q_n+a=z=q_m+b) &\implies (b-a=q_m-q_n\in\mdq^d)\\
&\implies (a\sim b) \implies([a]=[b])\\
&\implies (a=b)\implies (q_n=q_m)
\end{align*}
\end{beweis}
\textbf{Annahme:} $C\in\fb_d$, dann gilt nach (1):
\begin{align*}
3^d&=\lambda_d([-2,1]^d)\\
&\ge\lambda_d(\bigcup(q_n+C))\\
&=\sum \lambda_d(q_n+C)\\
&=\sum \lambda_d(C)
\end{align*}
Also ist $\lambda_d(C)=0$. Damit folgt aus (2):
\begin{align*}
1&=\lambda_d([0,1]^d)\\
&\le \lambda_d(\bigcup (q_n+C))\\
&=\sum \lambda_d(C)\\
&=0
\end{align*}
\end{beweis}


In diesem Kapitel seien $\emptyset\ne X,Y,Z$ Mengen.

\begin{definition}
\index{messbar!Raum}\index{Raum!messbarer}
Ist $\fa$ eine $\sigma$-Algebra auf $X$, so heißt $(X,\fa)$ ein \textbf{messbarer Raum}.
\end{definition}

\begin{definition}
\index{$\fa$-$\fb$-messbar}
\index{messbar!Funktion}
Sei $\fa$ eine $\sigma$-Algebra auf $X$, $\fb$ eine $\sigma$-Algebra auf $Y$ und $f:X\to Y$ eine Funktion. $f$ heißt genau dann \textbf{$\fa$-$\fb$-messbar}, wenn gilt:
\[\forall B\in\fb: f^{-1}(B)\in\fa\]
\end{definition}

\begin{bemerkung}
Seien die Bezeichnungen wie in obiger Definition, dann gilt:
\begin{enumerate}
\item $f$ sei $\fa$-$\fb$-messbar, $\fa'$ eine weitere $\sigma$-Algebra auf $X$ mit $\fa\subseteq\fa'$ und $\fb'$ sei eine $\sigma$-Algebra auf $Y$ mit $\fb'\subseteq\fb$.\\
Dann ist $f$ $\fa'$-$\fb'$-messbar.
\item Sei $X_0\in\fa$, dann gilt $\fa_{X_0}\subseteq\fa$ nach 
\ref{Satz 1.5}. Nun sei $f:X\to Y$ $\fa$-$\fb$-messbar, dann ist 
$f_{\mid X_0}:X_0\to Y$ $\fa_{X_0}$-$\fb$-messbar.
\end{enumerate}
\end{bemerkung}

\begin{beispiel}
\begin{enumerate}
\item Sei $\fa$ eine $\sigma$-Algebra auf $X$ und $A\subseteq X$. $\mathds{1}_A:X\to\mdr$ ist genau dann $\fa$-$\fb_1$-messbar, wenn $A\in\fa$ ist.
\item Sei $X=\mdr^d$. Ist $A\in\fb_d$, so ist $\mathds{1}_A$ $\fb_d$-$\fb_1$-messbar.
\item Ist $C$ wie in \ref{Satz 2.11}, so ist $\mathds{1}_C$ nicht $\fb_d$-$\fb_1$-messbar.
\item Es sei $f:X\to Y$ eine Funktion und $\fb$ ($\fa$) eine $\sigma$-Algebra auf $Y$ ($X$), dann ist $f$ $\cp(X)$-$\fb$-messbar ($\fa$-$\{Y,\emptyset\}$-messbar).
\end{enumerate}
\end{beispiel}

\begin{satz}
\label{Satz 3.1}
Seien \(\fa,\,\fb,\,\fc\) \(\sigma\)-Algebren auf \(X,\,Y\) bzw. \(Z\). Weiter seien \(f:\,X\to Y\) und \(g:\,Y\to Z\)
Funktionen.
\begin{enumerate}
\item Ist \(f\) \(\fa-\fb-\)messbar und ist \(g\) \(\fb-\fc-\)messbar, so ist \(g\circ f:\,X\to Z\) \(\fa-\fc-\)messbar.
\item Sei \(\emptyset\neq\ce\subseteq\cp(Y)\) und \(\sigma(\ce)=\fb\). Dann:
\begin{center}
\(f\) ist \(\fa-\fb-\)messbar, genau dann, wenn gilt: \(\forall E\in\ce:\,f^{-1}(E)\in\fa\)
\end{center}
\end{enumerate}
\end{satz}

\begin{beweis}
\begin{enumerate}
\item Sei \(C\in\fc\); \(g\) ist messbar, daraus folgt \(g^{-1}(C)\in\fb\);
\(f\) ist messbar, daraus folgt \(f^{-1}(g^{-1}(C))=(g\circ f)^{-1}(C)\in\fa\)
\item \begin{itemize}
\item[\(\Rightarrow\)] \checkmark
\item[\(\Leftarrow\)] \(\fd:=\Set{B\subseteq Y | f^{-1}(B)\in\fa}\)
Übung: \(\fd\) ist eine \(\sigma\)-Algebra auf \(Y\).

Aus der Voraussetzung folgt: \(\ce\subseteq\fd\).
Dann: \(\fb=\sigma(\ce)\subseteq\fd\). Ist \(B\in\fb\), so ist \(B\in\fd\), also
\(f^{-1}(B)\in\fa\).
\end{itemize}
\end{enumerate}
\end{beweis}

\begin{definition}
\index{messbar!Borel}\index{messbar}
Sei \(X\in\fb_{d}\). Ist \(f:\,X\to\mdr^{k}\) \(\fb(X)-\fb_{k}-\)messbar, so heißt \(f\) \textbf{(Borel-)messbar}.
\end{definition}
Ab jetzt sei stets \(\emptyset \neq X\in\fb_{d}\). 
(Erinnerung: \(\fb(X)=\Set{A\in\fb_{d} | A\subseteq X}\))

\begin{satz}
\label{Satz 3.2}
Seien \(f,\,g:\,X\to\mdr^{k}\) Abbildungen und \(\alpha,\beta\in\mdr\).
\begin{enumerate}
    \item Ist \(f\) auf \(X\) stetig, so ist \(f\) messbar.
    \item Ist \(f\) messbar und \(g(x):=\lVert f(x)\rVert\,(x\in X)\), so ist \(g\) messbar.
    \item Ist \(f=(f_{1},\dots,f_{k})\), so gilt: \(f\) ist messbar \(\Leftrightarrow\) alle \(f_{j}\) sind messbar.
    \item Sind \(f\) und \(g\) messbar, so ist \(\alpha f+\beta g\) messbar.
    \item Sei \(k=1\) und \(f\) und \(g\) seien messbar. Dann:
    \begin{enumerate}
        \item \(f \cdot g\) ist messbar
        \item Ist \(f(x)\neq 0 \quad \forall x\in X\), so ist 
              \(\frac{1}{f}\) messbar
        \item \(\Set{x\in X | f(x)\stackrel{>}{\geq} g(x)} \in \fb(X)\)
    \end{enumerate}
\end{enumerate}
\end{satz}

\begin{beweis}
\begin{enumerate}
\item Sei \(G\in\co(\mdr^{k})\). \(f\) ist stetig \folgtnach{§0}: \(f^{-1}(G)\in\co(X)\in\fb(X)\)

\(\sigma(\co(\mdr^{k}))=\fb_{k}\). \folgtnach{\ref{Satz 3.1}.(2)} Behauptung.
\item \(\vp(z) := \lVert z\rVert\quad(z\in\mdr^{k})\); \(\vp\) ist
stetig, also messbar.

Es ist \(g=\vp\circ f\). \folgtnach{\ref{Satz 3.1}.(1)} \(g\) ist messbar.
\item 
    \begin{itemize}
        \item["`\(\Rightarrow:\)"'] Für \(j=1, \dots,k\) sei 
            \(p_{j}:\mdr^{k}\to\mdr\) definiert durch 
            \(p_{j}(x_{1},\dots,x_{k}):=x_{j}\)
            \(p_{j}\) ist stetig, also messbar. Es ist 
            \(f_{j}=p_{j}\circ f\) \folgtnach{\ref{Satz 3.1}.(1)} 
            \(f_{j}\) ist messbar.
        \item["`\(\Leftarrow:\)"'] Sei \(I=(a,b]=\prod_{j=1}^{k}{(a_{j},b_{j}]}\in I_{k}\quad (a=(a_{1},\dots,a_{k}),\,b=(b_{1},\dots,b_{k}),\,a\leq b)\)\\
            Dann: \(f^{-1}(I)=\bigcap_{j=1}^{k}{\underbrace{f_{j}^{-1}(\underbrace{(a_{j},b_{j}]}_{\in\fb_{1}}}_{\in\fb(X)}}\in\fb(X)\)
        \(\sigma(I_{k})=\fb_{k}\) \folgtnach{\ref{Satz 3.1}.(2)} \(f\) ist messbar.
    \end{itemize}
\item \(h:=(f,g):\,X\to\mdr^{2k}\); aus (2): \(h\) ist messbar.

\(\vp(x,y):=\alpha x+\beta y\,(x,y\in\mdr^{k})\)

\(\vp\) ist stetig, also messbar. Es ist \(\alpha f+\beta g=\vp\circ h\)
\folgtnach{\ref{Satz 3.1}.(1)} \(\alpha f+\beta g\) ist messbar.
\item 
\begin{enumerate}
\item \(h:=(f,g):\,X\to\mdr^{2k}\) ist messbar (nach (2)); \(\vp(x,y):=xy\), \(\vp\) ist stetig, also messbar.

Es ist \(fg=\vp\circ h\) \folgtnach{\ref{Satz 3.1}.(1)}  \(fg\) ist messbar.
\item \(\vp(x):=\frac{1}{x}\), \(\vp\) ist stetig auf \(\mdr\setminus\{0\}\), also messbar.

\(\frac{1}{f}=\vp\circ f\) \folgtnach{\ref{Satz 3.1}.(1)}  \(\frac{1}{f}\) ist messbar.
\item \(A:=\Set{x\in X | f(x)\geq g(x)} = \Set{x\in X | f(x)-g(x)\in[0,\infty)}
          =\underbrace{(f-g)}_{\text{messbar nach (3)}}^{-1}(\overbrace{[0,\infty)}^{\in\fb_{1}})\in\fb(X)\)
\end{enumerate}
\end{enumerate}
\end{beweis}

\begin{folgerungen}
\label{Lemma 3.3}
    Seien \(A,\,B\in\fb(X),\,A\cap B=\emptyset\) und \(X=A\cup B\). 
    Weiter seien \(f:A\to\mdr^{k}\) und
    \(g:B\to\mdr^{k}\) messbar.\\
    Dann ist \(h:X\to\mdr^{k}\), definiert durch 
    \[
    h(x):=\begin{cases}f(x)&x\in A\\g(x)&x\in B\end{cases},
    \]
    messbar.
\end{folgerungen}

\begin{beweis}
    Sei \(C\in\fb_{k}\). Dann:
    \[
    h^{-1}(C)=\underbrace{f^{-1}(C)}_{\in\fb(A)\subseteq\fb(X)}\cup\underbrace{g^{-1}(C)}_{\in\fb(B)\subseteq\fb(X)}\in\fb(X)
    \]
\end{beweis}

\begin{beispiel}
\(X=\mdr^{2},\,f(x,y):=\begin{cases}\frac{\sin(y)}{x}&x\neq 0\\0&x=0\end{cases}\)

für \(x\neq 0:\,f(x,x)=\frac{\sin(X)}{x}\overset{x\to 0}{\to}1\neq 0=f(0,0)\), daraus folgt: \(f\) ist nicht stetig.

\(A:=\Set{(x,y)\in\mdr^{2} | x=0},\,B
   :=\Set{(x,y)\in\mdr^{2} | x\neq 0},\,X=A\cup B,\,A\cap B=\emptyset\). \(A\) ist
abgeschlossen, das heißt: \(A\in\fb_{2},\,B=A^{C}\in\fb_{2}\)

\begin{align*}
f_{1}(x,y)&:=0\quad((x,y)\in A)\\
f_{2}(x,y)&:=\frac{\sin(y)}{x}\quad((x,y)\in B)
\end{align*}

\(f_{1}\) ist stetig auf \(A\), \(f_{2}\) ist stetig auf \(B\). Also: \(f_{1},\,f_{2}\) ist messbar; mit \ref{Lemma 3.3} folgt: \(f\) ist messbar.
\end{beispiel}

\textbf{Ein neues Symbol kommt hinzu:} \(-\infty\){

\(\imdr:=[-\infty,+\infty]:=\mdr\cup\{-\infty,+\infty\}\)

In \(\imdr\) gelten folgende Regeln, wobei \(a\in\mdr\):
\begin{enumerate}
    \item \(-\infty<a<+\infty\)
    \item \(\pm\infty+(\pm\infty)=\pm\infty\)
    \item \(\pm\infty+a:=a+(\pm\infty):=\pm\infty\)
    \item \(a\cdot(\pm\infty):=(\pm\infty)\cdot a=
            \begin{cases}
                \pm\infty &a > 0\\
                0         &a = 0\\\mp\infty&a<0
            \end{cases}\)
    \item \(\frac{a}{\pm\infty}:=0\)
\end{enumerate}
}

\begin{definition}
\begin{enumerate}
\item Sei \((x_{n})\) eine Folge in 
\(\imdr\). \(x_{n}\rightarrow+\infty:\Leftrightarrow\forall c\in\mdr\,\exists n_{c}\in\mdn:x_{n}\geq c\quad\forall n\geq n_{c}\)\\
Analog für \(-\infty\).
\item Seien \(f,g: X\to\imdr\) Funktionen. Dann:
\begin{align*}
    \{f\leq g\}&:=\Set{x\in X | f(x)\leq g(x)}\\
    \{f\geq g\}&:=\Set{x\in X | f(x)\geq g(x)}\\
    \{f\neq g\}&:=\Set{x\in X | f(x)\neq g(x)}\\
    \{f<g\}&:=\Set{x\in X | f(x)<g(x)}\\
    \{f>g\}&:=\Set{x\in X | f(x)>g(x)}
\end{align*}
\item Sei \(a\in\imdr\) und \(f:\,X\to\imdr\). Dann:
\begin{align*}
    \{f\leq a\}&:=\Set{x\in X | f(x)\leq a}\\
    \{f\geq a\}&:=\Set{x\in X | f(x)\geq a}\\
    \{f\neq a\}&:=\Set{x\in X | f(x)\neq a}\\
    \{f<a\}    &:=\Set{x\in X | f(x)<a}\\
    \{f>a\}    &:=\Set{x\in X | f(x)>a}
\end{align*}
\end{enumerate}
\end{definition}

\begin{definition}
\index{Borel!$\sigma$-Algebra}\index{messbar}
\(\ifb_{1}:=\Set{B\cup E | B\in\fb_{1},\,E\subseteq\Set{-\infty,+\infty}}\). 
Dann: \(\fb_{1}\subseteq\ifb_{1}\)\\
Übung: \(\ifb_{1}\) ist eine \(\sigma\)-Algebra auf \(\imdr\).\\
Klar: \(\fb_{1} \subseteq \ifb_{1}\)
\(\ifb_{1}\) heißt \textbf{Borelsche \(\sigma\)-Algebra} auf \(\imdr\).\\
Sei \(f:\,X\to\imdr\). \(f\) heißt \textbf{(Borel-)messbar} (mb) \(:\Leftrightarrow\,f\) ist \(\fb(X)-\ifb_{1}-\) messbar.
\end{definition}

\begin{beispiel}
\(f: X \rightarrow \bar \mdr\) definiert durch \(f(x):=+\infty\quad(x\in X)\), also: \(f:\,X\to\imdr\)

Sei \(B\in\overline{\fb}_{1},\,A:=f^{-1}(B)=\Set{x\in X | f(x)\in B}\)
\begin{itemize}
\item[Fall 1:] \(+\infty\not\in B\), dann: \(A=\emptyset\in\fb(X)\)
\item[Fall 2:] \(+\infty\in B\), dann: \(A=X\in\fb(X)\)
\end{itemize}
\(f\) ist messbar.
\end{beispiel}

\begin{satz}
\label{Satz 3.4}
\begin{enumerate}
\item Definiere die Mengen:
\begin{align*}
\ce_1&:=\Set{[-\infty,a] | a\in\mdq} & \ce_2&:=\Set{[-\infty,a) | a\in\mdq}\\
\ce_3&:=\Set{(a,\infty] | a\in\mdq} & \ce_4 &:=\Set{[a,\infty] | a\in\mdq}
\end{align*}
Dann gilt:
\[\overline{\fb_1}=\sigma(\ce_j)\quad \text{ für }j\in\{1,2,3,4\}\]
\item Für $f:X\to\imdr$ sind die folgenden Aussagen äquivalent:
\begin{enumerate}
\item $f$ ist messbar.
\item $\forall a\in\mdq: \{f\le a\}\in\fb(X)$.
\item $\forall a\in\mdq: \{f\ge a\}\in\fb(X)$.
\item $\forall a\in\mdq: \{f< a\}\in\fb(X)$.
\item $\forall a\in\mdq: \{f> a\}\in\fb(X)$.
\end{enumerate}
\item Die Äquivalenzen in (2) gelten auch für Funktionen $f:X\to\mdr$.
\end{enumerate}
\end{satz}

\begin{beweis}
Die folgenden Beweise erfolgen exemplarisch für einen der Unterpunkte und funktionieren fast analog für die anderen.
\begin{enumerate}
    \item Für $a\in\mdq$ gilt:
    \[[-\infty,a]^c=(a,\infty]\in\sigma(\ce_1)\]
    D.h. es gilt $\ce_3\subseteq\sigma(\ce_1)$ und damit auch $\sigma(\ce_3)\subseteq\sigma(\ce_1)$.
    \item Es gilt:
    \[\forall a \in \mdq\colon \{f\le a\}=\Set{x\in X | f(x)\le a}=f^{-1}(\underbrace{[-\infty,a]}_{\ce_1}) (*)\]
    Die Äquivalenz folgt dann aus (1) und \ref{Satz 3.1}.
    \item Die Funktion $f:X\to\imdr$ kann aufgefasst werden als Funktion $\overline{f}:X\to\imdr$. Es ist $f$ genau dann $\fb(X)$-$\fb_1$-messbar wenn $\overline{f}$ $\fb(X)$-$\overline{\fb_1}$-messbar ist. 
\end{enumerate}
\end{beweis}

\begin{bemerkung}\ 
\begin{enumerate}
\item Ist $X \subseteq \mdr$ ein Intervall und $f: \bar X \rightarrow \mdr$ monoton, so ist
      $f$ messbar (vgl. 3. ÜB)
\item Wir wissen: $f: X \rightarrow \mdr$ mb $\Rightarrow |f|$ ist mb.
      Die Umkehrung ist im allgemeinen falsch!
\end{enumerate}
\end{bemerkung}

\begin{beispiel}
Sei $C \subseteq \mdr^d$ wie in 2.11, also $C \notin \fb_1$.
\[f(x) = \begin{cases}
1 & x \in C\\
0 & x \notin C
\end{cases}\\
\Set{f \geq 1} = \Set{x \in \mdr^d | f(x) \geq 1} = C \notin \fb \folgtnach{\ref{Satz 3.4}.(2)} f \text{ ist nicht mb.}\]
Es ist $|f(x)|=1 \quad \forall x \in \mdr^d$, also $|f| = \mathds{1}_{\mdr^d}$. D.h. $|f|$ ist mb.
\end{beispiel}

\begin{definition}
Sei $M\subseteq\imdr$.
\begin{enumerate}
\item Ist $M=\emptyset$ oder $M=\{-\infty\}$, so sei 
\[\sup M:=-\infty\]
\item Ist $M\setminus\{-\infty\}\ne\emptyset$ und nach oben beschränkt (also insbesondere $\infty\not\in M$), so sei 
\[\sup M:= \sup (M\setminus\{-\infty\})\]
\item Ist $M\setminus\{-\infty\}$ nicht nach oben beschränkt oder $\infty\in M$, so sei 
\[\sup M:=\infty\]
\item Es sei $\inf M:=-\sup(-M)$, wobei $-M:=\Set{-m | m\in M}$.
\end{enumerate}
\end{definition}

\begin{definition}
Sei $(f_n)$ eine Folge von Funktionen $f_n:X\to\imdr$.
\begin{enumerate}
\item Die Funktion $\sup_{n\in\mdn}(f_n):X\to\imdr$  $\left(\inf_{n\in\mdn}(f_n):X\to\imdr\right)$ ist definiert durch:
\[(\sup_{n\in\mdn} f_n)(x):=\sup\Set{f_n(x) | n\in\mdn}\quad x\in X\]
\[\left((\inf_{n\in\mdn} f_n)(x):=\inf\Set{f_n(x) | n\in\mdn}\quad x\in X\right)\]
\item Die Funktion $\limsup_{n\to\infty} f_n:X\to\imdr$ $\left(\liminf_{n\to\infty} f_n:X\to\imdr\right)$ ist definiert durch:
\begin{align*}
\tag{$*$} \limsup_{n\to\infty} f_n &:= \inf_{j\in\mdn}(\sup_{n\ge j} f_n)\\
\liminf_{n\to\infty} f_n &:= \sup_{j\in\mdn}(\inf_{n\ge j} f_n)
\end{align*}
\textbf{Erinnerung:} Für eine beschränkte Folge $(a_n)$ in $\mdr$ war
\[\limsup_{n\to\infty} a_n:=\inf\{\sup\Set{a_n | n\ge j}\mid j\in\mdn\}\]
\item Sei $N\in\mdn$ und $g_j:=f_j$ (für $j=1,\dots,N$), $g_j:=f_N$ (für $j>N$). Definiere:
\begin{align*}
\max_{1\le n\le N} f_n &:=\sup_{j\in\mdn} g_n\\
\min_{1\le n\le N} f_n &:=\inf_{j\in\mdn} g_n
\end{align*}
\item Ist $f_n(x)$ für jedes $x\in\imdr$ konvergent, so ist $\lim_{n\to\infty} f_n:X\to\imdr$ definiert durch:
\[(\lim_{n\to\infty} f_n)(x):=\lim_{n\to\infty} f_n(x)\]
(In diesem Fall gilt $\lim_{n\to\infty} f_n = \limsup_{n\to\infty} f_n = \liminf_{n\to\infty} f_n$.)
\end{enumerate}
\end{definition}

\begin{satz}
\label{Satz 3.5}
Sei $(f_n)$ eine Folge von Funktionen $f_n:X\to\imdr$ und jedes $f_n$ messbar.
\begin{enumerate}
\item Dann sind ebenfalls messbar:
\begin{align*}
&\sup_{n\in\mdn} f_n  &&\inf_{n\in\mdn} f_n &&\limsup_{n\in\mdn} f_n &&\liminf_{n\in\mdn} f_n
\end{align*}
\item Ist $(f_n(x))$ für jedes $x\in X$ in $\imdr$ konvergent, so ist $\lim_{n\to\infty} f_n$ messbar.
\end{enumerate}
\end{satz}

\begin{beweis}
\begin{enumerate}
\item Sei $a\in\mdq$, dann gilt (nach \ref{Satz 3.4}(2)):
\[\{\sup_{n\in\mdn} f_n\le a\}=\bigcap_{n\in\mdn}\{f_n\le a\}\in\fb(X)\]
Also ist $\sup_{n\in\mdn} f_n$ messbar. Analog lässt sich die Messbarkeit von $\inf_{n\in\mdn} f_n$ zeigen, der Rest folgt dann aus ($*$).
\item Folgt aus (1) und obiger Bemerkung in der Definition.
\end{enumerate}
\end{beweis}

\begin{beispiel}
Sei $X=I$ ein Intervall in $\mdr$ und $f:I\to\mdr$ sei auf $I$ differenzierbar.\\
Für $x\in I,n\in\mdn$ sei $f_n:= n(f(x-\frac1n)-f(x))$. Da $f$ stetig ist, ist auch jedes $f_n$ stetig, also insbesondere messbar und es gilt:
\[f_n(x)=\frac{f(x-\frac1n)-f(x)}{\frac1n}\stackrel{n\to\infty}{\to}f'(x)\]
Aus \ref{Satz 3.5}(2) folgt, dass $f'$ messbar ist. 
\end{beispiel}

\begin{definition}
\index{Positivteil}\index{Negativteil}
Sei $f:X\to\imdr$ eine Funktion.
\begin{enumerate}
\item $f_+:=\max\{f,0\}$ heißt \textbf{Positivteil} von $f$.
\item $f_-:=\max\{-f,0\}$ heißt \textbf{Negativteil} von $f$.
\end{enumerate}
Es gilt $f_+,f_-\ge 0$, $f=f_+-f_-$ und $|f|=f_++f_-$.
\end{definition}

\begin{satz}
\label{Satz 3.6}
Seien $f,g:X\to\imdr$ und $\alpha,\beta\in\mdr$.
\begin{enumerate}
\item Sind $f,g$ messbar und ist $\alpha f(x)+\beta g(x)$ für jedes $x\in X$ definiert, so ist $\alpha f+\beta g$ messbar.
\item Sind $f,g$ messbar und ist $f(x)g(x)$ für jedes $x\in X$ definiert, so ist $fg$ messbar.
\item $f$ ist genau dann messbar, wenn $f_+$ und $f_-$ messbar sind. In diesem Fall ist auch $|f|$ messbar.
\end{enumerate}
\end{satz}

\begin{beweis}
\begin{enumerate}
\item[(1)+(2)] Für alle $n\in\mdn, x\in X$ seien $f_n$ und $g_n$ wie folgt definiert:
\begin{align*}
f_n(x)&:=\max\{-n,\min\{f(x),n\}\}\\
g_n(x)&:=\max\{-n,\min\{g(x),n\}\}
\end{align*}
Dann sind $f_n(x),g_n(x)\in[-n,n]$ für alle $n\in\mdn,x\in X$. Nach \ref{Satz 3.2}(3) sind also $\alpha f_n+\beta g_n$ und $f_ng_n$ messbar. Außerdem gilt:
\begin{align*}
\alpha f_n(x)+\beta g_n(x)&\stackrel{n\to\infty}\to \alpha f(x)+\beta g(x)\\
f_n(x)g_n(x)&\stackrel{n\to\infty}\to f(x)g(x)
\end{align*}
Die Behauptung folgt aus \ref{Satz 3.5}(2).
\item[(3)] Nach \ref{Satz 3.5}(1) sind $f_+$ und $f_-$ messbar, wenn $f$ messbar ist. Die umgekehrte Implikation folgt aus \ref{Satz 3.6}(1). Sind $f_+$ und $f_-$ messbar, so folgt ebenfalls aus \ref{Satz 3.6}(1), dass $|f|=f_++f_-$ messbar ist.
\end{enumerate}
\end{beweis}

\begin{beispiel}
Sei $C\subseteq\mdr^d$ wie in \ref{Satz 2.11}, also $C\not\in\fb_d$. Definiere $f:\mdr^d\to\mdr$ wie folgt:
\[f(x):=\begin{cases} 1&,x\in C\\ -1&,x\not\in C\end{cases}\]
Dann ist $\{f\ge 1\}=C$, also $f$ \textbf{nicht} messbar. Aber für alle $x\in\mdr^d$ ist $|f(x)|=1$, also $|f|=\mathds{1}_{\mdr^d}$ und damit messbar.
\end{beispiel}

\begin{definition}
\index{einfach}
\index{Treppenfunktion}
\index{Normalform}
$f:X\to\mdr$ sei messbar.
\begin{enumerate}
\item $f$ heißt \textbf{einfach} oder \textbf{Treppenfunktion}, genau dann wenn $f(X)$ endlich ist.
\item $f$ sei einfach und $f(X)=\{y_1,\dots,y_m\}$ mit $y_i\ne y_j$ für $i\ne j$. Sei weiter $A_j:=f^{-1}(\{y_j\})$ für $j=1,\dots,m$. Dann sind $A_1,\dots,A_m\in\fb(X)$ und $X=\bigcup_{j=1}^m A_j$ disjunkte Vereinigung.
\[f=\sum_{j=1}^m y_j \mathds{1}_{A_j}\]
heißt \textbf{Normalform} von $f$.
\end{enumerate}
\end{definition}

\begin{beispiel}
Sei $A\in\fb(X)$. Definiere:
\[f:=\mathds{1}_A=2\cdot\mathds{1}_A-\mathds{1}_X+\mathds{1}_{X\setminus A}=\mathds{1}_A+0\cdot\mathds{1}_{X\setminus A}\]
Wobei das letzte die Normalform von $f$ ist. Man sieht also, dass einfache Funktionen mehrere Darstellungen haben können.
\end{beispiel}

\begin{satz}
\label{Satz 3.7}
Linearkombinationen und Produkte, sowie endliche Maxima und Minima einfacher Funktionen, sind einfach.
\end{satz}

\begin{satz}
\label{Satz 3.8}
\index{zulässig}
Sei $f:X\to\imdr$ messbar.
\begin{enumerate}
\item Ist $f\ge 0$ auf $X$, so existiert eine Folge $(f_n)$ von einfachen Funktionen $f_n:X\to[0,\infty)$, sodass $0\le f_n\le f_{n+1}$ auf $X$ ($\forall n\in\mdn$) und $f_n(x)\stackrel{n\to\infty}{\to}f(x)$ ($\forall x\in X$). In diesem Fall heißt $(f_n)$ \textbf{zulässig} für $f$.
\item Es existiert eine Folge $(f_n)$ von einfachen Funktionen $f_n:X\to\mdr$, sodass $|f_n|\le |f|$ auf $X$ ($\forall n\in\mdn$) und $f_n(x)\stackrel{n\to\infty}{\to}f(x)$ ($\forall x\in X$).
\item Ist $f$ beschränkt auf $X$ (also insbesondere $\pm\infty\not\in f(X)$), so kommt in (2) noch hinzu, dass $(f_n)$ auf $X$ gleichmäßig gegen $f$ konvergiert.
\end{enumerate}
\end{satz}

\begin{folgerungen}[(Beweis mit 3.8(2) und 3.5)]
Sei $f:X\to\imdr$ eine Funktion, dann ist $f$ genau dann messbar, wenn eine Folge einfacher Funktionen $(f_n)$ mit $f_n:X\to\mdr$ und $f_n(x)\stackrel{n\to\infty}\to f(x)$ für alle $x\in X$ existiert.
\end{folgerungen}

\begin{beweis}
\begin{enumerate}
\item Für $n\in\mdn$ definiere $\varphi_n:[0,\infty]\to[0,\infty)$ durch
\[\varphi_n(t):=\begin{cases}\frac{[2^nt]}{2^n} &,0\le t<n\\ n &,n\le t\le\infty\end{cases}\]
Dann ist $\varphi_n$ $(\fb_1)_{[0,\infty]}$-$\fb_1$-messbar, außerdem gilt:
\begin{align*}
\forall t\in[0,\infty]\forall n\in\mdn&: 0\le\varphi_1\le\dots\le t\\
\forall t\in[0,n]\forall n\in\mdn&: t-\frac1{2^n}\le\varphi_n(t)\le t 
\end{align*}
und es ist $\varphi_n(t)\stackrel{n\to\infty}\to t$ für alle $t\in[0\infty]$. Setze $f_n:=\varphi_n\circ f$. Dann leistet $(f_n)$ das gewünschte.
\item Es ist $f=f_+-f_-$ und $f_+,f_-\ge0$ auf $X$. Seien $(g_n),(h_n)$ zulässige Folgen für $f_+$ bzw. $f_-$. Definiere $f_n:=g_n-h_n$. Dann ist klar, dass gilt:
\[\forall x\in X: f_n(x)=g_n(x)-h_n(x)\stackrel{n\to\infty}\to f_+(x)-f_-(x)=f(x)\]
Weiter gilt:
\[|f_n|\le g_n+h_n\le f_++f_-=|f|\]
\item Ohne Beweis. 
\end{enumerate}
\end{beweis}

\clearpage
In diesem Kapitel sei $\emptyset\ne X\in\fb_d$. Wir schreiben außerdem $\lambda$ statt $\lambda_d$.

\begin{definition}
\index{Lebesgueintegral}
Sei $f:X\to [0,\infty)$ eine einfache Funktion mit der Normalform $f=\sum_{j=1}^m y_j\mathds{1}_{A_j}$.\\
Das \textbf{Lebesgueintegral} von $f$ ist definiert durch:
\[\int_X f(x)\text{ d}x:=\sum_{j=1}^m y_j\lambda(A_j)\]
\end{definition}

\begin{satz}
\label{Satz 4.1}
Sei $f:X\to[0,\infty)$ einfach, $z_1,\dots,z_k\in[0,\infty)$ und $B_1,\dots,B_k\in\fb(X)$ mit $\bigcup B_j=X$ und $f=\sum_{j=1}^k z_j\mathds{1}_{B_j}$. Dann gilt:
\[\int_X f(x)\text{ d}x=\sum_{j=1}^k z_j\lambda(B_j)\]
\end{satz}

\begin{beweis}
In der großen Übung.
\end{beweis}

\begin{satz}
\label{Satz 4.2}
Seien $f,g:X\to[0,\infty)$ einfach, $\alpha, \beta\in[0,\infty)$ und $A\in\fb(X)$.
\begin{enumerate}
\item $\int_X \mathds{1}_A(x)\text{ d}x=\lambda(A)$
\item $\int_X (\alpha f+\beta g)(x)\text{ d}x = \alpha\int_X f(x)\text{ d}x + \beta\int_X g(x)\text{ d}x$
\item Ist $f\le g$ auf $X$, so ist $\int_X f(x)\text{ d}x\le \int_X g(x)\text{ d}x$.
\end{enumerate}
\end{satz}

\begin{beweis}
\begin{enumerate}
\item Folgt aus der Definition und \ref{Satz 4.1}.
\item Es seien $f=\sum_{j=1}^m y_j \mathds{1}_{A_j}$ und $g=\sum_{j=1}^k z_j \mathds{1}_{B_j}$ die Normalformen von $f$ und $g$. Dann gilt:
\[\alpha f+ \beta g=\sum_{j=1}^m \alpha y_j\mathds{1}_{A_j}+\sum_{j=1}^k \beta z_j\mathds{1}_{B_j}\]
Dann gilt:
\begin{align*}
\int_X (\alpha f+\beta g) &\stackrel{\ref{Satz 4.1}}= \sum_{j=1}^m \alpha y_j \lambda(A_j) + \sum_{j=1}^k \beta z_j \lambda(B_j)\\
&= \alpha \sum_{j=1}^m y_j \lambda(A_j) + \beta \sum_{j=1}^k z_j \lambda(B_j)\\
&= \alpha \int_X f(x)\text{ d}x + \beta \int_X g(x)\text{ d}x
\end{align*}
\item Definiere $h:=g-f$. Dann ist $h\ge 0$ und einfach. Sei $h=\sum_{j=1}^m x_j\mathds{1}_{C_j}$ die Normalform von $h$, d.h. $x_1,\dots,x_m\ge 0$. Dann gilt:
\[\int_X h(x)\text{ d}x = \sum_{j=1}^m x_j\lambda(C_j)\ge 0\]
Also folgt aus $g=f+h$ und (2):
\[\int_X g(x)\text{ d}x=\int_X f(x)\text{ d}x +\int_X h(x)\text{ d}x\ge \int_X f(x)\text{ d}x\]
\end{enumerate}
\end{beweis}

\begin{definition}
\index{Lebesgueintegral}
Sei $f:X\to[0,\infty]$ messbar. $(f_n)$ sei eine für $f$ zulässige Folge. Das \textbf{Lebesgueintegral} von $f$ ist definiert als:
\begin{align*}
\tag{$*$}\int_X f(x)\text{ d}x:=\lim_{n\to\infty}\int_X f_n(x)\text{ d}x
\end{align*}
\end{definition}

\begin{bemerkung}\ 
\begin{enumerate}
\item In \ref{Satz 4.3} werden wir sehen, dass $(*)$ unabhängig ist von der Wahl der für $f$ zulässigen Folge $(f_n)$.
\item $(f_n(x))$ ist wachsend für alle $x\in X$, d.h.:
\[f(x)=\lim_{n\to\infty} f_n(x)=(\sup_{n\in\mdn} f_n)(x)\]
\item Aus \ref{Satz 4.2}(3) folgt dass $(\int_X f_n(x)\text{ d}x)$ wachsend ist, d.h.:
\[\lim_{n\to\infty} \int_X f_n(x)\text{ d}x = \sup\Set{\int_X f_n(x)\text{ d}x | n\in\mdn}=\int_X f_(x)\text{ d}x\]
\end{enumerate}
\end{bemerkung}

\textbf{Bezeichnung:}\\
Für messbare Funktionen $f:X\to[0,\infty]$ definiere
\[M(f):=\Set{\int_X g\text{ d}x\mid g:X\to[0,\infty) \text{ einfach und }g\le f\text{ auf }X}\]

\begin{satz}
\label{Satz 4.3}
Ist $f:X\to[0,\infty]$ messbar und $(f_n)$ zulässig für $f$, so gilt:
\[L:=\lim_{n\to\infty}\int_X f_n\text{ d}x=\sup M(f)\]
Insbesondere ist $\int_X f(x) \text{ d}x$ wohldefiniert.
\end{satz}

\begin{folgerungen}
\label{Folgerung 4.4}
Ist $f:X\to[0,\infty]$ messbar, so ist $\int_X f(x) \text{ d}x=\sup M(f)$.
\end{folgerungen}

\begin{beweis}
Sei \(\int_Xf_n\,dx\in M(f) \,\forall\natn \). Dann ist \[L = \sup\left\{\int_Xf_n\,dx\mid\natn\right\} \leq \sup M(f)\]\\
Sei nun $g$ einfach und \(0\leq g\leq f\). Sei weiter \[g=\sum^m_{j=1}y_j\mathds{1}_{A_j}\] die Normalform von $g$.\\
Sei \(\alpha>1\) und \(B_n:=\{\alpha f_n\geq g\}\). Dann ist \[B_n\in\fb(X) \text{ und }(B_n\subseteq B_{n+1}\text{, sowie } \mathds{1}_{B_n}g\leq\alpha f_n.\]
Sei \(x\in X\).\\
\textbf{Fall 1:} Ist \(f(x)=0\), so ist wegen \(0\leq g\leq f\) auch \(g(x)=0\). Somit ist \(x\in B_n\) für jedes \(\natn\).\\
\textbf{Fall 2:} Ist  \(f(x)>0\), so ist \[\frac{1}{\alpha}g(x)<f(x)\] (Dies ist klar für \(g(x)=0\) und falls gilt: \(g(x)>0\), so ist \(\frac{1}{\alpha}g(x)<g(x)\leq f(x) \) )\\
Da $f_n$ zulässig für $f$ ist, gilt: \(f_n(x)\to f(x)\  (n\to\infty)\), weshalb ein \(n(x)\in\mdn\) existiert mit:
\[\frac{1}{\alpha}g(x)<f(x)\text{für jedes } n\geq n(x)\]
Es folgt \(x\in B_n\) für jedes \(n\geq n(x)\).\\
\textbf{Fazit:} \(X=\bigcup B_n\). \[A_j=A_j\cap X=A_j\cap\left(\bigcup B_n\right) = \bigcup(A_j\cap B_n) \text{ und } A_j\cap B_n\subseteq A_j\cap B_{n+1} \]
Aus \ref{Satz 1.7} folgt \(\lambda(A_j)=\lim\limits_{n\to\infty}\lambda(A_j\cap B_n)\). Das liefert:
\begin{align*}
   \int\limits_Xg\,dx &= \sum\limits_{j=1}^m y_j\lambda(A_j) 
   = \sum\limits_{j=1}^m y_j\lim\limits_{n\to\infty}\lambda(A_j\cap B_n)\\ 
   &=\lim\limits_{n\to\infty}\sum\limits_{j=1}^m y_j\lambda(A_j\cap B_n)
   \overset{\ref{Satz 4.1}}= \lim\limits_{n\to\infty} \int\limits_X \mathds{1}_{B_n}g\,dx\\
   &\leq  \lim\limits_{n\to\infty} \int\limits_X \alpha f_n\,dx
   =\alpha L
\end{align*}
g war einfach und \(0\leq g\leq f\) beliebig, sodass \[\sup M(f)\leq\alpha L \overset{\alpha\to 1}\implies \sup M(f)\leq L \]
\end{beweis}

\begin{satz}
\label{Satz 4.5}
Seien $f,g:X\to[0,\infty]$ messbar und $\alpha,\beta\ge0$.
\begin{enumerate}
\item $\int_X (\alpha f+\beta g)(x) \text{ d}x=\alpha\int_X f(x) \text{ d}x+\beta\int_X g(x) \text{ d}x$
\item Ist $f\le g$ auf $X$, so gilt $\int_X f(x) \text{ d}x\le \int_X g(x) \text{ d}x$
\item $\int_X f(x) \text{ d}x=0 \iff \lambda(\{f>0\})=0$
\end{enumerate}
\end{satz}

\begin{beweis}
\begin{enumerate}
\item \((f_n)\) und \((g_n)\) seien zulässig für $f$ bzw. $g$. Weiter sei \((h_n):=\alpha (f_n)+\beta (g_n) \).
Dann ist wegen \ref{Satz 3.7} und \(\alpha , \beta \geq 0\), dass \((h_n)\) zulässig für \(\alpha f+\beta g\) ist. Dann:
\begin{align*}
\int_X(\alpha f + \beta g)\,dx
&= \lim\limits_{n\to\infty}\int_X \left( \alpha (f_n)+\beta (g_n) \right)\,dx\\
&\overset{\ref{Satz 4.2}}= \alpha\lim\limits_{n\to\infty}\int_X(f_n)\,dx + \beta\lim\limits_{n\to\infty}\int_X(g_n)\,dx\\
&=\alpha\int_Xf\,dx + \beta\int_Xg\,dx
\end{align*}
\item Wegen \(f\leq g\) auf $X$ ist \(M(f)\subseteq M(g)\) und somit auch \(\sup M(f)\leq\sup M(g)\). Aus \ref{Folgerung 4.4} folgt nun die Behauptung.
\item Setze \(A:=\{f>0\}=\{x\in X:f(x)>0\}\).
\begin{enumerate}
\item["'$\implies$"'] Sei \(\int_Xf\,dx=0\) und \(A_n:=\{f>\frac{1}{n}\}\). Dann ist \(A=\bigcup A_n\) und \(f\geq\frac{1}{n}\mathds{1}_{A_n}\). Damit folgt:
\begin{align*}
0 = \int_Xf\,dx 
\overset{\text{(2)}}\geq \int_X\frac1{n}\mathds{1}_{A_n}\,dx
=\frac1{n}\lambda(A_n)
\intertext{Es ist also \(\lambda(A_n)=0\) und damit gilt weiter}
\lambda(A)=\lambda(\bigcup A_n) \overset{\ref{Satz 1.7}}\leq \sum\lambda(A_n)=0
\end{align*}
Also ist auch \(\lambda(A)=0\).
\item["'$\impliedby$"'] Sei \(\lambda(A)=0\), \((f_n)\) zulässig für $f$ und \(c_n:=\max\{f_n(x):x\in X\}\). Dann ist \(f_n\leq c_n\mathds{1}_A\) und es gilt:
\[0 \leq \int_Xf_n\,dx\overset{\text{(2)}} \leq \int_Xc_n\mathds{1}_A\,dx = c_n\lambda(A) \overset{\text{Vor.}} = 0 \]
Es ist also  \(\int_Xf_n\,dx=0\) für jedes $\natn$ und somit auch \(\int_Xf\,dx=0\)
\end{enumerate}
\end{enumerate}
\end{beweis}

\begin{satz}[Satz von Beppo Levi (Version I)]
\label{Satz 4.6}
Sei $(f_n)$ eine Folge messbarer Funktionen $f_n:X\to[0,\infty]$ und es gelte $f_n\le f_{n+1}$ auf $X$ für jedes $n\in\mdn$.
\begin{enumerate}
\item Für alle $x\in X$ existiert $\lim_{n\to\infty} f_n(x)$.
\item Die Funktion $f:X\to[0,\infty]$ definiert durch:
\[f(x):=\lim_{n\to\infty} f_n(x)\]
ist messbar.
\item $\int_X \lim\limits_{n\to\infty}f_n(x) \text{ d}x=\int_X f(x) \text{ d}x=\lim\limits_{n\to\infty}\int_X f_n(x) \text{ d}x$
\end{enumerate}
\end{satz}

\begin{beweis}
\begin{enumerate}
\item Für alle $x\in X$ ist \(\left(f_n(x)\right)\) wachsend, also konvergent in \([0,+\infty]\).
\item folgt aus \ref{Satz 3.5}.
\item Sei \( \left(u_j^{(n)}\right)_{j\in\mdn} \) zulässig für $f_n$ und \(v_j:=\max\left\{u_j^{(1)}, u_j^{(2)}, \dots , u_j^{(j)} \right\} \).
Aus \ref{Satz 3.7} folgt, dass $v_j$ einfach ist und aus der Konstruktion lässt sich nachrechnen, dass gilt:
 \[0\leq v_j\leq v_{j+1} \text{ und } v_j\leq f_n\leq f \text{ und } f_n=\sup\limits_{j\in\mdn}u_j^{(n)} \leq \sup\limits_{j\in\mdn}v_j \text{ (auf $X$)}\]
Damit ist $(v_j)$ zulässig für $f$ und es gilt:
\[ \int_Xf\,dx=\lim\limits_{j\to\infty}\int_Xv_j\,dx\leq\lim\limits_{j\to\infty}\int_Xf_j\,dx\leq\int_Xf\,dx \]
\end{enumerate}
\end{beweis}

\begin{satz}[Satz von Beppo Levi (Version II)]
\label{Satz 4.7}
Sei $(f_n)$ eine Folge messbarer Funktionen $f_n:X\to[0,\infty]$.
\begin{enumerate}
\item Für alle $x\in X$ existiert $s(x):=\sum_{j=1}^\infty f_j(x)$.
\item $s:X\to[0,\infty]$ ist messbar.
\item $\int_X \sum_{j=1}^\infty f_j(x) \text{ d}x= \sum_{j=1}^\infty \int_X f_j(x) \text{ d}x$
\end{enumerate}
\end{satz}

\begin{beweis}
Setze \[s_n:=\sum\limits_{j=1}^nf_j\]
Dann erfüllt \((s_n)\) die Voraussetzungen von \ref{Satz 4.6}. Aus 4.6 und \ref{Satz 4.5}(1) folgt die Behauptung.
\end{beweis}

\begin{satz}
\label{Satz 4.8}
Sei $f:X\to[0,\infty]$ messbar und es sei $\emptyset\ne Y\in\fb(X)$ (also $Y\subseteq X$ und $Y\in\fb_d$). Dann sind die Funktionen $f_{|Y}:Y\to[0,\infty]$ und $\mathds{1}_Y\cdot f:X\to[0,\infty]$ messbar und es gilt:
\[\int_Y f(x) \text{ d}x:=\int_Y f_{|Y}(x) \text{ d}x=\int_X (\mathds{1}_Y\cdot f)(x) \text{ d}x\]
\end{satz}

\begin{beweis}
\textbf{Fall 1:} Die Behauptung ist klar, falls $f$ einfach ist. (Übung!)\\
\textbf{Fall 2:} Sei \((f_n)\) zulässig für $f$ und \(g_n:=f_{n|Y} , h_n:=\mathds{1}_Y f_n\)
Dann ist \((g_n)\) zulässig für \(f_{|Y}\) und \((h_n)\) ist zulässig für \(\mathds{1}_Y f_n\).
Insbesondere sind  \(f_{n|Y}\) und \(\mathds{1}_Y f_n\) nach \ref{Satz 3.5} messbar.
Weiter gilt:
\[ \int_Y f_{|Y}\,dx \overset{n\to\infty}\longleftarrow \int_Yg_n\,dx \overset{Fall 1}=\int_Xh_n\,dx\overset{n\to\infty}\longrightarrow \int_X\mathds{1}_Yf\,dx   \]
\end{beweis}

\begin{definition}
\index{integrierbar}\index{Integral}\index{Lebesgueintegral}
Sei $f:X\to\imdr$ messbar. $f$ heißt (Lebesgue-)\textbf{integrierbar} (über $X$), genau dann wenn $\int_X f_+(x) \text{ d}x<\infty$ \textbf{und} $\int_X f_-(x) \text{ d}x<\infty$.\\
In diesem Fall heißt:
\[\int_X f(x) \text{ d}x:=\int_X f_+(x) \text{ d}x-\int_X f_-(x) \text{ d}x\]
das (Lebesgue-)\textbf{Integral} von $f$ (über $X$).
\end{definition}

\textbf{Beachte:}\\
Ist $f:X\to[0,\infty]$ messbar, so ist $f$ genau dann integrierbar, wenn gilt:
\[\int_X f(x) \text{ d}x<\infty\]

\begin{beispiel}
Sei $X \in \fb_1$, $f(x) := \begin{cases} 1&,x\in X\cap\MdQ\\ 0&,x\in X\setminus\MdQ\end{cases} = \mathds{1}_{X\cap\MdQ}$.
$X, \MdQ \in \fb_1 \implies X \cap \MdQ \in \fb_1 \implies f$ ist messbar.
\[0 \leq \int_X f(x) \text{ d}x = \int_X \mathds{1}_{X\cap\MdQ} \text{ d}x = \lambda(X\cap\MdQ) \leq \lambda(\MdQ) = 0\]
\textbf{Das heißt:} $f \in \fl^1(X)$, $\int_X f \text{ d}x = 0$.
Ist speziell $X = [a,b]\quad (a<b)$, so gilt: $f \in \fl^1([a,b])$, aber $f \not\in R([a,b])$. 
\end{beispiel}

\begin{satz}[Charakterisierung der Integrierbarkeit]
\label{Satz 4.9}
Sei $f: X \to \imdr$ messbar. Die folgenden Aussagen sind äquivalent:
\begin{enumerate}
 \item $f$ ist integrierbar.
 \item Es existieren integrierbare Funktionen $u, v: X \to [0,+\infty]$ mit $u(x)=v(x)=\infty$ für \textbf{kein} $x \in X$ und $f=u-v$ auf $X$.
 \item Es existiert eine integrierbare Funktion $g: X \to [0,+\infty]$ mit $\lvert f \rvert \leq g$ auf $X$.
 \item $\lvert f \rvert$ ist integrierbar.
\end{enumerate}
\end{satz}

\textbf{Zusatz:}
\begin{enumerate}
 \item $\fl^1(X) = \{f: X \to \mdr \mid f$ ist messbar und $\int_X \lvert f \rvert \text{ d}x < \infty\}$ (folgt aus (1)-(4)).
 \item Sind $u,v$ wie in (2), so gilt: $ \int_X f \text{ d}x = \int_X u \text{ d}x - \int_X v \text{ d}x$.
\end{enumerate}


\begin{beweis}[des Satzes]
\begin{enumerate}
 \item[(1) $\Rightarrow$ (2)] $u:= f_+$, $v := f_-$.
 \item[(2) $\Rightarrow$ (3)] $g := u+v$, dann ist $u,v \geq 0$, $g \geq 0$, $\int_X g \text{ d}x \stackrel{4.5}{=} \int_X u \text{ d}x + \int_X v \text{ d}x < \infty$. $\implies g$ ist integrierbar und: $|f| = |u-v| \leq |u| + |v| = u+v = g$ auf $X$.
 \item[(3) $\Rightarrow$ (4)] \ref{Satz 4.5} $\implies \int_X |f| \text{ d}x \leq \int_X g \text{ d}x < \infty \implies f$ ist integrierbar.
 \item[(4) $\Rightarrow$ (1)] $f_+, f_- \leq |f|$ auf $X$. $\implies 0 \leq \int_X f_\pm \text{ d}x \leq \int_X |f| \text{ d}x < \infty \stackrel{Def.}{\implies} f$ ist integrierbar.
\end{enumerate}
\end{beweis}

\begin{beweis}[des Zusatzes]
\begin{enumerate}
 \item \checkmark
 \item Es ist $f = u-v = f_+ - f_- \implies u+f_- = f_+ + v$.
\[\implies \int_X u \text{ d}x + \int_X f_- \text{ d}x \stackrel{4.5}{=} \int_X (u+ f_-) \text{ d}x = \int_X (f_+ + v) \text{ d}x \stackrel{4.5}{=} \int_X f_+ \text{ d}x + \int_X v \text{ d}x\]
\[\implies \int_X u \text{ d}x - \int_X v \text{ d}x = \int_X f_+ \text{ d}x - \int_X f_- \text{ d}x \stackrel{Def.}{=} \int_X f \text{ d}x. \]
\end{enumerate}
\end{beweis}

\begin{folgerungen}
\label{Folgerung 4.10}
\label{Satz 4.10}
Sei $f:X\to\imdr$ integrierbar und $N := \{\lvert f \rvert = +\infty\} = \{x\in X : \lvert f(x) \rvert = + \infty\}$. Dann ist $N\in \fb(X)$ und $\lambda(N) = 0$.
\end{folgerungen}

\begin{beweis}
 $\ref{Satz 3.4} \implies N \in \fb(X).$ $n\mathds{1}_N \leq \lvert f \rvert$ für alle $n\in \MdN$. Dann: 
\[n \cdot \lambda(N) = \int_X n\mathds{1}_N \text{ d}x \stackrel{4.5}{\leq} \int_X \lvert f \rvert \text{ d}x \stackrel{4.9}{<} \infty \text{  für alle } n \in \mdn\]
Also: $0 \leq n\lambda(N) \leq \int_X \lvert f \rvert \text{ d}x \quad \forall n \in \mdn \implies \lambda(N) = 0$ 
\end{beweis}

\begin{satz}
\label{Satz 4.11}
$f, g: X \to \imdr$ seien integrierbar und es sei $\alpha \in \mdr$.
\begin{enumerate}
 \item $\alpha f$ ist integrierbar und $\int_X (\alpha f) \text{ d}x = \alpha \int_X f \text{ d}x$.
 \item Ist $f+g:X\to\imdr$ auf $X$ definiert, so ist $f+g$ integrierbar und es gilt:
 \[\int_X (f+g)\text{ d}x = \int_X f \text{ d}x + \int_X g \text{ d}x\]
(Für $f=+\infty$ und $g=-\infty$ ist $f+g$ beispielsweise nicht definiert.)
 \item $\fl^1(X)$ ist ein reeller Vektorraum und die Abbildung $f \mapsto \int_X f \text{ d}x$ ist linear auf $\fl^1(X)$.
 \item $\max\{f,g\}$ und $\min\{f,g\}$ sind integrierbar.
 \item Ist $f\leq g$ auf $X$, so ist $\int_X f \text{ d}x \leq \int_X g \text{ d}x$.
 \item $\lvert \int_X f \text{ d}x \rvert \leq \int_X \lvert f \rvert \text{ d}x$. (Dreiecksungleichung für Integrale)
 \item Sei $\emptyset\ne Y \in \fb(X)$. Dann sind die Funktionen $f_{|Y}: Y \to \imdr$ und $\mathds{1}_Y\cdot f: X \to \imdr$ integrierbar und
\[\int_Y f(x) \text{ d}x := \int_Y f_{|Y} (x) \text{ d}x = \int_X(\mathds{1}_Y \cdot f)(x) \text{ d}x\]
 \item Sei $\lambda(X) < \infty$ und $h: X \to \mdr$ sei messbar und beschränkt. Dann: $h \in \fl^1(X)$ und $\lvert \int_X h \text{ d}x\rvert \leq \|h\|_\infty \lambda(X) \quad$ (mit $\|h\|_\infty := \sup\{|h(x)| : x\in X\}$) 
\end{enumerate}
\end{satz}

\begin{beweis}
\begin{enumerate} 
\item folgt aus \(\alpha f)_{\pm}=\alpha f_{\pm}\), falls \(\alpha\geq0\) und \(\alpha f)_{\pm}=-\alpha f_{\mp}\), falls 
    \(\alpha<0\).
\item Es gilt \(f+g=\underbrace{f_{+}+g_{+}}_{=:u}-\underbrace{(f_{-}+g_{-})}_{=:v}=u-v\). Dann:
\[
\int_{X}{u\mathrm{d}x}=\int_{X}{f_{+}+g_{+}\mathrm{d}x}\overset{\ref{Satz 4.5}}{=}\int_{X}{f_{+}\mathrm{d}x}+\int_{X}{g_{+}\mathrm{d}x}<\infty
\]
Genauso: \(\int_{X}{v\mathrm{d}x}<\infty\)\\
Mit Satz \ref{Satz 4.9} folgt: \(f+g\) ist integrierbar. Weiter:
\begin{align*}
\int_{X}{(f+g)\mathrm{d}x}&\overset{\ref{Satz 4.9}}{=}\int_{X}{u\mathrm{d}x}-\int_{X}{v\mathrm{d}x}\\
    &=\int_{X}{f_{+}\mathrm{d}x}+\int_{X}{g_{+}\mathrm{d}x}-\left(\int_{X}{f_{-}\mathrm{d}x}+\int_{X}{g_{-}\mathrm{d}x}\right)\\
    &=\int_{X}{f\mathrm{d}x}+\int_{X}{g\mathrm{d}x}
\end{align*}
\item folgt aus (1) und (2).
\item Mit Satz \ref{Satz 3.5} folgt: \(\max\{f,g\}\) ist messbar. Es gilt:
\[
0\leq\lvert\max\{f,g\}\rvert\leq\lvert f\rvert+\lvert g\rvert
\]
Mit \ref{Satz 4.9} und Aussage (2) folgt \(\lvert f\rvert+\lvert g\rvert\) ist integrierbar. Dann folgt mit Satz \ref{Satz 4.9}:
\(\max\{f,g\}\) ist integrierbar.\\
Analog zeigt man: \(\min\{f,g\}\) ist integrierbar.
\item Nach Voraussetzung ist \(f\leq g\) auf \(X\). Dann gilt: \(f_{+}\leq g_{+}\) auf \(X\) und \(f_{-}\geq g_{-}\) auf \(X\).
Es folgt:
\[
\int_{X}{f\mathrm{d}x}=\int_{X}{f_{+}\mathrm{d}x}-\int_{X}{f_{-}\mathrm{d}x}\overset{\ref{Satz 4.5}}{\leq}\int_{X}{g_{+}\mathrm{d}x}-\int_{X}{g_{-}\mathrm{d}x}=\int_{X}{g\mathrm{d}x}
\]
\item Es ist \(\pm f\leq\lvert f\rvert\). Mit Aussage (1) und (5) folgt: 
    \(\pm\int_{X}{f\mathrm{d}x}=\int_{X}{(\pm f)\mathrm{d}x}\leq\int_{X}{\lvert f\rvert\mathrm{d}x}\).\\
Es ist \(\int_{X}{f\mathrm{d}x}=\lvert\int_{X}{f\mathrm{d}x}\rvert\) oder \(-\int_{X}{f\mathrm{d}x}=\lvert\int_{X}{f\mathrm{d}x}\rvert\)
\item Mit Bemerkung (2) vor \ref{Satz 3.1} und Satz \ref{Satz 3.6}.(2) folgt: \(f_{|Y}\) und \(\mathds{1}_{Y}\cdot f\) sind
messbar. Es gilt: \((f_{|Y})_{\pm}=(f_{\pm})_{|Y}\) und \((\mathds{1}_{Y}\cdot f)_{\pm}=\mathds{1}\cdot f_{\pm}\). Weiterhin 
gilt \(0\leq\mathds{1}_{Y}f_{\pm}\leq f_{\pm}\). Mit \ref{Satz 4.9} folgt dann, daß\ \(\mathds{1}_{Y}f_{\pm}\) integrierbar
ist. Dann:
\begin{align*}
\int_{X}{(\mathds{1}_{Y}f)\mathrm{d}x}&=\int_{X}{\mathds{1}f_{+}\mathrm{d}x}-\int_{X}{\mathds{1}_{Y}f\mathrm{d}x}\\
    &=\underbrace{\int_{Y}{(f_{+})_{|Y}\mathrm{d}x}}_{<\infty}-\underbrace{\int_{Y}{(f_{-})_{|Y}\mathrm{d}x}}_{<\infty}
\end{align*}
Es folgt: \(f_{|Y}\) ist integrierbar und \(\int_{Y}{f_{|Y}\mathrm{d}x}=\int_{Y}{(f_{+})_{|Y}\mathrm{d}x}-\int_{Y}{(f_{-})_{|Y}\mathrm{d}x}=\int_{X}{(\mathds{1}_{Y}f)\mathrm{d}x}\).
\item Es ist \(\lvert h\rvert\leq\lVert h\rVert_{\infty}\cdot\mathds{1}_{X}\). Dann folgt:
\[
\int_{X}{\lvert h\rvert\mathrm{d}x}\leq\int_{X}{\lVert h\rVert_{\infty}\mathds{1}_{X}\mathrm{d}x}=\lVert h\rVert_{\infty}\lambda(X)<\infty
\]
Damit: \(\lvert h\rvert\) ist integrierbar und mit \ref{Satz 4.9} auch \(h\). Da \(h\) beschränkt ist, folgt: 
\(h\in\fl^{1}(X)\). Schließlich:
\[
\left\lvert\int_{X}{h\mathrm{d}x}\right\rvert\leq\int_{X}{\lvert h\rvert\mathrm{d}x}\leq\lVert h\lVert_{\infty}\lambda(X)
\]
\end{enumerate}
\end{beweis}

\begin{satz}
\label{Satz 4.12}
\begin{enumerate}
 \item Sind $\emptyset\ne A,B \in \fb(X)$ disjunkt, $X = A \cup B$ und ist $f: X \to \imdr$ integrierbar (über $X$), so ist $f$ integrierbar über $A$ und integrierbar über $B$ und es gilt:
 \[\int_X f \text{ d}x = \int_A f \text{ d}x + \int_B f \text{ d}x\]
 \item Ist $\emptyset \neq K \subseteq \mdr^d $ kompakt und $f:K\to\mdr$ stetig, so ist $f \in \fl^1(K)$.
\end{enumerate}

\end{satz}

\begin{beweis}
\begin{enumerate}
 \item Aus \ref{Satz 4.11}(7) folgt: $f$ ist integrierbar über $A$ und integrierbar über $B$. Es ist 
\[ \int_X f(x) \text{ d}x = \int_X \left( \mathds{1}_{A\cup B} \cdot f \right)(x) \text{ d}x = \int_X \left( \left( \mathds{1}_A + \mathds{1}_B \right) f\right)(x) \text{ d}x \]
\[= \int_X \left(\mathds{1}_A f + \mathds{1}_B f \right)(x) \text{ d}x \stackrel{4.11(2)}{=} \int_X \mathds{1}_A f \text{ d}x + \int_X \mathds{1}_B f \text{ d}x \stackrel{4.11(7)}{=} \int_A f \text{ d}x + \int_B f \text{ d}x.\]

 \item $K$ ist kompakt, also gilt: $\lambda(K) < \infty$. Aus \ref{Satz 3.2}(1) folgt, dass $f$ messbar ist. Analysis II (\glqq stetige Funktionen auf kompakten Mengen nehmen Minimum und Maximum an\grqq ) liefert: $f$ ist beschränkt. Insgesamt folgt mit \ref{Satz 4.11}(8) schließlich: $f \in \fl^1(K)$.
\end{enumerate}
\end{beweis}

\begin{satz}
\label{Satz 4.13}
Seien $a,b\in\mdr$, $a<b$, $X:=[a,b]$ und $f\in C(X)$. Dann ist $f\in\fl^1(X)$ und es gilt:
\[L-\int_X f(x) \text{ d}x=R-\int_a^b f(x) \text{ d}x\]
\end{satz}

\begin{beweis}
Sei $\natn$, $t_j^{(n)}:=a+j\frac{b-a}{n}$ ($j=0,\dots,n$) und $I_j^{(n)}:=\left[t_{j-1}^{(n)},t_j^{(n)}\right]$ ($j=1,\dots,n$).
\begin{align*}
S_n:=\sum^n_{j=1} f \left(t_j^{(n)}\right) \underbrace{ \frac{b-a}{n}}_{= \lambda_1 \left(I_j^{(n)}\right)} \text{ ist Riemannsche Zwischensumme für R-} \int_a^bf(x)\,dx.
\end{align*}
Aus Analysis I folgt $S_n\to\text{R-}\int_a^bf(x)\,dx$ ($n\to\infty$). 
Definiere $f_n:=\sum^n_{j=1}f \left(t_j^{(n)} \right) \mathds{1}_{I_j^{(n)}} $. Dann ist $f_n$ einfach und 
\[\int_X f_n(x)\,dx=\sum_{j=1}^n f \left(t_j^{(n)} \right) \lambda_1 \left(I_j^{(n)}\right)=S_n\]
$f$ ist auf $X$ gleichmäßig stetig also konvergiert $f_n$ auf $X$ gleichmäßig gegen $f$ (Übung!), also gilt:
\[\lVert f_n-f \rVert_{\infty}=\text{sup} \left \{ \lvert f_n(x)-f(x) \rvert : x\in X \right\} \to 0 \  (n\to \infty)\]
Aus \ref{Satz 4.12}(2) folgt $f\in \mathfrak{L}^1(X)$
\begin{align*}
\left\lvert \text{L-} \int \limits_X f(x)\,dx -S_n \right\rvert = \left\lvert \text{L-} \int \limits_X (f-f_n)\,dx \right\rvert \stackrel{\text{4.11}}\leq \int \limits_X(f-f_n)\,dx \stackrel{\text{4.11}}\leq \lVert f-f_n \rVert_{\infty} \underbrace{\lambda(X)}_{=b-a} \to 0
\end{align*}
Daraus folgt $S_n \to$ L- $\int_X f\,dx$
\end{beweis}

\begin{satz}
\label{Satz 4.14}
Sei $a\in\mdr, X:=[a,\infty)$ und $f\in C(X)$. Dann gilt:
\begin{enumerate}
\item $f$ ist messbar.
\item $f\in\fl^1(X)$ genau dann wenn das uneigentliche Riemann-Integral $\int_a^\infty f(x) \text{ d}x$ \textbf{absolut} konvergent ist. In diesem Fall gilt:
\[L-\int_X f(x) \text{ d}x=R-\int_a^\infty f(x) \text{ d}x\]
Entsprechendes gilt für die anderen Typen uneigentlicher Riemann-Integrale.
\end{enumerate}
\end{satz}

\begin{beweis}
Eine Hälfte des Beweises folgt in Kapitel \ref{Kapitel 6}.
\end{beweis}

\begin{beispiel}
\begin{enumerate}
\item Sei $X=(0,1]$, $f(x)=\frac{1}{\sqrt{x}}$. Aus Analysis I wissen wir, dass R-$\int^1_0\frac{1}{\sqrt{x}}\,dx$ (absolut) konvergent ist. Also ist $f\in\mathfrak{L}^1(X)$.\\
Außerdem wissen wir aus Analysis I, dass R-$\int_0^1\frac{1}{x}$ divergent ist. Also ist $f^2\notin\mathfrak{L}^1(X)$.
\item Sei $X=[0,\infty)$, $f(x)=\frac{\sin(x)}{x}$. Aus Analysis I wissen wir, dass R-$\int^{\infty}_1f(x)\,dx$ konvergent, aber nicht absolut konvergent ist. Also ist $f\notin\mathfrak{L}^1(X)$.
\end{enumerate}
\end{beispiel}

\section{Lineare Kongruenzen}
\subsection{Allgemeine Informationen}
Zwei Zahlen $a, b \in \mathbb{Z}$ heißen kongruent modulo $m \in \mathbb{N}$, 
falls $a$ und $b$ bei der Division durch $m$ den den gleichen Rest lassen. 
Man schreibt $a \equiv b \imod{m}$\footnote{[Reiss], S. 179f}.

Gilt $ax \equiv b \imod{m}$, für $a, b, x \in \mathbb{Z}$ und $m \in \mathbb{N}$,
dann bedeutet das, dass $m | (ax - b)$ für ein passendes $x$. 
Man nennt $ax \equiv b \imod{m}$ ein lineares Kongruenzsystem. 
\clearpage 

\subsection{Chinesischer Restsatz}
Der Chinesische Restsatz sagt, ob lineare Kongruenzsysteme lösbar 
sind und wie diese Lösungen aussehen:

\begin{mdframed}[tikzsetting={draw=red,ultra thick}, innertopmargin=0.6cm]
Seien $m_1, m_2, ..., m_n$ paarweise teilerfremde natürliche Zahlen und
$a_1, a_2, \dots, a_n$ ganze Zahlen.

Dann ist das System linearer Kongruenzen
\vspace{-0.4cm}
\[x \equiv a_1 \imod{m_1},\;\;\; x \equiv a_2 \imod{m_2},\;\;\;\dots,\;\;\; x \equiv a_n \imod{m_n}\]
lösbar. Alle Lösungen des Systems liegen in einer gemeinsamen
 Restklasse modulo $M=\prod_{i = 1}^n m_i$
\end{mdframed}

\textbf{Beweis nach [Reiss], S. 221f:}
\begin{enumerate}[label=(\Roman{*}),labelsep=0.5em,noitemsep]
    \item $M_j = \frac{M}{m_j}$ für $j = 1, \dots, n$
    \item $y_j \cdot M_j \equiv 1 \imod{m_j}$, $y_j$ mit dem erweitertem Euklidischem Algorithmus bestimmen
    \item $a_j \cdot y_j \cdot M_j \equiv a_j \imod{m_j}$ für $j = 1, \dots, n$\\
Weil $m_j$ für $i \neq j$ ein Teiler von  $M_i$ ist, gilt auch:
    \item $a_i \cdot y_i \cdot M_i \equiv 0 \imod{m_j}$ für alle $i, j = 1, \dots, n$ mit $i \neq j$
\end{enumerate}

Da alle Summanden bis auf Einen ($j = i$) gleich Null sind, stimmt dieser Ausdruck:
\begin{align*}
a_i \cdot y_i \cdot M_i &\equiv \sum_{j=1}^n {a_j \cdot y_j \cdot M_j} \imod{m_i}\\
a_i &\equiv \sum_{j=1}^n {a_j \cdot y_j \cdot M_j} \imod{m_i}\text{, da }y_i \cdot M_i \equiv 1 \imod{m_i}
\end{align*}

$a_i$ ist die Lösung des Kongruenzsystems. Alle Lösungen liegen in dieser Restklasse.


\subsubsection*{Beispielaufgabe}
Folgende Aufgabe wurde [Berendt] entnommen:

\hangindent2em
\hangafter=0
17 chinesische Piraten erbeuten eine Truhe mit Goldstücken. Beim Versuch, diese gleichmäßig zu verteilen, bleiben 7 Goldstücke übrig. Um diese entbrennt ein heftiger Streit, bei dem einer der Piraten das Leben lässt. Die verbleibenden 16 versuchen erneut, die Goldstücke gerecht zu verteilen, behalten jedoch elf Stücke übrig. Bei der folgenden Auseinandersetzung geht wieder einer der Streitenden über Bord. Den 15 Überlebenden gelingt dann die Teilung. Wie viele Goldstücke müssen es mindestens gewesen sein?

\subsubsection*{Restklassensystem} % This should semantically rather be subsubsubsection
\begin{align*}
x &:= \text{Anzahl der Goldstücke}\\
x &\equiv 7 \imod{17}\\
x &\equiv 11 \imod{16}\\
x &\equiv 0 \imod{15}
\end{align*}

\subsubsection*{Lösung}
I Produkte
\begin{align*}
M   &= 17 \cdot 16 \cdot 15 = 4080\\
M_1 &= \frac{4080}{17} = 240\\
M_2 &= \frac{4080}{16} = 255\\
M_3 &= \frac{4080}{15} = 272
\end{align*}

II Multiplikativ Inverses der Restklassensysteme
\begin{align*}
 9 \cdot 240 &\equiv 1 \imod{17}\\
15 \cdot 255 &\equiv 1 \imod{16}\\
8 \cdot 272 &\equiv 1 \imod{15}
\end{align*}

III  Multiplikation der Restklassensysteme mit $a_j$
\begin{align*}
7 \cdot 9 \cdot 240     &\equiv 7   \imod{17}\\
11 \cdot 15 \cdot 255   &\equiv 11  \imod{16}\\
8 \cdot 272             &\equiv 0   \imod{15}
\end{align*}

IV Berechnung der Lösung des Restklassensystem
\begin{align*}
x = \sum_{j = 1}^3 a_j \cdot y_j \cdot M_j \imod{15 \cdot 16 \cdot 17} = 7 \cdot 240 \cdot 9 + 11 \cdot 255 \cdot 15 = 57195\\
57195 \equiv 75 \imod{4080}\\
75 \text{ ist die kleinste positive Lösung des Kongruenzsystems.}
\end{align*}

\subsubsection*{Antwort:}
Die Anzahl der von den Piraten erbeuteten Goldstücken muss mindestens $75$ betragen, kann aber auch $75 + 1 \cdot 4080$, $75 + 2 \cdot 4080$  oder ein beliebiger anderer positiver Vertreter dieser Restklasse$\imod{4080}$ sein.

\clearpage
\section{Multiplikativ inverses Element}\label{sec:Multiplikativ-Inverses}
\subsection{Definition und Beispiele}
Das multiplikativ inverse Element $d$ von $e$ ergibt bei der 
Multiplikation mit $e$ das neutrale Element der Multiplikation, also 
die Eins: $d \cdot e = 1$

In $\mathbb{R} \setminus \Set{0}$ hat jedes Element ein multiplikativ 
Inverses, den Kehrbruch. In $\mathbb{Z}/7 \mathbb{Z}$ ist das 
multiplikativ Inverse von zwei in der Restgruppe von vier, da 
$2 \cdot 4 = 8$ und $8 \equiv 1 \imod{7}$.
Mit dem erweitertem euklidischem Algorithmus kann man das 
multiplikativ Inverse von $a$ in $\mathbb{Z}/n \mathbb{Z}$ finden. 

\subsection{Erweiterter euklidischer Algorithmus}
Sind zwei Zahlen $a > b$ gegeben und will deren größten gemeinsamen 
Teiler berechnen, so kann man den erweiterten euklidischen 
Algorithmus anwenden:

\begin{enumerate}
    \item Größtmögliches $q$ wählen, so dass gilt $a = q_1 \cdot b + r_1$
    \item $b = q_2 \cdot r_1 + r_2$
    \item $r_1 = q_3 \cdot r_2 + r_3$
    \item \dots
    \item bis $r_{n-2} = q_n \cdot r_{n-1} + r_n$ mit $r_n = 0$
\end{enumerate}

Dann ist $r_{n-1} = ggT(a,b)$

Mit diesem Algorithmus kann man nun das multiplikativ Inverse von $a$ 
in $\mathbb{Z}/n \mathbb{Z}$ finden, wenn der größte gemeinsame Teiler von $a$ und 
$n$ gleich 1 ist. Da im vorletzten Schritt $r_{n - 1} = 1$ ist, kann man 1 als 
Linearkombination der Reste von $r_{n - 3}$ und $r_{n - 2}$ 
darstellen. Diese Reste kann man wiederum als Linearkombination 
vorhergehender Reste darstellen. Dies setzt man so lange fort, 
bis man eine Linearkombination mit $a$ und $n$ von 1 hat. Da wir im 
Restklassenring $n$ sind, muss man nur das Produkt mit $a$ betrachten 
und kann das multiplikativ Inverse zu $a$ im Restklassenring 
$\mathbb{Z}/n \mathbb{Z}$ ablesen. 


Hier ein Beispiel zur Veranschaulichung:

Sei $a = (\text{Primzahl}_1 - 1) \cdot (\text{Primzahl}_2 - 1) =(3 - 1) \cdot (47 - 1) = 92$ und $b=71$

Gesucht ist das multiplikativ Inverse $b \in \mathbb{Z} / a \mathbb{Z}$ von $x \cdot 71 \equiv 1 \imod{92}$: 

\begin{tabular}{lll}
\textbf{Schritt 1}: euklidischer Algorithmus & & \textbf{Schritt 2}: nach Rest auflösen\\
$91=1 \cdot 71 + 21$ \myDownArrow & $\rightarrow$     & $21 = 92 - 71$ \myUpArrow\\
$71=3 \cdot 21 + 8$     & $\rightarrow$     & $8 = 71 - 3 \cdot 21$\\
$21=2 \cdot 8 + 5$      & $\rightarrow$     & $5 = 21 - 2 \cdot 8$\\
$ 8=1 \cdot 5 + 3$      & $\rightarrow$     & $3 =  8 - 1 \cdot 5$\\
$ 5=1 \cdot 3 + 2$      & $\rightarrow$     & $2 =  5 - 1 \cdot 3$\\
$ 3=1 \cdot 2 + 1$      & $\rightarrow$     & $1 =  3 - 1 \cdot 2$
\end{tabular}

\textbf{Schritt 3}: so lange Reste einsetzen, bis eine Linearkombination der Form
$1 = x \cdot 92 + y \cdot 71$ gefunden ist:

\begin{align*}
1 &= 3 - (5 - 3)                             &&= 2 \cdot 3 - 5 \\
1 &= 2 \cdot (8 - 5) - (21 - 2 \cdot 8)        &&= 4 \cdot 8 - 2 \cdot 5 - 21 \\
1 &= 4 \cdot 8 - 2 \cdot (21 - 2 \cdot 8) - 21  &&= 8 \cdot 8 - 3 \cdot 21 \\
1 &= 8 \cdot (71 - 3 \cdot 21) - 3 \cdot (92 - 71) &&= 11 \cdot 71 - 24 \cdot 21 - 3 \cdot 92 \\
1 &= 11 \cdot 71 - 3 \cdot 92 - 24 \cdot (92 - 71) &&= 35 \cdot 71 - 27 \cdot 92
\end{align*}

Das bedeutet 35 ist das multiplikativ Inverse zu 71 in 
$ \mathbb{Z} / 92 \mathbb{Z}$ und erfüllt damit die Kongruenzgleichung
$35 \cdot 71 \equiv 1 \imod{92}$.

Zusätzlich hat man damit weitere multiplikativ Inverse gefunden:
\begin{itemize}
    \item $-27 \cdot 92 \equiv 1 \imod{71}$
    \item $-27 \cdot 92 \equiv 1 \imod{35}$
    \item $35 \cdot 71 \equiv 1 \imod{27}$
\end{itemize}

\clearpage
In diesem Kapitel sei stets \(\emptyset\neq X\in \fb_d\).

\begin{satz}
\label{Satz 7.1}
Sei \(U\in\fb_k, t_0\in U\) und es sei \(f\colon U\times X\to \mdr\) eine Funktion mit:
\begin{enumerate}
	\item 	Für jedes \(t\in U\) ist \(x\mapsto f(t,x)\) messbar.
	\item 	Es existiert eine Nullmenge \(N\subseteq X\) so, dass \(t\mapsto f(t,x)\) für jedes \(x\in X\setminus N\) stetig in $t_0$ ist.
	\item 	Es existiert eine integrierbare Funktion \(g\colon X\to [0,\infty]\) und zu jedem \(t\in U\) existiert eine Nullmenge \(N_t\subseteq X\) so, dass für
		jedes \(t\in U\) und jedes \(x\in X\setminus N_t\) gilt: \[ \lvert f(t,x)\rvert \leq g(x) \]
\end{enumerate}
Dann ist \(x\mapsto f(t,x)\) für jedes \(t\in U\) integrierbar. Ist \(F\colon U\to\mdr\) definiert durch
\[ F(t):=\int_Xf(t,x)\,dx,\]
so ist $F$ stetig in $t_0$.
\end{satz}

Also: \[ \lim\limits_{t\to t_0}\int_X f(t,x)\,dx = \lim\limits_{t\to t_0}F(t)=F(t_0) = \int_X f(t_0,x)\,dx =\int_X\lim\limits_{t\to t_0} f(t,x)\,dx \]

\begin{beweis}
Aus (1) und (3) folgt, dass \(x\mapsto f(t,x)\) für jedes \(t\in U\) integrierbar ist (zur Übung). Sei \((t_n)\) eine Folge in $U$ mit \(t_n\to t_0\) und
\[g_n(x):=f(t_n,x) \ (\natn, x\in X) \]
Setze \[ \tilde N := N\cup \left(\bigcup^\infty_{n=1}N_{t_n} \right) \]
Aus \ref{Lemma 5.1} folgt, dass \(\tilde N\) eine Nullmenge ist. Voraussetzung (2) liefert \(g_n(x)\to f(t_0,x)\) für jedes \(x\in X\setminus\tilde N\), also gilt
\[g_n(x)\to f(t_0,x) \text{ fast überall auf } X\]
Voraussetzung (3) liefert \(\lvert g_n(x)\rvert = \lvert f(t_n,x)\rvert \leq g(x) \) für jedes \(\natn\) und \(x\in X\setminus\tilde N\). Aus \ref{Satz 6.2} folgt
\[ F(t_n) = \int_X f(t_n,x)\,dx = \int_Xg_n\,dx \longrightarrow \int_X f(t_0,x)\,dx = F(t_0) \]
\end{beweis}

\textbf{Bezeichnung}\\
Sei \(I\subseteq\mdr\) ein Intervall, \(a:=\inf I\) und \(b:=\sup I\), wobei \(a=-\infty\) oder \(b=+\infty\) zugelassen sind. Weiter sei \(f\colon I\to\imdr\) integrierbar
(oder $f$ ist messbar und \(\geq 0\)) und
\[\int\limits^b_af(x)\,dx:=\int\limits_{(a,b)}f_{|(a,b)}(x)\,dx \]
Dann ist
\[ \int_I f(x) dx = \int_{(a,b)} f(x) dx\]
Ist z.B. \(I=[a,b)\), dann gilt, da \(\{a\}\) eine Nullmenge ist: \[\int_If\,dx=\int_{\{a\}}f\,dx + \int_{(a,b)}f\,dx= \int_{(a,b)}f\,dx \]

\begin{folgerung}
\label{Folgerung 7.2}
Sei \(I\subseteq\mdr\) ein Intervall, \(a=\inf I\) und \(f\colon I\to\mdr\) sei integrierbar. Definiert man \(F\colon I\to\mdr\) durch
\[F(t):=\int^t_a f(x)\,dx,\] so ist \(F\in C(I)\).
\end{folgerung}

\begin{beweis}
Für \(x,t\in I\) definiere \(h(t,x):=\mathds{1}_{(a,t)}f(x)\). Dann ist \(F(t)=\int_I h(t,x)\,dx\) und
\[\lvert h(t,x)\rvert = \mathds{1}_{(a,t)}\cdot \lvert f(x)\rvert \leq \lvert f(x)\rvert \text{ für alle } t,x\in I\]
Aus \ref{Satz 4.9} folgt, dass \(\lvert f\rvert\) integrierbar ist. Sei \(t_0\in I\) und \(N:=\{t_0\}\), also eine Nullmenge.
Dann ist \(t\mapsto h(t,x)\) für jedes \(x\in I\setminus N\) stetig in \(t_0\) (zur Übung). Die Behauptung folgt aus \ref{Satz 7.1}.
\end{beweis}

\begin{satz}
\label{Satz 7.3}
Sei \(U\subseteq \mdr^k\) offen, \(f\colon U\times X\to\mdr\) eine Funktion. Es sei \(g\colon X\to [0,+\infty]\) integrierbar und \(N\subseteq X\) sei eine Nullmenge.
Weiter gelte:
\begin{enumerate}
	\item 	für jedes \(t\in U\) sei \(x\mapsto f(t,x)\) integrierbar.
	\item 	für jedes \(x\in X\setminus N\) sei \(t\mapsto f(t,x)\) partiell differenzierbar auf $U$.
	\item 	\(\left\lvert \frac{ \partial f}{\partial t_j} \right\rvert \leq g(x) \) für jedes \(x\in X\setminus N\), jedes \(t\in U\) und jedes \(j\in\{1,\dots,k\}\)
\end{enumerate}
Ist dann \(F\colon U\to\mdr\) definiert durch \[F(t):=\int_X f(t,x)dx\] so ist $F$ auf $U$ partiell differenzierbar und für jedes \( t\in U\) sowie jedes \( j\in\{1,\dots,k\}\) gilt:
\[ \frac{\partial F}{\partial t_j}(t) = \int_X\frac{\partial f}{\partial t_j}(t,x)\,dx \]
\end{satz}
\textbf{Also: } \( \frac{\partial}{\partial t_j}\int_X f(t,x)\,dx = \int_X \frac{\partial f}{\partial t_j}(t,x)\,dx \).

\begin{beweis}
Sei o.B.d.A. \(k=1\), also \(U\subseteq\mdr\). Sei \(t_0\in U\) und \((h_n)\) eine Folge mit \(h_n\to 0\) und \(h_n\neq 0\) für alle \(\natn\).
Setze \[g_n(x):=\frac{f(t_0+h_n,x)-f(t_0,x)}{h_n} \ \ (x\in X, \natn) \]
Aus Voraussetzung (2) folgt für jedes \(x\in X\setminus N\): \[ g_n(x)\to \frac{\partial f}{\partial t}(t_0,x) \ \ (n\to\infty) \]
Nach dem Mittelwertsatz aus Analysis 1 existiert für jedes \(x\in X\setminus N\) und jedes \(\natn\) ein \(s_n=s_n(x)\) zwischen \(t_0\) und \(t_0+h_n\) mit:
\[ \left\lvert g_n(x) \right\rvert = \left\lvert \frac{\partial f}{\partial t}(s_n,x)\right\rvert \overset{(3)}\leq g(x) \]
Aus \ref{Satz 6.2} folgt \[ \int_X g_n\,dx \longrightarrow \int_X\frac{\partial f}{\partial t}(t_0,x)\,dx \]
Es ist nach Konstruktion  gerade \(\int_X g_n\,dx =\frac{F(t_0+h_n)-F(t_0)}{h_n} \).
\end{beweis}

\clearpage

\section{Literaturverzeichnis}

\textbf{Beutelspacher, A., Neumann, H. B. und Schwarzpaul, T.:}\\
Kryptografie in Theorie und Praxis.\\
Wiesbaden, Vieweg+Teubner Verlag, 2005.

\textbf{Brill, M.:}\\
Mathematik für Informatiker.\\
Wien, Hanser Verlag, 2004.

\textbf{Forster, O.:}\\
Algorithmische Zahlentheorie.\\
Wiesbaden, Vieweg Braunschweig/Wiesbaden, 1996.

\textbf{Pethö, A. und Pohst, M.:}\\
Algebraische Algorithmen.\\
Wiesbaden, Vieweg+Teubner Verlag, 1999.

\textbf{Reiss, K. und Schmieder, G.:}\\
Basiswissen Zahlentheorie.\\
Berlin Heidelberg, Springer-Verlag, 2007.

\textbf{Rothe, J.:}\\
Komplexitätstheorie und Kryptologie.\\
Berlin, Springer, 2008.

\textbf{Wrixon, F. B.:}\\
Geheimsprachen.\\
Königswinter, Tandem Verlag GmbH, 2006.

\newpage
\subsection*{Internetadressen}
{\footnotesize
\begin{tabular}{lp{12cm}}
{[Berendt]} & 
\parbox{12cm}{Berendt, Gerhard.\\
Seminar über Zahlentheorie/Kryptographie.\hfill10.04.2008\\
URL:  http://userpage.fu-berlin.de/$\sim$berendt/lehre2008\_neu.html\\
{[Stand: 09.01.2010]}} \\\\
{[Birthälmer]} &
\parbox{12cm}{Birthälmer, Melita.\\
Kryptografie\hfill13.04.2008\\
URL: http://www.birthaelmer.com/fileadmin/birthaelmer/portfolio/Kryptografie\_web.pdf\\
{[Stand: 09.01.2010]}} \\\\
{[msri]} &
\parbox{12cm}{SIAM News.\\
Still Guarding Secrets after Years of Attacks, RSA Earns Accolades for its Founders.\hfill17.06.2003\\
URL: http://www.msri.org/people/members/sara/articles/rsa.pdf\\
{[Stand: 09.01.2010]}} \\\\
{[Paixão]} &
\parbox{12cm}{Paixão, Cesar A. M.. \\
An efficient variant of the RSA cryptosystem.\hfill11.08.2009\\
URL: http://eprint.iacr.org/2003/159.pdf\\
{[Stand: 26.11.2009]}} \\\\
{[Petitcolas]} &
\parbox{12cm}{Petitcolas, Fabien. \\
DESIDERATA DE LA CRYPTOGRAPHIE MILITAIRE.\hfill20.06.2009\\
URL: http://www.petitcolas.net/fabien/kerckhoffs/la\_cryptographie\_militaire\_i.htm\#desiderata \\
{[Stand: 09.01.2010]}} \\\\
{[RSA-2190]} &
\parbox{12cm}{RSA Laboratories.\\
What are the best factoring methods in use today?\hfill03.11.2009\\
URL: http://www.rsa.com/rsalabs/node.asp?id=2190\\
{[Stand: 09.01.2010]}} \\\\
{[RSA-2191]} &
\parbox{12cm}{RSA Laboratories.\\
What improvements are likely in factoring capability?\hfill03.11.2009\\
URL: http://www.rsa.com/rsalabs/node.asp?id=2191\\
{[Stand: 09.01.2010]}} \\\\
{[RSA-2216]} &
\parbox{12cm}{RSA Laboratories.\\
What would it take to break the RSA cryptosystem?\hfill03.11.2009\\
URL: http://www.rsa.com/rsalabs/node.asp?id=2216\\
{[Stand: 09.01.2010]}} \\\\
{[RSA-2964]} &
\parbox{12cm}{RSA Laboratories.\\
RSA-640 factored!\hfill03.01.2010\\
URL: http://www.rsa.com/rsalabs/node.asp?id=2964\\
{[Stand: 09.01.2010]}} \\\\
{[wiki-RSA]} &
\parbox{12cm}{Ladenthin, Bernhard.\\
RSA-Kryptosystem.\hfill08.01.2010\\
URL: http://de.wikipedia.org/wiki/RSA-Kryptosystem\#Erzeugung\_des\_öffentlichen\_und\_privaten\_Schlüssels\\
{[Stand: 09.01.2010]}}
\end{tabular}
}

\clearpage

\section{Anhang}

\inputminted[linenos, numbersep=5pt, tabsize=4]{pascal}{Factoring.ari}

\clearpage

\newpage
\vspace{10cm}

Ich erkläre, dass ich die Facharbeit ohne fremde Hilfe angefertigt 
und nur die im Literaturverzeichnis angeführten Quellen und 
Hilfsmittel benützt habe.

\vspace{10cm}

   Langweid, 20.01.2010  				                         
	Ort, Datum						Unterschrift


\end{document}
