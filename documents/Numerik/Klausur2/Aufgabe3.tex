\section*{Aufgabe 3}
\textbf{Gegeben:}

\begin{table}[h!]
    \begin{tabular}{l||l|l|l|l}
    $f_i$ & 7  & 1 & -1 & 7 \\\hline
    $x_i$ & -1 & 0 & 1  & 2 \\
    \end{tabular}
\end{table}

\subsection*{Teilaufgabe a)}
Allgemein lauten Lagrange-Polynome:

\[L_i = \frac{\overbrace{\prod_{j=0, j \neq i}^n (x-x_j)}^\text{Produkt der Nullstellen}}{\underbrace{\prod_{j=0, j \neq i}^n (x_i - x_j)}_\text{Normalisierungsfaktor}}\]

Im speziellen:
\begin{align}
	L_0(x) &= \frac{(x-0)(x-1)(x-2)}{(-1-0)(-1-1)(-1-2)} &&=-\frac{1}{6} \cdot (x^3 - 3 x^2 + 2x)\\
	L_1(x) &= \frac{(x+1)(x-1)(x-2)}{(0+1)(0-1)(0-2)}    &&= \frac{1}{2} \cdot (x^3 - 2x^2 - x + 2)\\
	L_2(x) &= \frac{(x+1)x(x-2)}{(1+1)(1-0)(1-2)}        &&=-\frac{1}{2} \cdot (x^3 - x^2 - 2x)\\
	L_3(x) &= \frac{(x+1)(x-0)(x-1)}{(2+1)(2-0)(2-1)}    &&= \frac{1}{6} \cdot (x^3 - x)
\end{align}

Durch die Interpolationsformel von Lagrange

\[p(x) = \sum_{i=0}^n f_i L_i(x)\]

ergibt sich
\begin{align}
	p(x) &= x^3 + 2x^2 - 5x + 1
\end{align}
Anmerkung: Es ist nicht notwendig die Monomdarstellung zu berechnen.
In diesem Fall hat es jedoch das Endergebnis stark vereinfacht.

\subsection*{Teilaufgabe b)}
Zunächst die dividierten Differenzen berechnen:
\begin{align}
	f[x_0] &= 7,           &f[x_1] &= 1,       & f[x_2] &= -1,     & f[x_3] = 7\\
	f[x_0, x_1] &= -6,     &f[x_1, x_2] &= -2, &f[x_2, x_3] &= 8\\
	f[x_0, x_1, x_2] &= 2, &f[x_1, x_2, x_3] &= 5\\
	f[x_0, x_1, x_2, x_3] &= 1
\end{align}

Insgesamt ergibt sich also
\begin{align}
	p(x) &= 7 - (x+1) \cdot 6 + (x+1) \cdot x \cdot 2 + (x+1) \cdot x \cdot (x-1)
\end{align}

