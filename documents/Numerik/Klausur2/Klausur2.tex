\documentclass[a4paper]{article}

\usepackage[english]{babel}
\usepackage[utf8x]{inputenc}
\usepackage{amsmath}
\usepackage{graphicx}
\usepackage[colorinlistoftodos]{todonotes}
\usepackage{stmaryrd}
\usepackage{parskip} % damit keine "unsinnigen" Einrückungen passieren

\title{Musterlösungen für Numerik}
\author{Felix Benz-Baldas}

\begin{document}
\maketitle

\section{Klausur 2}
\subsection{Aufgabe 1}
\subsubsection*{(a)}

$
L =
\begin{pmatrix}
2 & 0 & 0 \\
1 & 2 & 0 \\
4 & 2 & 3 \\
\end{pmatrix}
$


\subsubsection*{(b)}
gesucht: det(A)

sei P * L = L * R, die gewohnte LR-Zerlegung

dann gilt:

$det(A) = det(L) * det(R) / det(P)$

det(L) = 1, da alle Diagonalelemente 1 sind und es sich um eine untere Dreiecksmatrix handelt.

$ det(R) = r_{11} * ... * r_{nn} $ da es sich um eine obere Dreiecksmatrix handelt.


$ det(P) = $ 1 oder -1

Das Verfahren ist also:
\begin{enumerate}
\item Berechne Restmatrix R mit dem Gaußverfahren.
\item \label{manker} Multipliziere die Diagonalelemente von R
\item falls die Anzahl an Zeilenvertauschungen ungerade ist negiere das Produkt aus \ref{manker} (eine Zeilenvertauschung verändert lediglich das Vorzeichen und P ist durch Zeilenvertauschungen aus der Einheitsmatrix hervorgegangen)
\end{enumerate}

\subsection{Aufgabe 2}
\subsubsection*{(a)}
Formel: $y_i = (b_i - \sum_{j=1}^{i-1} y_j \cdot l_{ij} ) \div l_{ii} $

Anmerkung: $l_{ii}$ kann nicht $0$ sein, da L dann nicht mehr invertierbar wäre.

Algorithmus:
\begin{itemize}
\item for i = 1 to i = n do
\begin{itemize}
\item sum = 0
\item for j = 1 to j = i - 1 do
\begin{itemize}
\item sum = sum + $y_i \cdot l_{ij}$
\end{itemize}
\item od
\item $y_i = (b_i - sum) \div l_{ii}$
\end{itemize}
\item od
\end{itemize}

\subsubsection*{(b)}
\begin{itemize}
\item function $ x = LoeseLGS(A,b)$
\begin{itemize}
\item $(P,L,R) = LRZer(A)$
\item $b'=P \cdot b $
\item $c = VorSub(L,b') $
\item $x=RueckSub(R,c)$
\end{itemize}
\item end

\end{itemize}


\subsubsection*{(c)}
Aufwand:
\begin{itemize}
\item Vorwärts-/Rückwärtssubstitution: jeweils $\frac{1}{2} \cdot n^2$
\item LR-Zerlegung: $\frac{1}{3}n^3$ (den Beweis dazu braucht man nicht wissen)
\item gesamt: $\frac{1}{3}n^3+n^2$
\end{itemize}

\end{document}
