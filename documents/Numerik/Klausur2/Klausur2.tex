\documentclass[a4paper]{scrartcl}
\usepackage{amssymb, amsmath} % needed for math
\usepackage[utf8]{inputenc} % this is needed for umlauts
\usepackage[ngerman]{babel} % this is needed for umlauts
\usepackage[T1]{fontenc}    % this is needed for correct output of umlauts in pdf
\usepackage{pdfpages}       % Signatureinbingung und includepdf
\usepackage{geometry}       % [margin=2.5cm]layout
\usepackage[pdftex]{hyperref}       % links im text
\usepackage{color}
\usepackage{framed}
\usepackage{enumerate}      % for advanced numbering of lists
\usepackage{marvosym}       % checkedbox
\usepackage{wasysym}
\usepackage{braket}         % for \Set{}
\usepackage{pifont}% http://ctan.org/pkg/pifont
\usepackage{gauss}
\usepackage{algorithm,algpseudocode}
\usepackage{parskip}
\usepackage{lastpage}
\usepackage{amsthm}
\allowdisplaybreaks

\newcommand{\cmark}{\ding{51}}%
\newcommand{\xmark}{\ding{55}}%

\title{Numerik Klausur 2{} - Musterlösung}
\makeatletter
\AtBeginDocument{
	\hypersetup{ 
	  pdfauthor   = {Felix Benz-Baldas, Martin Thoma, Peter Merkert},
	  pdfkeywords = {Numerik, KIT, Klausur}, 
	  pdftitle    = {\@title} 
  	}
	\pagestyle{fancy}
	\lhead{\@title}
	\rhead{Seite \thepage von \pageref{LastPage}}
}
\makeatother

\usepackage{fancyhdr}
\fancyfoot[C]{}

\makeatletter
\renewenvironment{proof}[1][\proofname]{\par
  \pushQED{\qed}%
  \normalfont \topsep6\p@\@plus6\p@\relax
  \list{}{\leftmargin=4em
          \rightmargin=\leftmargin
          \settowidth{\itemindent}{\itshape#1}%
          \labelwidth=\itemindent}

  \item[\hskip\labelsep
        \itshape
    #1\@addpunct{.}]\ignorespaces
}{%
  \popQED\endlist\@endpefalse
}
\makeatother

\begin{document}
	\section*{Aufgabe 1}
\subsection*{Teilaufgabe a}
\textbf{Gegeben:}

\[A = 
\begin{pmatrix}
    3 & 15 & 13 \\
    6 & 6  & 6  \\
    2 & 8  & 19
\end{pmatrix}\]

\textbf{Aufgabe:} LR-Zerlegung von $A$ mit Spaltenpivotwahl

\textbf{Lösung:} 

\begin{align*}
	&
	&
    A^{(0)} &= \begin{gmatrix}[p]
		3 & 15 & 13 \\
		6 & 6  & 6  \\
		2 & 8  & 19
	 \rowops
	 \swap{0}{1}
	\end{gmatrix}
	&\\
    P^{(1)} &= \begin{pmatrix}
		0 & 1 & 0\\
		1 & 0 & 0\\
     	0 & 0 & 1
	\end{pmatrix},
	&
    A^{(1)} &= \begin{gmatrix}[p]
		6 & 6  & 6  \\
		3 & 15 & 13 \\
		2 & 8  & 19
	 \rowops
	 \add[\cdot (-\frac{1}{2})]{0}{1}
	 \add[\cdot (-\frac{1}{3})]{0}{2}
	\end{gmatrix}
	&\\
	L^{(2)} &= \begin{pmatrix}
		1 & 0 & 0\\
		-\frac{1}{2} & 1 & 0\\
     	-\frac{1}{3} & 0 & 1
	\end{pmatrix},
	&
    A^{(2)} &= \begin{gmatrix}[p]
		6 & 6  & 6  \\
		0 & 12 & 10 \\
		0 & 6  & 17
	 \rowops
	 \add[\cdot (-\frac{1}{2})]{1}{2}
	\end{gmatrix}
	&\\
	L^{(3)} &= \begin{pmatrix}
		1 & 0 & 0\\
		0 & 1 & 0\\
     	0 & -\frac{1}{2} & 1
	\end{pmatrix},
	&
    A^{(3)} &= \begin{gmatrix}[p]
		6 & 6  & 6  \\
		0 & 12 & 10 \\
		0 & 0  & 12
	\end{gmatrix}
\end{align*}

Es gilt:

\begin{align}
	L^{(3)} \cdot L^{(2)} \cdot \underbrace{P^{(1)}}_{=: P} \cdot A^{0} &= \underbrace{A^{(3)}}_{=: R}\\
	\Leftrightarrow P A &= (L^{(3)} \cdot L^{(2)})^{-1} \cdot R \\
	\Rightarrow L &= (L^{(3)} \cdot L^{(2)})^{-1}\\
	&= \begin{pmatrix}
		1 & 0 & 0\\
		\frac{1}{2} & 1 & 0\\
		\frac{1}{3} & \frac{1}{2} & 1
	\end{pmatrix}
\end{align}

Nun gilt: $P A = L R = A^{(1)}$ (Kontrolle mit \href{http://www.wolframalpha.com/input/?i=%7B%7B1%2C0%2C0%7D%2C%7B0.5%2C1%2C0%7D%2C%7B1%2F3%2C0.5%2C1%7D%7D*%7B%7B6%2C6%2C6%7D%2C%7B0%2C12%2C10%7D%2C%7B0%2C0%2C12%7D%7D}{Wolfram|Alpha})

\subsection*{Teilaufgabe b}

\textbf{Gegeben:}

\[A = 
\begin{pmatrix}
    9 & 4 & 12 \\
    4 & 1  & 4 \\
   12 & 4  & 17
\end{pmatrix}\]

\textbf{Aufgabe:} $A$ auf positive Definitheit untersuchen, ohne Eigenwerte zu berechnen.

\textbf{Vorüberlegung:}
Eine Matrix $A \in \mathbb{R}^{n \times n}$ heißt positiv Definit $\dots$
\begin{align*}
  \dots & \Leftrightarrow \forall x \in \mathbb{R}^n, x \neq 0: x^T A x > 0\\
	& \Leftrightarrow \text{Alle Eigenwerte sind größer als 0}
\end{align*}

Falls $A$ symmetrisch ist, gilt:
\begin{align*}
 \text{$A$ ist pos. Definit} & \Leftrightarrow \text{alle führenden Hauptminore von $A$ sind positiv}\\
	& \Leftrightarrow \text{es gibt eine Cholesky-Zerlegung $A=GG^T$ mit $G$ ist reguläre untere Dreiecksmatrix}\\
\end{align*}

\subsubsection*{Lösung 1: Hauptminor-Kriterium}

\begin{align}
	\det(A_1) &= 9 > 0\\
	\det(A_2) &= 
		\begin{vmatrix}
			9 & 4 \\
			4 & 1 \\
		\end{vmatrix} = 9 - 16 < 0\\
	&\Rightarrow \text{$A$ ist nicht positiv definit}
\end{align}

\subsubsection*{Lösung 2: Cholesky-Zerlegung}
\begin{align}
	l_{11} &= \sqrt{a_{11}} = 3\\
	l_{21} &= \frac{a_{21}}{l_{11}} = \frac{4}{3}\\
	l_{31} &= \frac{a_{31}}{l_{11}} = \frac{12}{3} = 4\\
	l_{22} &= \sqrt{a_{22} - {l_{21}}^2} = \sqrt{1 - \frac{16}{9}}= \sqrt{-\frac{7}{9}} \notin \mathbb{R}\\
 & \Rightarrow \text{Es ex. keine Cholesky-Zerlegung, aber $A$ ist symmetrisch}\\
 & \Rightarrow \text{$A$ ist nicht pos. Definit}
\end{align}

	\section*{Aufgabe 2}
\subsection*{Teilaufgabe a)}

\textbf{Behauptung:} Für $x \in \mathbb{R}$ gilt, dass $cos(x_k) = x_{k+1}$ gegen den einzigen Fixpunkt $x^{*} = cos(x^{*})$ konvergiert.

\textbf{Beweis:} 
Sei $ D := [-1, 1]$.\\
Trivial: $D$ ist abgeschlossen.

Sei $ x \in D$, so gilt:
\begin{align*}
	0 < cos(x) \leq 1
\end{align*}
Also: $cos(x) \in D$.\\ Wenn $x \not\in D$, so gilt $y := cos(x)$ und $cos(y) \in D$. D.h. bereits nach einem Iterationschritt wäre $cos(x) \in D$ für $x \in \mathbb{R}$! Dies ist wichtig, da damit gezeigt ist, dass $cos(x_k) = x_{k+1}$ für jedes $x \in \mathbb{R}$ konvergiert! Es kommt nur dieser einzige Iteratationsschritt für $x \not\in \mathbb{R}$ hinzu.

Nun gilt mit $ x, y \in D, x < y, \xi \in (x,y) $ und dem Mittelwert der Differentialrechnung:
\begin{align*}
	\frac{cos(x) - cos(y)}{x - y} = cos'(\xi) \\
	\Leftrightarrow cos(x) - cos(y) =  cos'(\xi) * (x - y)  \\
	\Leftrightarrow | cos(x) - cos(y) | = | cos'(\xi) * (x - y) | \leq | cos'(\xi) | * | (x - y) | 
\end{align*}
Da $ \xi \in (0, 1) $ gilt:
\begin{align*}
	0 \leq | cos'(\xi) | = | sin(\xi) | < 1 
\end{align*}
Damit ist gezeigt, dass $cos(x) : D \rightarrow D$ Kontraktion auf $D$.

Damit sind alle Voraussetzung des Banachschen Fixpunktsatzes erfüllt.

Nach dem Banachschen Fixpunktsatz folgt die Aussage.
	\section*{Aufgabe 3}
\[f' (x,y) = \begin{pmatrix}
	3     & \cos y\\
	3 x^2 & e^y
\end{pmatrix}\]

Und jetzt die Berechnung

\[f'(x, y) \cdot (x_0, y_0) = f(x,y)\]

LR-Zerlegung für $f'(x, y)$ kann durch scharfes hinsehen durchgeführt
werden, da es in $L$ nur eine unbekannte (links unten) gibt. Es gilt
also ausführlich:

\begin{align}
	\begin{pmatrix}
		3     & \cos y\\
		3 x^2 & e^y
	\end{pmatrix}
	&=
	\overbrace{\begin{pmatrix}
		1      & 0\\
		l_{12} & 1
	\end{pmatrix}}^L \cdot 
	\overbrace{\begin{pmatrix}
		r_{11} & r_{12}\\
		0      & r_{22}
	\end{pmatrix}}^R\\
	\Rightarrow r_{11} &= 3\\
	\Rightarrow r_{12} &= \cos y\\
	\Rightarrow \begin{pmatrix}
		3     & \cos y\\
		3 x^2 & e^y
	\end{pmatrix}
	&=
	\begin{pmatrix}
		1      & 0\\
		l_{12} & 1
	\end{pmatrix} \cdot 
	\begin{pmatrix}
		3 & \cos y\\
		0 & r_{22}
	\end{pmatrix}\\
	\Rightarrow 3x^2 &\stackrel{!}{=} l_{12} \cdot 3 + 1 \cdot 0\\
	\Leftrightarrow l_{12} &= x^2\\
	\Rightarrow e^y &\stackrel{!}{=} x^2 \cdot \cos y + 1 \cdot r_{22}\\
	\Leftrightarrow r_{22} &= -x^2 \cdot \cos y + e^y\\
	\Rightarrow \begin{pmatrix}
		3     & \cos y\\
		3 x^2 & e^y
	\end{pmatrix}
	&=
	\begin{pmatrix}
		1   & 0\\
		x^2 & 1
	\end{pmatrix} \cdot 
	\begin{pmatrix}
		3 & \cos y\\
		0 & -x^2 \cdot \cos y + e^y
	\end{pmatrix}\\
	P &= I_2\\
-f ( \frac{-1}{3}, 0) &= \begin{pmatrix} 2\\ -\frac{26}{27}\end{pmatrix}\\
c &= \begin{pmatrix} 2\\ \frac{82}{27} \end{pmatrix}\\
(x_1, y_1) &= \begin{pmatrix} \frac{5}{3}\\ \frac{82}{27}\end{pmatrix}
\end{align}

	\section*{Aufgabe 4}
\subsection*{Teilaufgabe a)}
\begin{enumerate}
    \item Ordnung 3 kann durch geschickte Gewichtswahl erzwungen werden.
    \item Ordnung 4 ist automatisch gegeben, da die QF symmetrisch sein soll.
    \item Aufgrund der Symmetrie gilt Äquivalenz zwischen Ordnung 5 und 6.
          Denn eine hätte die QF Ordnung 5, so wäre wegen der
          Symmetrie Ordnung 6 direkt gegeben. Ordnung 6 wäre aber
          bei der Quadraturformel mit 3 Knoten das Maximum, was nur
          mit der Gauß-QF erreicht werden kann. Da aber $c_1 = 0$ gilt,
          kann es sich hier nicht um die Gauß-QF handeln. Wegen
          erwähnter Äquivalenz kann die QF auch nicht Ordnung 5 haben.
\end{enumerate}

Da $c_1 = 0$ gilt, muss $c_3 = 1$ sein (Symmetrie). Und dann muss $c_2 = \frac{1}{2}$
sein. Es müssen nun die Gewichte bestimmt werden um Ordnung 3 zu
garantieren mit:

\begin{align}
    b_i &= \int_0^1 L_i(x) \mathrm{d}x\\
    b_1 &= \frac{1}{6},\\
    b_2 &= \frac{4}{6},\\
    b_3 &= \frac{1}{6}
\end{align}

\subsection*{Teilaufgabe b)}
Als erstes ist festzustellen, dass es sich hier um die Simpsonregel handelt und die QF
\begin{align}
    \int_a^b f(x) \mathrm{d}x &= (b-a) \cdot \frac{1}{6} \cdot \left ( f(a) + 4 \cdot f(\frac{a+b}{2}) + f(b) \right )
\end{align}

ist. Wenn diese nun auf $N$ Intervalle aufgepflittet wird gilt folgendes:

\begin{align}
    \int_a^b f(x) \mathrm{d}x &= (b-a) \cdot \frac{1}{6} \cdot \left [ f(a) + f(b) + 2 \cdot \sum_{i=1}^{N-1} f(i \cdot \frac{1}{N}) + 4 \cdot \sum_{i=1}^N f(i \cdot \frac{1}{2N})\right ]
\end{align}

$\sum_{i=1}^{N-1} f(i \cdot \frac{1}{N})$  sind die Grenzknoten der Intervalle
 (deshalb werden sie doppelt gezählt). Von den Grenzknoten gibt es
insgesamt $s-2$ Stück, da die tatsächlichen Integralgrenzen $a$ und $b$
nur einmal in die Berechnung mit einfließen.

$\sum_{i=1}^N f(i \cdot \frac{1}{2N})$ sind die jeweiligen
mittleren Knoten der Intervalle. Davon gibt es $s-1$ Stück.

\begin{figure}[h]
    \centering
    \includegraphics*[width=\linewidth, keepaspectratio]{aufgabe4-b.png}
\end{figure}

\subsection*{Teilaufgabe c)}
TODO

	\section*{Aufgabe 5}
\subsection*{Teilaufgabe a}
Eine Quadraturformel $(b_i, c_i)_{i=1, \dots, s}$ hat die Ordnung
$p$, falls sie exakte Lösungen für alle Polynome vom Grad $\leq p -1$
liefert.

\subsection*{Teilaufgabe b}
Für die ersten 3. Ordnungsbedingungen gilt:

\begin{align*}
	1 = \sum_{i = 0}^{s} b_i \\
 	\frac{1}{2} = \sum_{i = 0}^{s} b_i * c_i \\
 	\frac{1}{3} = \sum_{i = 0}^{s} b_i * c_i^2
\end{align*}

\subsection*{Teilaufgabe c}
Da die Ordnung 4 gewünscht ist müssen nach VL die Knoten der QF symmetrisch sein. Damit folgt sofort $c_2 = \frac{1}{2}$. Sind die Knoten gewählt, so sind die Gewichte eindeutig bestimmt. Die Berechnung erfolgt mit den Lagrangepolynomen. Es gilt $b_0 = b_2 = \frac{1}{6}, b_1 = \frac{4}{6}$. Entweder man setzt alles in die 4. Ordnungsbedingung ein oder aber argumentiert, dass es sich hierbei um die Simpson-Regel handelt und diese die Ordnung 4 hat.
\end{document}
