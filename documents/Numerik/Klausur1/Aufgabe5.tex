\section*{Aufgabe 5}
\subsection*{Teilaufgabe a}
Eine Quadraturformel $(b_i, c_i)_{i=1, \dots, s}$ hat die Ordnung
$p$, falls sie exakte Lösungen für alle Polynome vom Grad $\leq p -1$
liefert.

\subsection*{Teilaufgabe b}
\[\sum_{i=1}^s b_i c_i^{q-1} = \frac{1}{q} \text{ für } q = 1, \dots, p\]

\subsection*{Teilaufgabe c}
\paragraph{Aufgabe} Bestimmen Sie zu den Knoten $c_1 = 0$ und $c_2 = \frac{2}{3}$ Gewichte, um eine Quadraturformel
maximaler Ordnung zu erhalten. Wie hoch ist die Ordnung?

\paragraph{Lösung}

Als erstes stellen wir fest, dass die Knoten nicht symmetrisch (d.h. 
gespiegelt bei $\frac{1}{2}$) sind. TODO: Warum ist das wichtig?

$\stackrel{\text{Satz 28}}{\Rightarrow}$ Wenn wir Ordnung $s = 2$ fordern, sind die Gewichte eindeutig bestimmt.

Da $c_1 = 0$ kann es sich nicht um die Gauß-QF handeln. Somit können 
wir nicht Ordnung 4  erreichen.

Nach VL kann bei Vorgabe von $s$ Knoten auch die Ordnung $s$ durch 
geschickte Wahl der Gewichte erreicht werden. 
Also berechnen wir die Gewichte, um die Ordnung 2 zu sichern:

\begin{align}
	b_i &= \int_0^1 L_i(x) \mathrm{d}x\\
	b_1 &= \frac{1}{4}\\
	b_2 &= \frac{3}{4}
\end{align}

Diese Gewichte $b_1, b_2$ erfüllen die 1. und 2. Ordnungsbedingung.

\begin{align}
	\frac{1}{3} &= \sum_{i=1}^2 b_i \cdot c_i^2\\
			&= \frac{3}{4} \cdot \frac{4}{9}\\
			&= \frac{1}{3}
\end{align}

Damit ist auch die 3. Ordnungsbedingung und mit den Knoten maximale Ordnung erfüllt.
