\section*{Aufgabe 5}
\subsection*{Teilaufgabe a}
Eine Quadraturformel $(b_i, c_i)_{i=1, \dots, s}$ hat die Ordnung
$p$, falls sie exakte Lösungen für alle Polynome vom Grad $\leq p -1$
liefert.

\subsection*{Teilaufgabe b}
Die Ordnungsbedingungen, mit denen man zeigen kann, dass eine Quadraturformel
mindestens Ordnung $p$ hat, lautet:
\[\forall p \in \Set{1, \dots, p}: \sum_{i=1}^s b_i c_i^{q-1} = \frac{1}{q}\]

\subsection*{Teilaufgabe c}
\paragraph{Aufgabe} Bestimmen Sie zu den Knoten $c_1 = 0$ und $c_2 = \frac{2}{3}$ Gewichte, um eine Quadraturformel
maximaler Ordnung zu erhalten. Wie hoch ist die Ordnung?

\paragraph{Lösung}

Nach VL kann bei Vorgabe von $s$ Knoten auch die Ordnung $s$ durch
geschickte Wahl der Gewichte erreicht werden. Nach Satz 27 ist diese
Wahl eindeutig.
Also berechnen wir die Gewichte, um die Ordnung $p=2$ zu sichern.

Dazu stellen wir zuerst die Lagrange-Polynome auf:

\begin{align}
	L_1(x) &= \frac{x-x_2}{x_1 - x_2} = \frac{x-c_2}{c_1-c_2} = \frac{x-\nicefrac{2}{3}}{-\nicefrac{2}{3}} = -\frac{3}{2} x + 1\\
    L_2(x) &= \frac{x-x_1}{x_2 - x_1} = \frac{x-c_1}{c_2-c_1} = \frac{x}{\nicefrac{2}{3}} = \frac{3}{2} x
\end{align}

Nun gilt für die Gewichte:
\begin{align}
	b_i &= \int_0^1 L_i(x) \mathrm{d}x\\
	b_1 &= \int_0^1 -\frac{3}{2} x + 1 \mathrm{d}x = \left [ -\frac{3}{4}x^2 + x \right ]_0^1 = \frac{1}{4}\\
	b_2 &= \frac{3}{4}
\end{align}

Nun sind die Ordnungsbedingungen zu überprüfen:
\begin{align}
    \nicefrac{1}{1} &\stackrel{?}{=} b_1 c_1^0 + b_2 c_2^0 = \nicefrac{1}{4} + \nicefrac{3}{4} \text{\;\;\cmark}\\
    \nicefrac{1}{2} &\stackrel{?}{=} b_1 c_1^1 + b_2 c_2^1 = \frac{1}{4} \cdot 0 + \frac{3}{4} \cdot \frac{2}{3} \text{\;\;\cmark}\\
    \nicefrac{1}{3} &\stackrel{?}{=} b_1 c_1^2 + b_2 c_2^2 = \frac{1}{4} \cdot 0 + \frac{3}{4} \cdot \frac{4}{9} \text{\;\;\cmark}\\
    \nicefrac{1}{4} &\stackrel{?}{=} b_1 c_1^3 + b_2 c_2^3 = \frac{1}{4} \cdot 0 + \frac{3}{4} \cdot \frac{8}{27} \text{\;\;\xmark}\\
\end{align}

Die Quadraturformel mit den Knoten $c_1 = 0$, $c_2 = \nicefrac{2}{3}$ sowie
den Gewichten $b_1 = \nicefrac{1}{4}$, $b_2 = \nicefrac{3}{4}$ erfüllt
also die 1., 2. und 3. Ordnungsbedingung, nicht jedoch die 4.
Ordnungsbedingung. Ihre maximale Ordnung ist also $p=3$.

\textbf{Anmerkungen:} Da $c_1 = 0$ kann es sich nicht um die Gauß-QF handeln.
Somit können wir nicht Ordnung $p=4$ erreichen.

Bei der Suche nach den Gewichten hätte man alternativ auch das folgende
LGS lösen können:

\begin{align}
    \begin{pmatrix}
        c_1^0 & c_2^0\\
        c_1^1 & c_2^1
    \end{pmatrix}
    \cdot x
    =
    \begin{pmatrix}
        1\\
        \nicefrac{1}{2}
    \end{pmatrix}
\end{align}
