\section*{Aufgabe 4}
\textbf{Aufgabe}: 

\[I(f) = \int_a^b f(x) \mathrm{d}x \]

\begin{enumerate}
    \item Integrand am linken und am rechten Rand interpolieren
	\item Interpolationspolynom mit Quadraturformel integrieren
\end{enumerate}

\textbf{Lösung}:

Nutze Interpolationsformel von Lagrange:

\[p(x) = \sum_{i=0}^{1} f_i \cdot L_i(x)\]

Berechne Lagrangepolynome:

\begin{align}
    L_0(x) = \frac{x-b}{a-b} \\
    L_1(x) = \frac{x-a}{b-a}
\end{align}

So erhalten wir:

\[p(x) = f(a) \frac{x-b}{a-b} + f(b) \frac{x-a}{b-a}\]

Nun integrieren wir das Interpolationspolynom:

\[ \int_a^b p(x)dx = \int_a^b f(a) \frac{x-b}{a-b}dx + \int_a^b f(b) \frac{x-a}{b-a}dx \]
\[ = \int_a^b \frac{f(a) \cdot x}{a-b}dx - \int_a^b \frac{f(a) \cdot b}{a-b}dx + \int_a^b \frac{f(b) \cdot x}{b-a}dx - \int_a^b \frac{f(b) \cdot a}{b-a}dx \]
\[ = \frac{1}{2} \cdot \frac{f(a) \cdot b^2}{a-b} - \frac{1}{2} \cdot \frac{f(a) \cdot a^2}{a-b} - \frac{f(a) \cdot b^2}{a-b} + \frac{f(a) \cdot b \cdot a}{a-b} + \frac{1}{2} \cdot \frac{f(b) \cdot b^2}{b-a} \]
\[ - \frac{1}{2} \cdot \frac{f(b) \cdot a^2}{b-a} - \frac{f(b) \cdot a \cdot b}{b-a} + \frac{f(b) \cdot a^2}{b-a}\]
\[=(b-a)\cdot(\frac{f(a)}{2} + \frac{f(b)}{2})\]

Betrachtet man nun die allgemeine Quadraturformel,
\[
\int_a^b f(x)dx \approx (b-a) \sum_{i=1}^s b_i f(a+c_i(b-a))
\]
so gilt für die hergeleitete Quadraturformel also $s=2$, $c_1=0, c_2=1$ und $b_1 = b_2 = \frac{1}{2}$. Sie entspricht damit der Trapezregel.

\subsection*{Teilaufgabe b)}
Sei nun $f(x) = x^2$ und $a = 0$ sowie $b = 4$. Man soll die ermittelte
Formel zwei mal auf äquidistanten Intervallen anwenden.

\textbf{Lösung:}

\begin{align}
	\int_0^4 p(x) dx = \int_0^2 p(x)dx + \int_2^4 p(x)dx = 24
\end{align}
