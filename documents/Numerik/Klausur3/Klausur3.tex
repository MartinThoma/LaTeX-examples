\documentclass[a4paper]{scrartcl}
\usepackage{amssymb, amsmath} % needed for math
\usepackage[utf8]{inputenc} % this is needed for umlauts
\usepackage[ngerman]{babel} % this is needed for umlauts
\usepackage[T1]{fontenc}    % this is needed for correct output of umlauts in pdf
\usepackage{pdfpages}       % Signatureinbingung und includepdf
\usepackage{geometry}       % [margin=2.5cm]layout
\usepackage[pdftex]{hyperref}       % links im text
\usepackage{color}
\usepackage{framed}
\usepackage{enumerate}      % for advanced numbering of lists
\usepackage{marvosym}       % checkedbox
\usepackage{wasysym}
\usepackage{braket}         % for \Set{}
\usepackage{pifont}% http://ctan.org/pkg/pifont
\usepackage{gauss}
\usepackage{algorithm,algpseudocode}
\usepackage{parskip}
\usepackage{lastpage}
\allowdisplaybreaks

\newcommand{\cmark}{\ding{51}}%
\newcommand{\xmark}{\ding{55}}%

\title{Numerik Klausur 3 - Musterlösung}
\makeatletter
\AtBeginDocument{
	\hypersetup{ 
	  pdfauthor   = {Felix Benz-Baldas, Martin Thoma, Peter},
	  pdfkeywords = {Numerik, KIT, Klausur}, 
	  pdftitle    = {\@title} 
  	}
	\pagestyle{fancy}
	\lhead{\@title}
	\rhead{Seite \thepage{} von \pageref{LastPage}}
}
\makeatother

\usepackage{fancyhdr}
\fancyfoot[C]{}

\begin{document}
	\section*{Aufgabe 1}
\subsection*{Teilaufgabe a}
\textbf{Gegeben:}

\[A = 
\begin{pmatrix}
    3 & 15 & 13 \\
    6 & 6  & 6  \\
    2 & 8  & 19
\end{pmatrix}\]

\textbf{Aufgabe:} LR-Zerlegung von $A$ mit Spaltenpivotwahl

\textbf{Lösung:} 

\[P = 
\begin{pmatrix}
    0 & 1 & 0 \\
    1 & 0 & 0  \\
    0 & 0 & 1
\end{pmatrix}\]

durch scharfes hinsehen.

Nun $L, R$ berechnen:

\begin{align}
	&\begin{gmatrix}[p]
		6 & 6  & 6  \\
		3 & 15 & 13 \\
		2 & 8  & 19
	 \rowops
	 \add[\cdot (-\frac{1}{2})]{0}{1}
	 \add[\cdot (-\frac{1}{3})]{0}{2}
	\end{gmatrix}
	\\
  = \begin{pmatrix}
		          1 & 0 & 0 \\
	   -\frac{1}{2} & 1 & 0  \\
	   -\frac{1}{3} & 0 & 1
	\end{pmatrix} \cdot
	&\begin{gmatrix}[p]
		6 & 6  & 6  \\
		0 & 12 & 10 \\
		0 & 6  & 17
	 \rowops
	 \add[\cdot (-\frac{1}{2})]{1}{2}
	\end{gmatrix}
	\\
  = \begin{pmatrix}
		          1 & 0 & 0 \\
	              0 & 1 & 0  \\
	              0 & -\frac{1}{2} & 1
	\end{pmatrix} \cdot
    \begin{pmatrix}
		          1 & 0 & 0 \\
	   -\frac{1}{2} & 1 & 0  \\
	   -\frac{1}{3} & 0 & 1
	\end{pmatrix} \cdot
	&\begin{gmatrix}[p]
		6 & 6  & 6  \\
		0 & 12 & 10 \\
		0 & 0  & 12
	 \colops
	 \add[\cdot (-1)]{0}{1}
	 \add[\cdot (-1)]{0}{2}
	\end{gmatrix}
	\\
  = \begin{pmatrix}
		          1 & 0 & 0 \\
	   -\frac{1}{2} & 1 & 0  \\
	   -\frac{1}{12} & - \frac{1}{2} & 1
	\end{pmatrix} \cdot
	&\begin{gmatrix}[p]
		6 & 0  & 0  \\
		0 & 12 & 10 \\
		0 & 0  & 12
	 \colops
	 \add[\cdot (-\frac{10}{12})]{1}{2}
	\end{gmatrix}
	\cdot
	\begin{pmatrix}
		          1 & -1 & -1 \\
	              0 &  1 &  0  \\
	              0 &  0 &  1
	\end{pmatrix}
	\\
  = \begin{pmatrix}
		          1 & 0 & 0 \\
	   -\frac{1}{2} & 1 & 0  \\
	   -\frac{1}{12} & - \frac{1}{2} & 1
	\end{pmatrix} \cdot
	&\begin{gmatrix}[p]
		6 & 0  & 0 \\
		0 & 12 & 0 \\
		0 & 0  & 12
	 \colops
	  	\mult{0}{\cdot \frac{1}{6}}
	  	\mult{1}{\cdot \frac{1}{12}}
	  	\mult{2}{\cdot \frac{1}{12}}
	\end{gmatrix}
	\cdot
	\begin{pmatrix}
		          1 & -1 & -1 \\
	              0 &  1 &  0 \\
	              0 &  0 &  1
	\end{pmatrix}
	\cdot
	\begin{pmatrix}
		          1 &  0 &  0 \\
	              0 &  1 &  -\frac{10}{12} \\
	              0 &  0 &  1
	\end{pmatrix}
	\\
  = \begin{pmatrix}
		          1 & 0 & 0 \\
	   -\frac{1}{2} & 1 & 0  \\
	   -\frac{1}{12} & - \frac{1}{2} & 1
	\end{pmatrix} \cdot
	&\begin{gmatrix}[p]
		1 & 0 & 0 \\
		0 & 1 & 0 \\
		0 & 0 & 1
	\end{gmatrix}
	\cdot
	\begin{pmatrix}
		          1 & -1 & -\frac{1}{6} \\
	              0 &  1 & -\frac{5}{6} \\
	              0 &  0 &  1
	\end{pmatrix}
	\cdot
	\begin{pmatrix}
	    \frac{1}{6} &  0 & 0 \\
	              0 &  \frac{1}{12} & 0 \\
	              0 &  0 & \frac{1}{12}
	\end{pmatrix}
	\\
  = \underbrace{\begin{pmatrix}
		          1 & 0 & 0 \\
	   -\frac{1}{2} & 1 & 0  \\
	   -\frac{1}{12} & - \frac{1}{2} & 1
	\end{pmatrix}}_L \cdot
	&\begin{gmatrix}[p]
		1 & 0 & 0 \\
		0 & 1 & 0 \\
		0 & 0 & 1
	\end{gmatrix}
	\cdot \underbrace{\frac{1}{72}
	\begin{pmatrix}
		          12 & -6 & -1 \\
	               0 &  6 & -5 \\
	               0 &  0 &  6
	\end{pmatrix}}_R
\end{align}

ACHTUNG: Ich habe mich irgendwo verrechnet!
Siehe \href{http://www.wolframalpha.com/input/?i=%7B%7B1%2C0%2C0%7D%2C%7B-1%2F2%2C1%2C0%7D%2C%7B-1%2F12%2C-1%2F2%2C1%7D%7D*%7B%7B12%2C-6%2C-1%7D%2C%7B0%2C6%2C-5%7D%2C%7B0%2C0%2C6%7D%7D}{WolframAlpha}

\subsection*{Teilaufgabe b}

\textbf{Gegeben:}

\[A = 
\begin{pmatrix}
    9 & 4 & 12 \\
    4 & 1  & 4 \\
   12 & 4  & 17
\end{pmatrix}\]

\textbf{Aufgabe:} $A$ auf positive Definitheit untersuchen, ohne Eigenwerte zu berechnen.

\textbf{Lösung:}
Eine Matrix $A \in \mathbb{R}^{n \times n}$ heißt positiv Definit $\dots$
\begin{align*}
  \dots & \Leftrightarrow \forall x \in \mathbb{R}^n: x^T A x > 0\\
	& \Leftrightarrow \text{Alle Eigenwerte sind größer als 0}
\end{align*}

Falls $A$ symmetrisch ist, gilt:
\begin{align*}
 \text{$A$ ist pos. Definit} & \Leftrightarrow \text{alle führenden Hauptminore von $A$ sind positiv}\\
	& \Leftrightarrow \text{es gibt eine Cholesky-Zerlegung $A=GG^T$ mit $G$ ist reguläre untere Dreiecksmatrix}\\
\end{align*}

Mit dem Hauptminor-Kriterium gilt:

\begin{align}
	\det(A_1) &= 9 > 0\\
	\det(A_2) &= 
		\begin{vmatrix}
			9 & 4 \\
			4 & 1 \\
		\end{vmatrix} = 9 - 16 < 0\\
	&\Rightarrow \text{$A$ ist nicht positiv definit}
\end{align}

	\section*{Aufgabe 2}
\subsection*{Teilaufgabe i}
Es gilt:

\begin{align}
    2x - e^{-x} &= 0\\
    \Leftrightarrow 2x &= e^{-x}\\
\end{align}

Offensichtlich ist $g(x) := 2x$ streng monoton steigend und $h(x) := e^{-x}$ streng
monoton fallend.

Nun gilt: $g(0) = 0 < 1 = e^0 = h(0)$. Das heißt, es gibt keinen
Schnittpunkt für $x \leq 0$.

Außerdem: $g(1) = 2$ und $h(1) = e^{-1} = \frac{1}{e} < 2$.
Das heißt, für $x \geq 1$ haben $g$ und $h$ keinen Schnittpunkt.

Da $g$ und $h$ auf $[0,1]$ stetig sind und $g(0) < h(0)$ sowie $g(1) > h(1)$
gilt, müssen sich $g$ und $h$ im Intervall mindestens ein mal schneiden.
Da beide Funktionen streng monoton sind, schneiden sie sich genau
ein mal.

Ein Schnittpunkt der Funktion $g,h$ ist äquivalent zu einer
Nullstelle der Funktion $f$. Also hat $f$ genau eine Nullstelle
und diese liegt in $[0,1]$.

\subsection*{Teilaufgabe ii}
    \begin{align}
        2x - e^{-x} &= 0\\
    \Leftrightarrow 2x &= e^{-x}\\
    \Leftrightarrow x &= \frac{1}{2} \cdot e^{-x} = F_1(x) \label{a2iif1}\\
    \stackrel{x \in \mathbb{R}^+}{\Rightarrow} \ln(2x) &= -x\\
    \Leftrightarrow x &= - \ln(2x) = F_2(x)\label{a2iif2}
    \end{align}

Gleichung \ref{a2iif1} zeigt, dass der Fixpunkt von $F_1$ mit der 
Nullstelle von $f$ übereinstimmt.

Gleichung \ref{a2iif2} zeigt, dass der Fixpunkt von $F_1$ mit der 
Nullstelle von $f$ übereinstimmt. Da es nur in $[0,1]$ eine Nullstelle
gibt (vgl. Teilaufgabe i), ist die Einschränkung von $x$ auf $\mathbb{R}^+$
irrelevant.

Man sollte $F_1$ zur Fixpunktiteration verwenden, da $\ln(x)$ nur für
$x>0$ definiert ist. Bei der Iteration kommt man aber schnell in
einen Bereich, der nicht erlaubt ist (das erlaubte Intervall ist klein;
Rechenungenauigkeit)

$F_1$ ist auf $[0,1]$ eine Kontraktion mit Kontraktionszahl $\theta = \frac{1}{2}$:

Nach dem Mittelwertsatz der Differentialrechnung ex. ein $\xi \in (a,b)$ mit $ 0 \leq a < b \leq 1$, sodass 
gilt:


\begin{align}
    \frac{F(b) - F(a)}{b-a} &= f'(\xi) \\
    \Leftrightarrow \frac{F(b) - F(a)}{b-a} &= - \frac{1}{2} e^{- \xi} \\
    \Leftrightarrow \frac{\|F(b) - F(a)\|}{\|b-a\|} &= \frac{1}{2} \frac{1}{e^{\xi}} < \frac{1}{2 e^a} \\
    \Leftrightarrow \|F(b) - F(a)\| &< \frac{1}{2 e^a} |b-a|\\
    \Rightarrow \forall x, y \in [0,1]: |F(x) - F(y)| &< \frac{1}{2} |x-y|
\end{align}

$F_2$ ist auf $(0,1]$ eine Kontraktion mit Kontraktionszahl $\theta$:
\begin{align}
    \|- \ln (2x) + \ln(2y) \| &\leq \theta \cdot \|x-y\|\\
    \Leftrightarrow \| \ln(\frac{2y}{2x}) \| &\leq \theta \cdot \|x-y\|\\
    \Leftrightarrow \| \ln(\frac{y}{x}) \| &\leq \theta \cdot \|x-y\|
\end{align}

TODO: Beweis ist nicht mal wirklich angefangen

Gegen $F_2$ spricht auch, dass $\log$ nur auf $\mathbb{R}^+$ definiert
ist. Das kann bei Rundungsfehlern eventuell zu einem Fehler führen.
(vgl. Python-Skript)

\subsection*{Teilaufgabe iii}
\[x_{k+1} = x_k - \frac{2x_k - e^{-x_k}}{2 + e^{-x_k}}\]

Laut \href{http://www.wolframalpha.com/input/?i=2x-e%5E(-x)%3D0}{Wolfram|Alpha} ist die Lösung etwa 0.35173371124919582602

	\section*{Aufgabe 3}

(Die Lösung findet ihr bei Klausur 3 / Aufgabe 3, da die Aufgaben identisch sind.)

	\section*{Aufgabe 4}
Ein Polynom, das vier Punkte interpoliert, hat maximal Grad 3.
Da wir das Integral über dieses Polynom im Bereich $[x_2, x_3]$
exakt berechnen sollen, muss die Quadraturformel vom Grad $p=4$ sein.

TODO

	\section*{Aufgabe 5}
TODO

\end{document}
