\section*{Aufgabe 5}
\subsection*{Teilaufgabe a}
Eine Quadraturformel $(b_i, c_i)_{i=1, \dots, s}$ hat die Ordnung
$p$, falls sie exakte Lösungen für alle Polynome vom Grad $\leq p -1$
liefert.

\subsection*{Teilaufgabe b}
Für die ersten 3. Ordnungsbedingungen gilt:

\begin{align*}
	1 = \sum_{i = 0}^{s} b_i \\
 	\frac{1}{2} = \sum_{i = 0}^{s} b_i * c_i \\
 	\frac{1}{3} = \sum_{i = 0}^{s} b_i * c_i^2
\end{align*}

\subsection*{Teilaufgabe c}
Da die Ordnung 4 gewünscht ist müssen nach VL die Knoten der QF symmetrisch sein. Damit folgt sofort $c_2 = \frac{1}{2}$. Sind die Knoten gewählt, so sind die Gewichte eindeutig bestimmt. Die Berechnung erfolgt mit den Lagrangepolynomen. Es gilt $b_0 = b_2 = \frac{1}{6}, b_1 = \frac{4}{6}$.