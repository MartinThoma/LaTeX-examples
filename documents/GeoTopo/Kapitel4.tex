%%%%%%%%%%%%%%%%%%%%%%%%%%%%%%%%%%%%%%%%%%%%%%%%%%%%%%%%%%%%%%%%%%%%%
% Mitschrieb vom 09.01.2014                                         %
%%%%%%%%%%%%%%%%%%%%%%%%%%%%%%%%%%%%%%%%%%%%%%%%%%%%%%%%%%%%%%%%%%%%%
\chapter{Euklidische und Nichteuklidische Geometrie}
\section{Axiome für die euklidische Ebene}
Axiome\xindex{Axiom} bilden die Grundbausteine jeder mathematischen Theorie. Eine
Sammlung aus Axiomen nennt man Axiomensystem\xindex{Axiomensystem}.
Da der Begriff des Axiomensystems so grundlegend ist, hat man auch 
ein paar sehr grundlegende Vorderungen an ihn: Axiomensysteme sollen
\textbf{widerspruchsfrei} sein, die Axiome sollen möglichst
\textbf{unabhängig} sein und \textbf{Vollständigkeit} wäre auch toll.
Mit Unabhängigkeit ist gemeint, dass kein Axiom sich aus einem anderem
herleiten lässt. Dies scheint auf den ersten Blick eine einfache
Eigenschaft zu sein. Auf den zweiten Blick muss man jedoch einsehen, 
dass das Parallelenproblem, also die Frage ob das Parallelenaxiom 
unabhängig von den restlichen Axiomen ist, über 2000 Jahre nicht 
gelöst wurde. Ein ganz anderes Kaliber ist die Frage nach der
Vollständigkeit. Ein Axiomensystem gilt als Vollständig, wenn
jede Aussage innerhalb des Systems verifizierbar oder falsifizierbar
ist. Interessant ist hierbei der Gödelsche Unvollständigkeitssatz, 
der z.~B. für die Arithmetik beweist, dass nicht alle Aussagen
formal bewiesen oder wiederlegt werden können.

Kehren wir nun jedoch zurück zur Geometrie. Euklid hat in seiner 
Abhandlung \enquote{Die Elemente} ein Axiomensystem für die Geometrie
aufgestellt. 

\textbf{Euklids Axiome}
\begin{itemize}
    \item \textbf{Strecke} zwischen je zwei Punkten
    \item Jede Strecke bestimmt genau eine \textbf{Gerade}
    \item \textbf{Kreis} (um jeden Punkt mit jedem Radius)
    \item Je zwei rechte Winkel sind gleich (Isometrie, Bewegung)
    \item Parallelenaxiom: Euklid:\\
        Wird eine Gerade so von zwei Geraden geschnitten, dass die 
        Summe der Innenwinkel zwei Rechte ist, dann schneiden sich
        diese Geraden auf der Seite dieser Winkel.
        \todo[inline]{Bild}
        Man mache sich klar, dass das nur dann nicht der Fall ist, 
        wenn beide Geraden parallel sind und senkrecht auf die erste stehen.
\end{itemize}

\begin{definition}\xindex{Ebene!euklidische}%In Vorlesung: Definition 14.2
    Eine \textbf{euklidische Ebene} ist ein metrischer Raum $(X,d)$ 
    zusammen mit einer Teilmenge $G \subseteq \powerset{X}$, sodass die
    Axiome~\ref{axiom:1}~-~IV erfüllt sind:
    \begin{enumerate}[label=§\arabic*),ref=§\arabic*]
        \item \enquote{Inzidenzaxiome}:\label{axiom:1}
            \begin{enumerate}[label=(\roman*),ref=\theenumi{} (\roman*)]
                \item Zu $P \neq Q \in X$ gibt es genau ein $g \in G$ mit
                      $\Set{P, Q} \subseteq g$.
                \item $|g| \geq 2 \;\;\; \forall g \in G$
                \item $X \in G$
            \end{enumerate}
        \item \enquote{Abstandsaxiom}: Zu $P, Q, R \in X$ gibt es \label{axiom:2}
              genau dann ein $g \in G$ mit $\Set{P, Q, R} \subseteq g$,
              wenn gilt: 
              \begin{itemize}[]
                \item $d(P, R) = d(P, Q) + d(Q, R)$ oder
                \item $d(P, Q) = d(P, R) + d(R, Q)$ oder
                \item $d(Q, R) = d(Q, P) + d(P, R)$
              \end{itemize}
    \end{enumerate}
\end{definition}

\begin{definition}
    \begin{enumerate}[label=\alph*)]
        \item $P, Q, R$ liegen \textbf{kolinear}\xindex{kolinear}, 
              wenn es $g \in G$ gibt mit $\Set{P, Q, R} \subseteq g$.
        \item $Q$ \textbf{liegt zwischen}\xindex{liegt zwischen} $P$
              und $R$, wenn $d(P, R) = d(P, Q) + d(Q, R)$
        \item $\overline{PR} := \Set{Q \in X | Q \text{ liegt zwischen } P \text{ und } R}$
        \item $PR^+ := \Set{Q \in X | Q \text{ liegt zwischen } P \text{ und } R \text{ oder } R \text{ liegt zwischen } P \text{ und } Q}$\\
              $PR^- := \Set{Q \in X | P \text{ liegt zwischen } Q \text{ und } R}$\\ 
    \end{enumerate}
\end{definition}

\begin{korollar}
    \begin{enumerate}[label=(\roman*)]
        \item $PR^+ \cup PR^- = PR$
        \item $PR^+ \cap PR^- = \Set{P}$
    \end{enumerate}
\end{korollar}

\begin{beweis}\leavevmode
    \begin{enumerate}[label=(\roman*)]
        \item \enquote{$\subseteq$} folgt direkt aus der Definition von $PR^+$ und $PR^-$\\
              \enquote{$\supseteq$}: Sei $Q \in PR \Rightarrow P, Q, R$ 
              sind kolinear.\\
              $\stackrel{\ref{axiom:2}}{\Rightarrow}
              \begin{cases} 
                Q \text{ liegt zwischen } P \text{ und } R \Rightarrow Q \in PR\\
                R \text{ liegt zwischen } P \text{ und } Q \Rightarrow Q \in PR\\
                P \text{ liegt zwischen } Q \text{ und } R \Rightarrow Q \in PR
              \end{cases}$
        \item \enquote{$\supseteq$} ist offensichtlich\\
              \enquote{$\subseteq$}: Sei $PR^+ \cap PR^-$. Dann ist
              $d(Q,R) = d(P,Q) + d(P,R)$ weil $Q \in PR^-$ und
              \begin{align*}
                &\left \{ \begin{array}{l}
                        d(P,R) = d(P,Q) + d(Q,R) \text{ oder }\\
                        d(P,Q) = d(P,R) + d(R,Q)
                       \end{array} \right \}\\
                &\Rightarrow d(Q,R) = 2d(P,Q) + d(Q,R)\\
                &\Rightarrow d(P,Q) = 0\\
                &\Rightarrow P=Q\\
                &d(P,Q) = 2d(P,R) + d(P,Q)\\
                &\Rightarrow P=R\\
                &\Rightarrow \text{Widerspruch}
              \end{align*}
    \end{enumerate}
\end{beweis}

\begin{definition}
    \begin{enumerate}[label=§\arabic*),ref=§\arabic*,start=3]
        \item \enquote{Anordnungsaxiom}\label{axiom:3}
            \begin{enumerate}[label=(\roman*),ref=§\theenumi{} (\roman*)]
                \item  Zu jedem $P \in X$ jeder Halbgerade $H$ mit \label{axiom:3.1}
                      Anfangspunkt $P$ und jedem $r \in \mdr_{\geq 0}$
                      gibt es genau ein $Q \in H$ mit $d(P,Q) = r$.
                \item Jede Gerade zerlegt $X \setminus g = H_1 \dcup H_2$\label{axiom:4}
                      in zwei nichtleere Teilmengen $H_1, H_2$.
                      (Diese Teilmengen heißen \textbf{Halbebenen}\xindex{Halbebene} bzgl. $g$),
                      sodass für alle $A \in H_i$, $B \in H_j$
                      $(i,j \in \Set{1,2})$ gilt: $\overline{AB} \cap g \neq \emptyset \Leftrightarrow i \neq j$
            \end{enumerate}
        \item \enquote{Bewegungsaxiome}: Zu $P, Q, P', Q' \in X$
            mit $d(P,Q) = d(P', Q')$. Isometrien $\varphi_1, \varphi_2$
            mit $\varphi_i (P) = P'$ und $\varpi_i(Q) = Q', i=1,2$
            (Spiegelung an der Gerade durch $P$ und $Q$ ist nach 
             Identifizierung von $P \cong P'$ und $Q \cong Q'$ eine
             weitere Isometrie.)
    \end{enumerate}
\end{definition}

\begin{proposition}%In Vorlesung: Satz 14.4
    Aus den Axiomen \ref{axiom:1}~-~\ref{axiom:3} folgt, dass es in 
    den Situation \ref{axiom:4} höchstens zwei Isometrien mit
    $\varphi_i(P) = P'$ und $\varphi_i(Q) = Q'$ gibt.
\end{proposition}

\begin{beweis}
    Seien $\varphi_1, \varphi_2, \varphi_3$ Isometrien mit
    $\varphi_i(P) = P'$, $\varphi_i(Q) = Q'$, $i=1,2,3$

    \begin{behauptung}
        Es gibt $R \in PQ$ mit $\varphi_{A_i} (R) = \varphi_{Z_j} (R)$
        mit $i \neq j$.

        \Obda sei $i=1$ und $j=2$, also $\varphi_1(R) = \varphi_2(R)$.
    \end{behauptung}
    \begin{behauptung}
        Hat eine Isometrie $\varphi$ 3 Fixpunkte, die nicht kolinear sind,
        so ist $\varphi = \id_X$.

        Aus Beh. 1 und Beh. 2 folgt, dass $\varphi_2^{-1} \circ \varphi_1 = \id_X$,
        also $\varphi_2 = \varphi_1$.
    \end{behauptung}

    \begin{beweis}\leavevmode
        \begin{behauptung}
            Sind $P \neq Q$ Fixpunkte einer Isometrie, so ist 
            $\varphi(R) = R$ für jedes $R \in PQ$.
        \end{behauptung}
        \begin{beweis}
            Es ist $\varphi(PQ) = \varphi(P) \varphi(Q)$ weil $\varphi$
            wegen \ref{axiom:2} kolinearität erhält.

            Sei nun $R \in PQ$. Dann ist $d(P, \varphi(R)) \stackrel{P \text{ ist Fixpunkt}}{=} d(\varphi(P), \varphi(R)) = d(P, R)$.
            Weiter ist $\varphi (PQ^+) = \varphi(P) \varphi(Q)^+ = PQ^+$
            $\stackrel{\ref{axiom:3.1}}{\Rightarrow} R = \varphi(R)$
        \end{beweis}
    \end{beweis}
\end{beweis}
