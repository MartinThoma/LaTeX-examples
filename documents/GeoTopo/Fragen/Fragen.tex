\documentclass[a5paper,oneside]{scrbook}
\usepackage{etoolbox}
\usepackage{amsmath,amssymb}% math symbols / fonts
\usepackage{mathtools}      % \xRightarrow
\usepackage{nicefrac}       % \nicefrac
\usepackage[utf8]{inputenc} % this is needed for umlauts
\usepackage[ngerman]{babel} % this is needed for umlauts
\usepackage[T1]{fontenc}    % this is needed for correct output of umlauts in pdf
\usepackage[framed,amsmath,thmmarks,hyperref]{ntheorem}
\usepackage{framed}
\usepackage{marvosym}
\usepackage{makeidx}        % for automatically generation of an index
\usepackage{xcolor}
\usepackage[bookmarks,bookmarksnumbered,hypertexnames=false,pdfpagelayout=OneColumn,colorlinks,hyperindex=false]{hyperref} % has to be after makeidx
\usepackage{enumitem}       % Better than \usepackage{enumerate}, because it allows to set references
\usepackage{tabto}
\usepackage{braket}         % needed for \Set
\usepackage{csquotes}       % \enquote{}
\usepackage{subfig}         % multiple figures in one
\usepackage{parskip}        % nicer paragraphs
\usepackage{xifthen}        % \isempty
\usepackage{changepage}     % for the adjustwidth environment
\usepackage{pst-solides3d}
\usepackage[colorinlistoftodos]{todonotes}
\usepackage{pgfplots}
\pgfplotsset{compat=1.7}
\usepackage[arrow, matrix, curve]{xy}
\usepackage{caption}        % get newlines within captions
\usepackage{tikz}           % draw
\usepackage{tikz-3dplot}    % draw
\usepackage{tkz-fct}        % draw
\usepackage{tkz-euclide}    % draw
\usetkzobj{all}             % tkz-euclide
\usetikzlibrary{3d,calc,intersections,er,arrows,positioning,shapes.misc,patterns,fadings,decorations.pathreplacing}
\usepackage{tqft}
\usepackage{xspace}   % for new commands; decides weather I want to insert a space after the command
\usepackage[german,nameinlink]{cleveref} % has to be after hyperref, ntheorem, amsthm
\usepackage[left=10mm,right=10mm, top=2mm, bottom=10mm]{geometry}
\usepackage{../shortcuts}

\hypersetup{ 
  pdfauthor   = {Martin Thoma}, 
  pdfkeywords = {Geometrie und Topologie}, 
  pdftitle    = {Fragen zu Definitionen} 
}
\allowdisplaybreaks

%%%%%%%%%%%%%%%%%%%%%%%%%%%%%%%%%%%%%%%%%%%%%%%%%%%%%%%%%%%%%%%%%%%%%
% Begin document                                                    %
%%%%%%%%%%%%%%%%%%%%%%%%%%%%%%%%%%%%%%%%%%%%%%%%%%%%%%%%%%%%%%%%%%%%%
\begin{document}
\chapter{Fragen zu Definitionen}
\section*{1.) Definition topologischer Raum}
\begin{definition}\xindex{Raum!topologischer}\xindex{Menge!offene}\xindex{Menge!abgeschlossene}%
    Ein \textbf{topologischer Raum} ist ein Paar $(X, \fT)$ bestehend
    aus einer Menge $X$ und $\fT \subseteq \powerset{X}$ mit
    folgenden Eigenschaften
    \begin{defenumprops}
        \item $\emptyset, X \in \fT$
        \item \label{def:topologie.ii} Sind $U_1, U_2 \in \fT$, so ist $U_1 \cap U_2 \in \fT$
        \item Ist $I$ eine Menge und $U_i \in \fT$ für jedes $i \in I$,
              so ist $\displaystyle \bigcup_{i \in I} U_i \in \fT$
    \end{defenumprops}
    Die Elemente von $\fT$ heißen \textbf{offene Teilmengen} von $X$. 

    $A \subseteq X$ heißt \textbf{abgeschlossen}, wenn $X \setminus A$ offen ist.
\end{definition}

Ich glaube es ist unnötig in (i) zu fordern, dass $\emptyset \in \fT$ gilt,
da man das mit (iii) bereits abdeckt:

Sei in (iii) die Indexmenge $I = \emptyset$. Dann muss gelten:

$\displaystyle \bigcup_{i \in \emptyset} U_i = \emptyset \in \fT$

\section*{4.) Knotendiagramm:}
\begin{definition}\xindex{Knotendiagramm}%
    Ein \textbf{Knotendiagramm} eines Knotens $\gamma$ ist eine 
    Projektion $\pi: \mdr^3 \rightarrow E$ auf eine Ebene $E$, sodass
    $|\pi^{-1}(x) \cap C| \leq 2$ für jedes $x \in {\color{red}D}$, wobei $C = \gamma(S^1)$.

    Ist ${\color{red}(\pi|C)}^{-1}(x) = \Set{y_1, y_2}$, so \textbf{liegt $y_1$ über $y_2$},
    wenn $(y_1-x) = \lambda (y_2 - x)$ für ein $\lambda > 1$ ist.
\end{definition}

Sollte das jeweils $\pi|_C$ (sprich: \enquote{$\pi$ eingeschränkt auf $C$})
sein?

Was ist $D$? Ich vermute, das sollte $E$ sein.

Ich würde die Definition eher so schreiben:

\begin{definition}\xindex{Knotendiagramm}%
    Sei $\gamma: [0,1] \rightarrow \mdr^3$ ein Knoten, $E$ eine Ebene und 
    $\pi: \mdr^3 \rightarrow E$ eine Projektion auf $E$.

    $\pi$ heißt \textbf{Knotendiagramm} von $\gamma$, wenn gilt:
    \[\left | \left (\pi|_{\gamma([0,1])} \right )^{-1}(x) \right | \leq 2 \;\;\; \forall x \in E\]

    Ist $(\pi|_{\gamma([0,1])})^{-1}(x) = \Set{y_1, y_2}$, so \textbf{liegt $y_1$ über $y_2$},
    wenn gilt:
    \[\exists \lambda > 1: (y_1-x) = \lambda (y_2 - x)\]
\end{definition}

Ist meine Definition äquivalent zu der aus der Vorlesung?

\section*{5.) Isotopie/Knoten}
\begin{definition}
    Zwei Knoten $\gamma_1, \gamma_2: S^1 \rightarrow \mdr^3$ heißen
    \textbf{äquivalent}, wenn es eine stetige Abbildung
    \[H: S^1 \times [0,1] \rightarrow \mdr^3\]
    gibt mit 
    \begin{align*}
        H(z,0) &= \gamma_1(z) {\;\;\;\color{red} \forall z \in S^1}\\
        H(z,1) &= \gamma_2(z) {\;\;\;\color{red} \forall z \in S^1}
    \end{align*}
    und für jedes
    feste $t \in [0,1]$ ist 
    \[H_z: S^1 \rightarrow \mdr^2, z \mapsto H(z,t)\]
    ein Knoten. Die Abbildung $H$ heißt \textbf{Isotopie} zwischen
    $\gamma_1$ und $\gamma_2$.
\end{definition}

Fehlt hier nicht etwas wie \enquote{$\forall z \in S^1$} (nun rot ergänzt).

\section*{6.) Basisbeispiele}
\begin{itemize}
    \item Kennst du ein Beispiel für eine Subbasis in einem Topologischen Raum,
die zugleich eine Basis ist?
    \item Kennst du ein Beispiel für eine Subbasis in einem Topologischen Raum,
die keine Basis ist?
    \item Kennst du ein Beispiel für eine Basis in einem Topologischen Raum,
die keine Subbasis ist?
\end{itemize}

\section*{9.) Mannigfaltigkeit mit Rand}
\begin{definition}%
    Sei $X$ ein topologischer Raum und $n \in \mdn$.
    \begin{defenum}
        \item Eine $n$-dimensionale \textbf{Karte}\xindex{Karte} auf
              $X$ ist ein Paar $(U, \varphi)$, wobei $U \subseteq X$
              offen und $\varphi: U \rightarrow V$ Homöomorphismus
              von $U$ auf eine offene Teilmenge $V \subseteq \mdr^n$.
        \item Ein $n$-dimensionaler \textbf{Atlas}\xindex{Atlas} $\atlas$ auf $X$ ist eine
              Familie $(U_i, \varphi_i)_{i \in I}$ von Karten auf $X$,
              sodass $\bigcup_{i \in I} U_i = X$.
        \item $X$ heißt (topologische) $n$-dimensionale \textbf{Mannigfaltigkeit}\xindex{Mannigfaltigkeit},
              wenn $X$ hausdorffsch ist, eine abzählbare Basis der 
              Topologie hat und ein $n$-dimensionalen Atlas besitzt.
    \end{defenum}
\end{definition}
\begin{definition}\xindex{Mannigfaltigkeit!mit Rand}%
    Sei $X$ ein Hausdorffraum mit abzählbarer Basis der Topologie.
    $X$ heißt $n$-dimensionale \textbf{Mannigfaltigkeit mit Rand},
    wenn es einen Atlas $(U_i, \varphi_i)$ gibt, wobei $U_i \subseteq X_i$
    offen und $\varphi_i$ ein Homöomorphismus auf eine offene 
    Teilmenge von 
    \[R_{+,0}^n := \Set{(x_1, \dots, x_n) \in \mdr^n | x_m \geq 0}\]
    ist.
\end{definition}

Wieso wird bei der Mannigfaltigkeit mit Rand nicht gefordert, dass
sie eine abzählbare Basis haben soll? Sollte man nicht vielleicht
hinzufügen, dass der Atlas $n$-dimensional sein soll?

\section*{11.) Produkttopologie}
\begin{definition}\xindex{Produkttopologie}%
    Seien $X_1, X_2$ topologische Räume.\\
    $U \subseteq X_1 \times X_2$ sei offen, wenn es zu jedem $x = (x_1, x_2) \in U$
    Umgebungen $U_i$ um $x_i$  mit $i=1,2$ gibt, sodass $U_1 \times U_2 \subseteq U$
    gilt.

    $\fT = \Set{U \subseteq X_1 \times X_2 | U \text{ offen}}$
    ist eine Topologie auf $X_1 \times X_2$. Sie heißt \textbf{Produkttopologie}.
    $\fB = \Set{U_1 \times U_2 | U_i \text{ offen in } X_i, i=1,2}$
    ist eine Basis von $\fT$.
\end{definition}
Gibt es ein Beispiel, das zegit, dass nicht $\fB = \fT$ gilt?

\section*{12.) $\Delta^2$ explizit}
Wie sieht der Standard-Simplex der dim. 2, also $\Delta^2$, explizit
notiert aus? Praktisch ist das ja die konvexe Hülle der Standard-Basisvektoren
$e_0, e_1, e_2$ (also $\begin{pmatrix}0\\0\\1\end{pmatrix},\begin{pmatrix}0\\1\\0\end{pmatrix},\begin{pmatrix}1\\0\\0\end{pmatrix}$),
also ein Polyeder mit vier Flächen im $\mdr^3$ (jedoch kein regelmäßiges Tetraeder, oder?)

Das ist dann nur das Gitter dieses Polyeders, aber nicht die Flächen
oder sogar etwas innerhalb vom Polyeder, oder?

\section*{13.) Normalenvektor}
\begin{definition}%In Vorlesung: Definition 16.2
    Sei $\gamma: I \rightarrow \mdr^2$ eine durch Bogenlänge
    parametrisierte Kurve.

    \begin{defenum}
        \item Für $t \in I$ sei $n(t)$ \textbf{Normalenvektor}\xindex{Normalenvektor}
              an $\gamma$ in $t$, d.~h.
              \[\langle n(t), \gamma'(t) \rangle = 0, \;\;\; \|n(t)\|=1 \]
              und $\det((\gamma_1(t), n(t))) = +1$
        \item Nach \cref{bem:16.1d} sind $n(t)$ und $\gamma''(t)$ linear
              abhängig, d.~h. es gibt $\kappa(t) \in \mdr$ mit
              \[\gamma''(t) = \kappa(t) \cdot n(t)\]
              $\kappa(t)$ heißt \textbf{Krümmung}\xindex{Krümmung}
              von $\gamma$ in $t$.
    \end{defenum}
\end{definition}

\begin{definition}%In Vorlesung: Def.+Bem. 16.4
    Sei $\gamma: I \rightarrow \mdr^3$ eine durch Bogenlänge parametrisierte
    Kurve.

    \begin{defenum}
        \item Für $t \in I$ heißt $\kappa(t) := \|\gamma''(t)\|$ die
              \textbf{Krümmung}\xindex{Krümmung} von $\gamma$ in $t$.
        \item Ist für $t \in I$ die Ableitung $\gamma''(t) \neq 0$,
              so heißt $\gamma''(t)$ \textbf{Normalenvektor}\xindex{Normalenvektor}
              an $\gamma$ in $t$.
        \item \label{def:16.4c} $b(t)$ sei ein Vektor, der $\gamma'(t), n(t)$
              zu einer orientierten Orthonormalbasis von $\mdr^3$ ergänzt.
              Also gilt:
              \[\det(\gamma'(t), n(t), b(t)) = 1\]
              $b(t)$ heißt \textbf{Binormalenvektor}\xindex{Binormalenvektor},
              die Orthonormalbasis 
              \[\Set{\gamma'(t), n(t), b(t)}\]
              heißt \textbf{begleitendes Dreibein}\xindex{Dreibein!begreitendes}.
    \end{defenum}
\end{definition}

Die beiden Definitionen eins Normalenvektors / der Krümmung scheinen mir äquivalent zu sein.
Warum haben wir beide? Ich würde die zweite bevorzugen.

\section*{14.) Dimension von Simplizes}
Gibt es 0-Dimensionale Simplizes?

\section*{15.) Existenz der Parallelen}
\begin{definition}%
    \begin{enumerate}[label=§\arabic*),ref=§\arabic*,start=5]
        \item \label{axiom:5}\textbf{Parallelenaxiom}\xindex{Parallele}:
            Für jedes $g \in G$ und jedes
            $P \in X \setminus g$ gibt es höchstens ein $h \in G$ mit
            $h \cap g = \emptyset$. $h$ heißt \textbf{Parallele zu $g$ durch $P$}.
    \end{enumerate}
\end{definition}

Soll hier wirklich \enquote{mindestens} stehen? Wie beweist man, dass es genau eine gibt?

\section{15.) Simpliziale Abbildungen}
Wenn man Simpliziale Abbildungen wie folgt definiert

\begin{definition}\xindex{Abbildung!simpliziale}%
    Seien $K, L$ Simplizialkomplexe. Eine stetige Abbildung
    \[f:|K| \rightarrow |L|\]
    heißt \textbf{simplizial}, wenn für
    jedes $\Delta \in K$ gilt:
    \begin{defenum}
        \item $f(\Delta) \in L$
        \item $f|_{\Delta} : \Delta \rightarrow f(\Delta)$ ist eine
              affine Abbildung.
    \end{defenum}
\end{definition}

dann ist die Forderung \enquote{$f(\Delta) \in L$} doch immer erfüllt, oder?
Gibt es eine Abbildung
\[f:|K| \rightarrow |L|\]
mit $f(\Delta) \notin L$?
\end{document}