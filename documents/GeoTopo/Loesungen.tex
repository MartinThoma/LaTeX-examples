\chapter*{Lösungen der Übungsaufgaben}
\addcontentsline{toc}{chapter}{Lösungen der Übungsaufgaben}
\begin{solution}[\ref{ub1:aufg1}]
    \textbf{Teilaufgabe a)} Es gilt:
    \begin{enumerate}[label=(\roman*)]
        \item $\emptyset, X \in \fT_X$.
        \item $\fT_X$ ist offensichtlich unter Durchschnitten abgeschlossen,
              d.~h. es gilt für alle $U_1, U_2 \in \fT_X: U_1 \cap U_2 \in \fT_X$.
        \item Auch unter beliebigen Vereinigungen ist $\fT_X$ abgeschlossen,
              d.~h. es gilt für eine beliebige Indexmenge $I$ und alle
              $U_i \in \fT_X$ für alle $i \in I: \bigcup_{i \in I} U_i \in \fT_X$
    \end{enumerate}

    Also ist $(X, \fT_X)$ ein topologischer Raum.

    \textbf{Teilaufgabe b)} Wähle $x=1, y=0$. Dann gilt $x \neq y$
    und die einzige Umgebung von $x$ ist $X$. Da $y=0 \in X$ können
    also $x$ und $y$ nicht durch offene Mengen getrennt werden.
    $(X, \fT_X)$ ist also nicht hausdorffsch.

    \textbf{Teilaufgabe c)} Nach Bemerkung \ref{Trennungseigenschaft}
    sind metrische Räume hausdorffsch. Da $(X, \fT_X)$ nach (b) nicht 
    hausdorffsch ist, liefert die Kontraposition der Trennungseigenschaft,
    dass $(X, \fT_X)$ kein metrischer Raum sein kann.
\end{solution}

\begin{solution}[\ref{ub1:aufg4}]
    \todo[inline]{Lösung schreiben}
\end{solution}

\begin{solution}[\ref{ub2:aufg4}]
    \begin{enumerate}[label=(\alph*)]
        \item \underline{Beh.:} Die offenen Mengen von $P$ sind
              Vereinigungen von Mengen der Form 
              \[\prod_{j \in J} U_j \times \prod_{i \in \mdn, i \neq j} P_i\]
              wobei $J \subseteq \mdn$ endlich und $U_j \subseteq P_j$
              offen ist.
              \begin{beweis}
                Nach Definition der Produkttopologie bilden Mengen
                der Form
                \[\prod_{i \in J} U_j \times \prod_{\stackrel{i \in \mdn}{i \notin J}} P_i, \text{ wobei } J \subseteq \mdn \text{ endlich und } U_j \subseteq P_j \text{offen } \forall{j \in J}\]
                eine Basis der Topologie. Damit sind die offenen 
                Mengen von $P$ Vereinigungen von Mengen der obigen
                Form. $\qed$
              \end{beweis}
        \item \underline{Beh.:} Die Zusammenhangskomponenten von $P$
              sind alle einpunktig.\xindex{Total Unzusammenhängend}
              \begin{beweis}
                Es seinen $x,y \in P$ und $x$ sowie $y$ liegen in der
                gleichen Zusammenhangskomponente $Z \subseteq P$.
                Da $Z$ zusammenhängend ist und $\forall{i \in I}: p_i : P \rightarrow P_i$
                ist stetig, ist $p_i(Z) \subseteq P_i$ zusammenhängend
                für alle $i \in \mdn$. Die zusammenhängenden Mengen
                von $P_i$ sind genau $\Set{0}$ und $\Set{1}$, d.~h.
                für alle $i \in \mdn$ gilt entweder $p_i(Z) \subseteq \Set{0}$
                oder $p_i(Z) \subseteq \Set{1}$. Es sei $z_i \in \Set{0,1}$
                so, dass $p_i(Z) \subseteq \Set{z_i}$ für alle $i \in \mdn$.
                Dann gilt also: 
                \[\underbrace{p_i(x)}_{= x_i} = z_i = \underbrace{p_i(y)}_{= y_i} \forall i \in \mdn\]
                Somit folgt: $x = y \qed$
                
              \end{beweis}
    \end{enumerate}
\end{solution}
