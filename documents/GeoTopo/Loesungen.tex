%!TEX root = GeoTopo.tex
\chapter*{Lösungen der Übungsaufgaben\markboth{Lösungen der Übungsaufgaben}{Lösungen der Übungsaufgaben}}
\addcontentsline{toc}{chapter}{Lösungen der Übungsaufgaben}
\begin{solution}[\ref{ub1:aufg1}]
    \textbf{Teilaufgabe a)} Es gilt:
    \begin{enumerate}[label=(\roman*)]
        \item $\emptyset, X \in \fT_X$.
        \item $\fT_X$ ist offensichtlich unter Durchschnitten abgeschlossen,
              d.~h. es gilt für alle $U_1, U_2 \in \fT_X: U_1 \cap U_2 \in \fT_X$.
        \item Auch unter beliebigen Vereinigungen ist $\fT_X$ abgeschlossen,
              d.~h. es gilt für eine beliebige Indexmenge $I$ und alle
              $U_i \in \fT_X$ für alle $i \in I: \bigcup_{i \in I} U_i \in \fT_X$
    \end{enumerate}

    Also ist $(X, \fT_X)$ ein topologischer Raum.

    \textbf{Teilaufgabe b)} Wähle $x=1, y=0$. Dann gilt $x \neq y$
    und die einzige Umgebung von $x$ ist $X$. Da $y=0 \in X$ können
    also $x$ und $y$ nicht durch offene Mengen getrennt werden.
    $(X, \fT_X)$ ist also nicht hausdorffsch.

    \textbf{Teilaufgabe c)} Nach Bemerkung \ref{Trennungseigenschaft}
    sind metrische Räume hausdorffsch. Da $(X, \fT_X)$ nach (b) nicht 
    hausdorffsch ist, liefert die Kontraposition der Trennungseigenschaft,
    dass $(X, \fT_X)$ kein metrischer Raum sein kann.
\end{solution}

\begin{solution}[\ref{ub1:aufg4}]
    \textbf{Teilaufgabe a)} 

    \textbf{Beh.:}  $\forall a \in \mdz: \Set{a}$ ist abgeschlossen.

    Sei $a \in \mdz$ beliebig. Dann gilt:

    Wenn jemand diese Aufgabe gemacht hat, bitte die Lösung an info@martin-thoma.de
    schicken.%TODO

    \textbf{Teilaufgabe b)} 

    \textbf{Beh.:} $\Set{-1, 1}$ ist nicht offen

    \textbf{Bew.:} durch Widerspruch

    Annahme: $\Set{-1, 1}$ ist offen.

    Dann gibt es $T \subseteq \fB$, sodass $\bigcup_{M \in T} M = \Set{-1, 1}$.
    Aber alle $U \in \fB$ haben unendlich viele Elemente. Auch endlich
    viele Schnitte von Elementen in $\fB$ haben unendlich viele
    Elemente $\Rightarrow$ keine endliche nicht-leere Menge kann
    in dieser Topologie offen sein $\Rightarrow \Set{-1,1}$ ist
    nicht offen. $\qed$

    \textbf{Teilaufgabe c)} 

    \textbf{Beh.:} Es gibt unendlich viele Primzahlen.

    \textbf{Bew.:} durch Widerspruch

    Annahme:  Es gibt nur endlich viele Primzahlen $p \in \mdp$

    Dann ist 
    \[\mdz \setminus \Set{-1, +1} \overset{\text{FS d. Arithmetik}}= \bigcup_{p \in \mdp} U_{0,p}\]
    endlich. Das ist ein Widerspruch zu $|\mdz|$ ist unendlich und
    $|\Set{-1,1}|$ ist endlich. $\qed$
\end{solution}

\begin{solution}[\ref{ub2:aufg4}]
    \begin{enumerate}[label=(\alph*)]
        \item \textbf{Beh.:} Die offenen Mengen von $P$ sind
              Vereinigungen von Mengen der Form 
              \[\prod_{j \in J} U_j \times \prod_{i \in \mdn, i \neq j} P_i\]
              wobei $J \subseteq \mdn$ endlich und $U_j \subseteq P_j$
              offen ist.
              \begin{beweis}
                Nach Definition der Produkttopologie bilden Mengen
                der Form
                \[\prod_{i \in J} U_j \times \prod_{\overset{i \in \mdn}{i \notin J}} P_i, \text{ wobei } J \subseteq \mdn \text{ endlich und } U_j \subseteq P_j \text{offen } \forall{j \in J}\]
                eine Basis der Topologie. Damit sind die offenen 
                Mengen von $P$ Vereinigungen von Mengen der obigen
                Form. $\qed$
              \end{beweis}
        \item \textbf{Beh.:} Die Zusammenhangskomponenten von $P$
              sind alle einpunktig.\xindex{Total Unzusammenhängend}
              \begin{beweis}
                Es seinen $x,y \in P$ und $x$ sowie $y$ liegen in der
                gleichen Zusammenhangskomponente $Z \subseteq P$.
                Da $Z$ zusammenhängend ist und $\forall{i \in I}: p_i : P \rightarrow P_i$
                ist stetig, ist $p_i(Z) \subseteq P_i$ zusammenhängend
                für alle $i \in \mdn$. Die zusammenhängenden Mengen
                von $P_i$ sind genau $\Set{0}$ und $\Set{1}$, d.~h.
                für alle $i \in \mdn$ gilt entweder $p_i(Z) \subseteq \Set{0}$
                oder $p_i(Z) \subseteq \Set{1}$. Es sei $z_i \in \Set{0,1}$
                so, dass $p_i(Z) \subseteq \Set{z_i}$ für alle $i \in \mdn$.
                Dann gilt also: 
                \[\underbrace{p_i(x)}_{= x_i} = z_i = \underbrace{p_i(y)}_{= y_i} \forall i \in \mdn\]
                Somit folgt: $x = y \qed$
                
              \end{beweis}
    \end{enumerate}
\end{solution}

\begin{solution}[\ref{ub3:aufg1}]
    \begin{enumerate}[label=(\alph*)]
        \item \textbf{Beh.:} $\GL_n(\mdr)$ ist nicht kompakt.\\
            \textbf{Bew.:} $\det: \GL_n(\mdr) \rightarrow \mdr \setminus \Set{0}$
                ist stetig. Außerdem ist 
                $\det(\GL_n(\mdr)) = \mdr \setminus \Set{0}$ nicht 
                kompakt. $\overset{\ref{kor:5.6}}{\Rightarrow}$ 
                $\GL_n(\mdr)$ ist nicht kompakt. $\qed$
        \item \textbf{Beh.:} $\SL_1(\mdr)$ ist nicht kompakt, für $n > 1$ ist $\SL_n(\mdr)$ kompakt.\\
            \textbf{Bew.:} Für $\SL_1(\mdr)$ gilt:
                $\SL_1(\mdr) = \Set{A \in \mdr^{1 \times 1} | \det A = 1} = \begin{pmatrix}1\end{pmatrix} \cong \Set{1}$.
                $\overset{\ref{kor:5.6}}{\Rightarrow} \SL_1(\mdr)$ ist
                kompakt.\\

                $\SL_n(\mdr) \subseteq \GL_n(\mdr)$ lässt sich mit einer
                Teilmenge des $\mdr^{n^2}$ identifizieren. Nach \cref{satz:heine-borel}
                sind diese genau dann kompakt, wenn sie beschränkt und 
                abgeschlossen sind. Definiere nun für für $n \in \mdn_{\geq 2}, m \in \mdn$:
                \[A_m = \text{diag}_n(m, \frac{1}{m}, \dots, 1)\]
                Dann gilt: $\det A_m = 1$, d.~h. $A_m \in \SL_n(\mdr)$,
                und $A_m$ ist unbeschränkt, da $\|A_m\|_\infty =m \xrightarrow[m \rightarrow \infty]{} \infty$.$\qed$
        \item \textbf{Beh.:} $\praum(\mdr)$ ist kompakt.\\
            \textbf{Bew.:} $\praum(\mdr) \cong S^n/_{x \sim -x}$.
                Per Definition der Quotiententopologie ist die Klassenabbildung stetig.
                Da $S^n$ als abgeschlossene und beschränkte Teilmenge
                des $\mdr^{n+1}$ kompakt ist $\overset{\ref{kor:5.6}}{\Rightarrow}$
                $\praum(\mdr)$ ist kompakt. $\qed$
    \end{enumerate}
\end{solution}

\begin{solution}[\ref{ub3:meinsExtra}]
    Die Definition von Homöomorphismus kann auf \cpageref{def:homoeomorphismus}
    nachgelesen werden.

    \begin{definition}\xindex{Homomorphismus}%
        Seien $(G, *)$ und $(H, \circ)$ Gruppen und 
        $\varphi:G \rightarrow H$ eine Abbildung.

        $\varphi$ heißt \textbf{Homomorphismus}, wenn
        \[\forall g_1, g_2 \in G: \varphi(g_1 * g_2) = \varphi(g_1) \circ \varphi(g_2)\]
        gilt.
    \end{definition}

    Es folgt direkt:
    \begin{bspenum}
        \item Sei $X = \mdr$ mit der Standarttopologie und $\varphi_1: \id_\mdr$ und $\mdr = (\mdr,+)$. Dann ist $\varphi_1$ ein Gruppenhomomorphismus und ein Homöomorphismus.
        \item Sei $G = (\mdz, +)$ und $H = (\mdz / 3 \mdz, +)$. Dann ist $\varphi_2 : G \rightarrow H, x \mapsto x \mod 3$ ein Gruppenhomomorphismus.
              Jedoch ist $\varphi_2$ nicht injektiv, also sicher kein Homöomorphismus.
        \item Sei $X$ ein topologischer Raum. Dann ist $\id_X$ ein Homöomorphismus. Da keine Verknüpfung auf $X$ definiert wurde, ist $X$ keine Gruppe und daher auch kein Gruppenhomomorphismus.
    \end{bspenum}

    Also: Obwohl die Begriffe ähnlich klingen, werden sie in ganz unterschiedlichen
    Kontexten verwendet.
\end{solution}

\begin{solution}[\ref{ub3:meinsExtra2}]
    Die Definition einer Isotopie kann auf \cpageref{def:Isotopie} nachgelesen
    werden, die einer Isometrie auf \cpageref{def:Isometrie}.
    
    \begin{definition}\xindex{Isomorphismus}%
        Seien $(G, *)$ und $(H, \circ)$ Gruppen und 
        $\varphi:G \rightarrow H$ eine Abbildung.

        $\varphi$ heißt \textbf{Isomorphismus}, wenn $\varphi$ ein bijektiver
        Homomorphismus ist.
    \end{definition}

    Eine Isotopie ist also für Knoten definiert, Isometrien machen nur in 
    metrischen Räumen Sinn und ein Isomorphismus benötigt eine Gruppenstruktur.
\end{solution}

\begin{solution}[\ref{ub4:aufg1}]
    \begin{enumerate}[label=(\alph*)]
        \item \textbf{Vor.:} Sei $M$ eine topologische Mannigfaltigkeit.\\
              \textbf{Beh.:} $M$ ist wegzusammehängend $\gdw M$ ist zusammenhängend
              \begin{beweis}
                \enquote{$\Rightarrow$}: Da $M$ insbesondere ein
                topologischer Raum ist folgt diese Richtung direkt 
                aus \cref{kor:wegzusammehang-impliziert-zusammenhang}.

                \enquote{$\Leftarrow$}: Seien $x,y \in M$ und
                \[Z := \Set{z \in M | \exists \text{Weg von } x \text{ nach } z}\]
                Es gilt:
                \begin{enumerate}[label=(\roman*)]
                    \item $Z \neq \emptyset$, da $M$ lokal wegzusammenhängend ist
                    \item $Z$ ist offen, da $M$ lokal wegzusammenhängend ist
                    \item $Z^C := \Set{\tilde{z} \in M | \nexists \text{Weg von } x \text{ nach } \tilde{z}}$ ist offen

                    Da $M$ eine Mannigfaltigkeit ist, existiert zu jedem
                    $\tilde{z} \in Z^C$ eine offene und wegzusammenhängende Umgebung 
                    $U_{\tilde{z}} \subseteq M$.

                    Es gilt sogar $U_{\tilde{z}} \subseteq Z^C$, denn
                    gäbe es ein $U_{\tilde{z}} \ni \overline{z} \in Z$,
                    so gäbe es Wege $\gamma_2:[0,1] \rightarrow M, \gamma_2(0) = \overline{z}, \gamma_2(1) = x$
                    und $\gamma_1:[0,1] \rightarrow M, \gamma_1(0) = \tilde{z}, \gamma_1(1) = \overline{z}$.
                    Dann wäre aber
                    \begin{align*}
                        \gamma:[0,1] &\rightarrow M,\\
                        \gamma(x) &= \begin{cases}
                            \gamma_1(2x)   &\text{falls } 0 \leq x \leq \frac{1}{2}\\
                            \gamma_2(2x-1) &\text{falls } \frac{1}{2} < x \leq 1
                            \end{cases}
                    \end{align*}
                    ein stetiger Weg von $\tilde{z}$ nach $x$
                    $\Rightarrow$ Widerspruch.

                    Da $M$ zusammenhängend ist und $M = \underbrace{Z}_{\mathclap{\text{offen}}} \cup \underbrace{Z^C}_{\mathclap{\text{offen}}}$,
                    sowie $Z \neq \emptyset$ folgt $Z^C = \emptyset$.
                    Also ist $M=Z$ wegzusammenhängend.$\qed$
                \end{enumerate}
              \end{beweis}
        \item \textbf{Beh.:} $X$ ist wegzusammenhängend.\\
            \begin{beweis}
                $X:= (\mdr \setminus \Set{0}) \cup \Set{0_1, 0_2}$
                und $(\mdr \setminus \Set{0}) \cup \Set{0_2}$ sind
                homöomorph zu $\mdr$. Also sind die einzigen kritischen
                Punkte, die man nicht verbinden können könnte
                $0_1$ und $0_2$.

                Da $(\mdr \setminus \Set{0}) \cup \Set{0_1}$ homöomorph
                zu $\mdr$ ist, exisitert ein Weg $\gamma_1$ von $0_1$
                zu einem beliebigen Punkt $a \in \mdr \setminus \Set{0}$.
                
                Da $(\mdr \setminus \Set{0}) \cup \Set{0_2}$ ebenfalls
                homöomorph zu $\mdr$ ist, existiert außerdem ein Weg
                $\gamma_2$ von $a$ nach $0_2$. Damit existiert ein
                (nicht einfacher)
                Weg $\gamma$ von $0_1$ nach $0_2$. $\qed$
            \end{beweis}
    \end{enumerate}
\end{solution}

%Das scheint mir etwas zu lang zu sein...
%\begin{solution}[\ref{ub7:aufg1}]
%    \textbf{Beh.:} $H_k = \begin{cases}\mdr &\text{für } k\in \Set{0,1}\\
%                                       0    &\text{für } k \geq 2$
%    \newcommand{\triangleSimplizialkomplex}{\mathord{\includegraphics[height=5ex]{figures/triangleSimplizialkomplex.pdf}}}
%    \textbf{Bew.:} $S^1$ ist homöomorph zum Simplizialkomplex 
%                   $X = \triangleSimplizialkomplex$, d.~h. dem Rand
%                   von $\Delta^2$. Es gilt:
%            \[X = \Set{\underbrace{v_0, v_1, v_2}_{A_0(X)}, \underbrace{\Delta (v_1, v_2)}_{=: a_0}, \underbrace{\underbrace{\Delta (v_0, v_2)}_{=: a_1}, \underbrace{\Delta(v_0, v_1)}_{=: a_2}}_{A_1(X)}}\]
%            Damit folgt: 
%        \begin{enumerate}
%            \item Für $k \geq 2$ ist $C_k(X) \cong 0$, da es in diesen 
%                Dimensionen keine Simplizes gibt, d.~h. $A_k(X) = \emptyset$ gilt.\\
%                Also: $H_k(X) \cong 0 \; \forall k \geq 2$
%            \item $C_0(X) = \Set{\sum_{i=0}^2 c_i v_i | c_i \in \mdr}$, da 
%                $A_0(x)$ Basis von $C_0(X)$ ist;\\
%                $C_1(X) = \Set{\sum_{i=0}^2 c_i a_i | c_i \in \mdr}$, da 
%                $A_1(X)$ Basis von $C_1(X)$ ist.
%            \item Für die Randabbildungen $d_i: C_i(X) \rightarrow C_{i-1}(X)$ gilt:
%                $d_0 \equiv 0$, $d_1: C_1(X) \rightarrow C_0(X)$ ist definiert durch
%                $d_1(a_k) = \sum_{i=0}^1 (-1)^i \partial_i(a_k) = \partial_0 (a_k) - \partial_1(a_k) \; \forall k \in \Set{0,1,2}$
%        \end{enumerate}
%\end{solution}

%Auch diese Aufgabe ist zu lang
%\begin{solution}[\ref{ub7:aufg3}]
%
%\end{solution}

\begin{solution}[\ref{ub11:aufg3}]
    \textbf{Vor.:} Sei $(X, d)$ eine absolute Ebene, $A, B, C \in X$
        und $\triangle ABC$ ein Dreieck.

    \begin{enumerate}[label=(\alph*)]
        \item \textbf{Beh.:} $\overline{AB} \cong \overline{AC} \Rightarrow \angle ABC \cong \angle ACB$\\
              \textbf{Bew.:} Sei $\overline{AB} \cong \overline{AC}$.\\
              $\Rightarrow \exists$ Isometrie $\varphi$ mit $\varphi(B) = C$ und
              $\varphi(C) = B$ und $\varphi(A) = A$.\\
              $\Rightarrow \varphi(\angle ABC) = \angle ACB$\\
              $\Rightarrow \angle ABC \cong \angle ACB \qed$
        \item \textbf{Beh.:} Der längeren Seite von $\triangle ABC$ liegt der größere Winkel gegenüber und
              umgekehrt.\\
              \textbf{Bew.:} Sei $d(A,C) > d(A,B)$. Nach \ref{axiom:3.1}
              gibt es $C' \in AC^+$ mit $d(A, C') = d(A,B)$\\
              $\Rightarrow C'$ liegt zwischen $A$ und $C$.\\
              Es gilt $\measuredangle ABC' < \measuredangle ABC$ und
              aus \cref{ub11:aufg3.a} folgt: $\measuredangle ABC' = \measuredangle AC' B$.\\
              $\angle BC' A$ ist ein nicht anliegender Außenwinkel zu
              $\angle BCA \xRightarrow{\crefabbr{bem:14.9}} \measuredangle BC' A > \measuredangle BCA$\\
              $\Rightarrow \measuredangle BCA < \measuredangle BC' A = \measuredangle ABC' < \measuredangle ABC $
              Sei umgekehrt $\measuredangle ABC > \measuredangle BCA$,
              kann wegen 1. Teil von \cref{ub11:aufg3.b} nicht 
              $d(A,B) > d(A,C)$ gelten.\\
              Wegen \cref{ub11:aufg3.a} kann nicht $d(A,B) = d(A,C)$
              gelten.\\
              $\Rightarrow d(A,B) < d(A, C) \qed$
        \item \textbf{Vor.:} Sei $g$ eine Gerade, $P \in X$ und $P \notin g$\\
              \textbf{Beh.:} $\exists!$ Lot\\
              \textbf{Bew.:} ÜB10 A4(a): Es gibt Geradenspiegelung $\varphi$
              an $g$. $\varphi$ vertauscht die beiden Halbebenen bzgl.
              $g$.\\
              $\Rightarrow \varphi(P)P$ schneidet $g$ in $F$.

              %Nach ÜB 10 A4(a):
              Es gibt eine Geradenspiegelung $\varphi$ an $g$.
              $\varphi$ vertauscht die beiden Halbebenen bzgl. $g$
              $\Rightarrow \varphi(P)P$ schneidet $g$ in $F$.

              Sei $A \in g \setminus \Set{F}$. Dann gilt $\varphi(\angle AFP) = \angle AF \varphi(P) = \pi$
              $\Rightarrow \angle AFP$ ist rechter Winkel.

              Gäbe es nun $G \in g \setminus \Set{F}$, so dass $PG$ weiteres Lot von $P$ auf $g$ ist,
              wäre $\triangle PFG$ ein Dreieck mit zwei rechten Innenwinkeln (vgl. \cref{fig:two-perpendiculars}).

              \begin{figure}[htp]
                  \centering
                  \begin{tikzpicture}
    \tkzSetUpPoint[shape=circle,size=10,color=black,fill=black]
    \tkzSetUpLine[line width=1]
    \tkzDefPoints{0/3/A, 4/0/B, 3/3/P, 3/0.75/G}
    \tkzDefLine[perpendicular=through P,/tikz/overlay](A,B)\tkzGetPoint{x}
    \tkzInterLL(A,B)(P,x) \tkzGetPoint{F}

    \tkzMarkAngle[arc=l,size=0.4cm,color=red,fill=red!20](B,F,P)
    \tkzLabelAngle[pos = 0.2](B,F,P){$\cdot$}

    \tkzMarkAngle[arc=l,size=0.4cm,color=red,fill=red!20](P,G,A)
    \tkzLabelAngle[pos = 0.2](P,G,A){$\cdot$}

    \tkzDrawPoints(A,F,P,G)


    \tkzDrawSegments(A,B)
    \tkzDrawLines(A,B)
    \tkzDrawLine[dashed,color=orange,add=0.5 and 0.2](F,P)
    \tkzDrawLine[dashed,color=blue,add=0.5 and 0.2](G,P)
    %
    \tkzLabelPoint[below left](A){$A$}
    \tkzLabelPoint[below left](G){$G$}
    \tkzLabelPoint[above left](P){$P$}
    \tkzLabelPoint[left](F){$F$}
    \tkzLabelLine[below,pos=1](A,B){$g$}
\end{tikzpicture}
                  \caption{Zwei Lote zu einer Geraden $g$ durch einen Punkt $P$}
                  \label{fig:two-perpendiculars}
              \end{figure}

              Nach \cref{folgerung:14.10} ist die Summe von zwei Innenwinkeln immer $< \pi$\\
              $\Rightarrow G$ gibt es nicht. $\qed$
    \end{enumerate}
\end{solution}

\begin{solution}[\ref{ub-tut-24:a1}]
    Sei $f \parallel h$ und \obda $f \parallel g$.

    $f \nparallel h \Rightarrow f \cap h \neq \emptyset$, sei also $x \in f \cap h$.
    Mit Axiom \ref{axiom:5} folgt: Es gibt höchstens eine Parallele
    zu $g$ durch $x$, da $x \notin g$. Diese ist $f$, da $x \in f$
    und $f \parallel g$. Da aber $x \in h$, kann $h$ nicht parallel
    zu $g$ sein, denn ansonsten gäbe es zwei Parallelen zu $g$ durch
    $x$ ($f \neq h$).
    $\Rightarrow g \nparallel h$ $\qed$
\end{solution}

\begin{solution}[\ref{ub-tut-24:a3}]\xindex{Kongruenzsatz!SSS}%
    Sei $(X,d,G)$ eine Geometrie, die \ref{axiom:1}-\ref{axiom:4} erfüllt.
    Seien außerdem $\triangle ABC$ und $\triangle A'B' C'$ Dreiecke, für die gilt:
    \begin{align*}
        d(A, B)  &= d(A', B')\\
        d(A, C)  &= d(A', C')\\
        d(B, C)  &= d(B', C')
    \end{align*}

    Sei $\varphi$ die Isometrie mit $\varphi(A) = A'$, $\varphi(B) = B'$ und 
    $\varphi(C')$ liegt in der selben Halbebene bzgl. $AB$ wie $C$. Diese
    Isometrie existiert wegen \ref{axiom:4}.

    Es gilt $d(A,C) = d(A', C') = d(\varphi(A'), \varphi(C')) = d(A, \varphi(C'))$
    und $d(B,C) = d(B', C') = d(\varphi(B'), \varphi(C')) = d(B, \varphi(C'))$.\\
    $\xRightarrow{\crefabbr{kor:14.6}} C = \varphi(C)$.

    Es gilt also $\varphi(\triangle A'B'C') = \triangle ABC$. $\qed$
\end{solution}