%%%%%%%%%%%%%%%%%%%%%%%%%%%%%%%%%%%%%%%%%%%%%%%%%%%%%%%%%%%%%%%%%%%%%%
% Begriffslexikon zur Beschreibung des Produkts						 %
%%%%%%%%%%%%%%%%%%%%%%%%%%%%%%%%%%%%%%%%%%%%%%%%%%%%%%%%%%%%%%%%%%%%%%
%\newglossaryentry{sortierschluessel}
%{
%  name=Sortierschlüssel,
%  description={ein Schlüssel, anhand dessen diese Einträge sortiert werden}
%}
%\newacronym{abc}{Blub}{Bananarama}

%%%%%%%%%%%%%%%%%%%%%%%%%%%%%%%%%%%%%%%%%%%%%%%%%%%%%%%%%%%%%%%%%%%%%
% Mengenoperationen                                                 %
%%%%%%%%%%%%%%%%%%%%%%%%%%%%%%%%%%%%%%%%%%%%%%%%%%%%%%%%%%%%%%%%%%%%%
\newglossaryentry{Abschluss}
{
  name={\ensuremath{\overline{M}}},
  description={Abschluss der Menge $M$},
  sort=MengenoperationFAbschluss
}

\newglossaryentry{Rand}
{
  name={\ensuremath{\partial M}},
  description={Rand der Menge $M$},
  sort=MengenoperationFRand
}

\newglossaryentry{Inneres}
{
  name={\ensuremath{M^\circ}},
  description={Inneres der Menge $M$},
  sort=MengenoperationFInneres
}

\newglossaryentry{Kreuzprodukt}
{
  name={\ensuremath{A \times B}},
  description={Kreuzprodukt zweier Mengen},
  sort=MengenoperationNKreuzprodukt
}
\newglossaryentry{subseteq}
{
  name={\ensuremath{A \subseteq B}},
  description={Teilmengenbeziehung},
  sort=MengenoperationNSubseteq
}
\newglossaryentry{subsetneq}
{
  name={\ensuremath{A \subsetneq B}},
  description={echte Teilmengenbeziehung},
  sort=MengenoperationNSubsetneq
}

%%%%%%%%%%%%%%%%%%%%%%%%%%%%%%%%%%%%%%%%%%%%%%%%%%%%%%%%%%%%%%%%%%%%%
% Zahlenmengen                                                      %
%%%%%%%%%%%%%%%%%%%%%%%%%%%%%%%%%%%%%%%%%%%%%%%%%%%%%%%%%%%%%%%%%%%%%

\newglossaryentry{Z}
{
  name={\ensuremath{\mdz}},
  description={Ganze Zahlen},
  sort=KoerperAZ
}

\newglossaryentry{Q}
{
  name={\ensuremath{\mdq}},
  description={Rationale Zahlen},
  sort=KoerperBQ
}

\newglossaryentry{R}
{
  name={\ensuremath{\mdr}},
  description={Reele Zahlen},
  sort=KoerperR
}

\newglossaryentry{Rplus}
{
  name={\ensuremath{\mdr^+}},
  description={Echt positive reele Zahlen},
  sort=KoerperRplus
}

\newglossaryentry{Einheitengruppe}
{
  name={\ensuremath{\mdr^\times}},
  description={Multiplikative Einheitengruppe von $\mdr$},
  sort=KoerperREinheiten
}

%%%%%%%%%%%%%%%%%%%%%%%%%%%%%%%%%%%%%%%%%%%%%%%%%%%%%%%%%%%%%%%%%%%%%
% Fraktale Symbole                                                  %
%%%%%%%%%%%%%%%%%%%%%%%%%%%%%%%%%%%%%%%%%%%%%%%%%%%%%%%%%%%%%%%%%%%%%
\newglossaryentry{fB}
{
  name={\ensuremath{\fB}},
  description={Basis einer Topologie},
  sort=fB
}

\newglossaryentry{fT}
{
  name={\ensuremath{\fT}},
  description={Topologie},
  sort=fT
}


% Setze den richtigen Namen für das Glossar
\renewcommand*{\glossaryname}{\glossarName}
\deftranslation{Glossary}{\glossarName}

% Drucke das gesamte Glossar
\glsaddall
\printglossaries

% Trage das Glossar in das Inhaltsverzeichnis ein
%\stepcounter{section}
%\addcontentsline{toc}{section}{\numberline {\thesection} \glossarName}
