\documentclass[mycards,frame]{flashcards}
\usepackage{amsmath,amssymb}% math symbols / fonts
\usepackage[utf8]{inputenc} % this is needed for umlauts
\usepackage[ngerman]{babel} % this is needed for umlauts
\usepackage[T1]{fontenc}    % this is needed for correct output of umlauts in pdf
\usepackage{ntheorem}
\newcommand{\thmfoot}{}
\theoremstyle{break}
\setlength\theoremindent{0.7cm}
\theoremheaderfont{\kern-0.7cm\normalfont\bfseries} 
\theorembodyfont{\normalfont} % nicht mehr kursiv
\theoremseparator{\thmfoot}
\newtheorem{definition}{Definition}

\begin{document}

\begin{flashcard}{Jordankurve}
\begin{definition}
    Sei $X$ ein topologischer Raum. Eine (geschlossene)
    \textbf{Jordankurve} in $X$ ist ein Homöomorphismus 
    $\gamma: [0, 1] \rightarrow C \subseteq X$
    ($\gamma: S^1 \rightarrow C \subseteq X$)
\end{definition}
\end{flashcard}

\begin{flashcard}{Knoten}
\begin{definition}
    Eine geschlossene Jordankurve in $r^3$ heißt \textbf{Knoten}.
\end{definition}
\end{flashcard}

\begin{flashcard}{äquivalente Knoten}
\begin{definition}
    Zwei Knoten $\gamma_1, \gamma_2: S^1 \rightarrow r^3$ heißen
    \textbf{äquivalent}, wenn es eine stetige Abbildung
    \[H: S^1 \times [0,1] \Rightarrow r^3\]
    gibt mit 
    \begin{align*}
        H(z,0) &= \gamma_1(z)\\
        H(z,1) &= \gamma_2(z)
    \end{align*}
    und für jedes
    feste $t \in [0,1]$ ist 
    \[H_z: S^1 \rightarrow r^2, z \mapsto H(z,t)\]
    ein Knoten. Die Abbildung $H$ heißt \textbf{Isotopie} zwischen
    $\gamma_1$ und $\gamma_2$.
\end{definition}
\end{flashcard}
\end{document}
