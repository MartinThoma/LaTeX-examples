\subsection{Grundlagen}
\begin{frame}{Graph}
\begin{block}{Graph}
Ein Graph ist ein Tupel $(E, K)$, wobei $E \neq \emptyset$ die Eckenmenge und 
$K \subseteq E \times E$ die 
Kantenmenge bezeichnet.
\end{block}
\pause
\tikzstyle{vertex}=[draw,fill=black,circle,minimum size=10pt,inner sep=0pt]

\begin{gallery}
    \galleryimage[Green]{graphs/graph-1}
    \galleryimage[Green]{graphs/graph-2}
    \galleryimage[Green]{graphs/k-3-3}
    \galleryimage[Green]{graphs/k-5}\\
    \galleryimage[Green]{graphs/k-16}
    \galleryimage[Green]{graphs/graph-6}
    \galleryimage[Green]{graphs/star-graph}
    \galleryimage[Green]{graphs/tree}
\end{gallery}
\end{frame}

\begin{frame}{Synonyme}

\begin{center}
\Huge{Knoten $\Leftrightarrow$ Ecken}
\end{center}

\end{frame}

\begin{frame}{Isomorphe Graphen}
\begin{center}
\href{http://www.martin-thoma.de/uni/graph.html}{martin-thoma.de/uni/graph.html}
\end{center}
\end{frame}

\begin{frame}{Inzidenz}
\begin{block}{Inzidenz}
Sei $e \in E$ und $k = \Set{e_1, e_2} \in K$.

$e$ heißt \textbf{inzident} zu $k :\Leftrightarrow e = e_1$ oder $e = e_2$
\end{block}

\pause
\tikzstyle{vertex}=[draw,fill=black,circle,minimum size=10pt,inner sep=0pt]

\begin{gallery}
    \galleryimage[Green]{inzidenz/graph-1}
    \galleryimage[Green]{inzidenz/graph-2}
    \galleryimage[Green]{inzidenz/k-3-3}
    \galleryimage[Green]{inzidenz/k-5}\\
    \galleryimage[Green]{inzidenz/k-16}
    \galleryimage[red]{inzidenz/graph-6}
    \galleryimage[Green]{inzidenz/star-graph}
    \galleryimage[Green]{inzidenz/tree}
\end{gallery}
\end{frame}
