\subsection{Grundlagen}
\begin{frame}{Graph}
\begin{block}{Graph}
Ein Graph ist ein Tupel $(V, E)$, wobei $V \neq \emptyset$ die Knotenmenge und 
$E \subseteq V \times V$ die 
Kantenmenge bezeichnet.
\end{block}
\pause
\tikzstyle{vertex}=[draw,fill=black,circle,minimum size=10pt,inner sep=0pt]

\begin{gallery}
    \galleryimage{graphs/graph-1}
    \galleryimage{graphs/graph-2}
    \galleryimage{graphs/k-3-3}
    \galleryimage{graphs/k-5}\\
    \galleryimage{graphs/k-16}
    \galleryimage{graphs/graph-6}
    \galleryimage{graphs/star-graph}
    \galleryimage{graphs/tree}
\end{gallery}
\end{frame}

\begin{frame}{Synonyme}

\begin{center}
\Huge{Knoten $\Leftrightarrow$ Ecken}
\end{center}

\end{frame}

\begin{frame}{Inzidenz}
\begin{block}{Inzidenz}
Sei $v \in V$ und $e = \Set{v_1, v_2} \in E$.

$v$ heißt \textbf{inzident} zu $e :\Leftrightarrow v = v_1$ oder $v = v_2$
\end{block}

\pause
\tikzstyle{vertex}=[draw,fill=black,circle,minimum size=10pt,inner sep=0pt]

\begin{gallery}
    \galleryimage{inzidenz/graph-1}
    \galleryimage{inzidenz/graph-2}
    \galleryimage{inzidenz/k-3-3}
    \galleryimage{inzidenz/k-5}\\
    \galleryimage{inzidenz/k-16}
    \galleryimage{inzidenz/graph-6}
    \galleryimage{inzidenz/star-graph}
    \galleryimage{inzidenz/tree}
\end{gallery}
\end{frame}

\begin{frame}{Vollständige Graphen}
\begin{block}{Vollständiger Graph}
Sei $G = (V, E)$ ein Graph.

$G$ heißt \textbf{vollständig} $:\Leftrightarrow E = V \times V \setminus \Set{v \in V: \Set{v, v}}$
\end{block}

Ein vollständiger Graph mit $n$ Knoten wird als $K_n$ bezeichnet.
\pause
\tikzstyle{vertex}=[draw,fill=black,circle,minimum size=10pt,inner sep=0pt]
\begin{gallery}
    \galleryimage{vollstaendig/k-1}
    \galleryimage{vollstaendig/k-2}
    \galleryimage{vollstaendig/k-3}
    \galleryimage{vollstaendig/k-4}\\
    \galleryimage{vollstaendig/k-5}
    \galleryimage{vollstaendig/k-6}
    \galleryimage{vollstaendig/k-7}
    \galleryimage{vollstaendig/k-16}
\end{gallery}
\end{frame}

\begin{frame}{Bipartite Graphen}
\begin{block}{Bipartite Graph}
Sei $G = (V, E)$ ein Graph und $A, B \subset V$ zwei disjunkte Knotenmengen mit
$V \setminus A = B$.

$G$ heißt \textbf{bipartit} $:\Leftrightarrow \forall_{e = \Set{v_1, v_2} \in E}: (v_1 \in A \text{ und } v_2 \in B) \text{ oder } (v_1 \in B \text{ und } v_2 \in A) $
\end{block}

TODO: 8 Bilder von Graphen
\end{frame}

\begin{frame}{Vollständig bipartite Graphen}
\begin{block}{Vollständig bipartite Graphen}
Sei $G = (V, E)$ ein bipartiter Graph und $\Set{A, B}$ bezeichne die Bipartition.

$G$ heißt \textbf{vollständig bipartit} $:\Leftrightarrow \forall_{a \in A} \forall_{b \in B}: \Set{a, b} \in E$
\end{block}

TODO: 8 Bilder von Graphen
\end{frame}

\begin{frame}{Vollständig bipartite Graphen}
Bezeichnung: Vollständig bipartite Graphen mit der Bipartition $\Set{A, B}$ 
bezeichnet man mit $K_{|A|, |B|}$.

TODO: $K_{2,2}$
TODO: $K_{2,3}$
TODO: $K_{3,3}$
\end{frame}

\begin{frame}{Kantenzug}
\begin{block}{Kantenzug}
Sei $G = (V, E)$ ein Graph.

Dann heißt eine Folge $e_1, e_2, \dots, e_s$ von Kanten, zu denen es Knoten
$v_0, v_1, v_2, \dots, v_s$ gibt, so dass
\begin{itemize}
    \item $e_1 = \Set{v_0, v_1}$
    \item $e_2 = \Set{v_1, v_2}$
    \item \dots
    \item $e_s = \Set{v_{s-1}, v_s}$
\end{itemize}
gilt ein \textbf{Kantenzug}, der $v_0$ und $v_s$ \textbf{verbindet} und $s$ 
seine \textbf{Länge}.
\end{block}

TODO: 8 Bilder
\end{frame}

\begin{frame}{Geschlossener Kantenzug}
\begin{block}{Geschlossener Kantenzug}
Sei $G = (V, E)$ ein Graph und $A = (e_1, e_2 \dots, e_s)$ ein Kantenzug.

A heißt \textbf{geschlossen} $:\Leftrightarrow v_s = v_0$ .
\end{block}

TODO: 8 Bilder
\end{frame}

\begin{frame}{Weg}
\begin{block}{Weg}
Sei $G = (V, E)$ ein Graph und $A = (e_1, e_2 \dots, e_s)$ ein Kantenzug.

A heißt \textbf{Weg} $:\Leftrightarrow \forall_{i, j \in [1, s] \cap \mathbb{N}}: i \neq j \Rightarrow e_i \neq e_j$ .
\end{block}

TODO: 8 Bilder
\end{frame}

\begin{frame}{Kreis}
\begin{block}{Kreis}
Sei $G = (V, E)$ ein Graph und $A = (e_1, e_2 \dots, e_s)$ ein Kantenzug.

A heißt \textbf{Kreis} $:\Leftrightarrow A$ ist geschlossen und ein Weg.
\end{block}

TODO: 8 Bilder
\end{frame}

\begin{frame}{Zusammenhängender Graph}
\begin{block}{Zusammenhängender Graph}
Sei $G = (V, E)$ ein Graph.

$G$ heißt \textbf{zusammenhängend} $:\Leftrightarrow \forall v_1, v_2 \in V: $ Es ex. ein Kantenzug, der $v_1$ und $v_2$ verbindet
\end{block}

TODO: 8 Bilder
\end{frame}

\begin{frame}{Grad eines Knotens}
\begin{block}{Grad eines Knotens}
Der \textbf{Grad} eines Knotens ist die Anzahl der Kanten, die von diesem Knoten
ausgehen.
\end{block}

\begin{block}{Isolierte Knoten}
Hat ein Knoten den Grad 0, so nennt man ihn \textbf{isoliert}.
\end{block}

TODO: 8 Bilder
\end{frame}
