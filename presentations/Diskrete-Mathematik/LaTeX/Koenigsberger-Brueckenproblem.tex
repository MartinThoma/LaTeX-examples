\subsection{Königsberger Brückenproblem}

\framedgraphic{Königsberg heute}{../images/koenigsberg-bruecken-luftbild}

\framedgraphic{Königsberger Brückenproblem}{../images/Konigsberg_bridges.png}

\framedgraphic{Übersetzung in einen Graphen}{../images/Konigsberg_bridges-graph.png}

\begin{frame}{Übersetzung in einen Graphen}
\begin{center}
\adjustbox{max size={\textwidth}{0.8\textheight}}{
\input{koenigsberg/koenigsberg-1}
}
\end{center}
\end{frame}

\begin{frame}{Eulerscher Kreis}
\begin{block}{Eulerscher Kreis}
Sei $G$ ein Graph und $A$ ein Kreis in $G$.

$A$ heißt \textbf{eulerscher Kreis} $:\Leftrightarrow \forall_{e \in E}: e \in A$.
\end{block}

\begin{block}{Eulerscher Graph}
Ein Graph heißt \textbf{eulersch}, wenn er einen eulerschen Kreis enthält.
\end{block}
\end{frame}

\begin{frame}{Eulerscher Kreis}
TODO: $K_5$ eulerkreis animieren
\end{frame}

\begin{frame}{Satz von Euler}
\begin{block}{Satz von Euler}
Wenn ein Graph $G$ eulersch ist, dann hat jeder Knoten von $G$ geraden Grad.
\end{block}

Wenn $G$ einen Knoten mit ungeraden Grad hat, ist $G$ nicht eulersch.
\end{frame}

\begin{frame}{Umkehrung des Satzes von Euler}
\begin{block}{Umkehrung des Satzes von Euler}
Wenn in einem zusammenhängenden Graphen $G$ jeder Knoten geraden Grad hat, dann 
ist $G$ eulersch.
\end{block}

Beweis per Induktion

TODO
\end{frame}

\begin{frame}{Offene eulersche Linie}
\begin{block}{Offene eulersche Linie}
Sei $G$ ein Graph und $A$ ein Weg, der kein Kreis ist.

$A$ heißt \textbf{offene eulersche Linie} von $G :\Leftrightarrow$ Jede Kante in $G$ kommt genau ein mal in $A$ vor.
\end{block}

Ein Graph kann genau dann "`in einem Zug"' gezeichnet werden, wenn er eine 
offene eulersche Linie besitzt.
\end{frame}

\begin{frame}{Offene eulersche Linie}
\begin{block}{Satz 8.2.3}
Sei $G$ ein zusammenhängender Graph.

$G$ hat eine offene eulersche Linie $:\Leftrightarrow G$ hat genau zwei Ecken 
ungeraden Grades.
\end{block}

TODO: Haus des Nikolaus-Animation.
TODO: Beweis
\end{frame}
