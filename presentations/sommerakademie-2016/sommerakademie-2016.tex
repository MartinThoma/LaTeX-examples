\documentclass{beamer}
\usetheme{Frankfurt}
\usecolortheme{default}
\usepackage{hyperref}
\usepackage[utf8]{inputenc} % this is needed for german umlauts
\usepackage[english]{babel} % this is needed for german umlauts
\usepackage[T1]{fontenc}    % this is needed for correct output of umlauts in pdf
\usepackage{booktabs}
\usepackage{csquotes}
\usepackage{siunitx}

\begin{document}

\title{Semantische Segmentierung von medizinischen Instrumenten mit Deep Learning Techniken}
\author{Martin Thoma}
\date{August 2016}
\subject{Computer Science}
\setbeamertemplate{navigation symbols}{}


\section{Deep Learning ist der Goldstandard für Bilderkennung}
\begin{frame}[plain]{Wissenschaftliche Aussage}
\begin{center}
    \only<1-2>{\textbf{Deep Learning ist der Goldstandard für Bilderkennung}}

    \uncover<2>{Was ist \enquote{Deep Learning}?}

    \only<3-4>{\textbf{Neuronale Netze sind der Goldstandard für Bilderkennung}}

    \uncover<4>{Klassifikation? Semantische Segmentierung? Detektion? Lokalisierung?}

    \only<5-6>{\textbf{Neuronale Netze sind der Goldstandard für Bildklassifikation}}

    \uncover<6>{Fotos, medizinische Bilder, Luftbilder, Dokumente, \dots?}

    \only<7-8>{\textbf{Neuronale Netze sind der Goldstandard für die Klassifikation von Fotos}\\}
    \only<8>{Goldstandard ist ein Schlagwort. Es wird einerseits zur Bezeichnung von Verfahren verwendet, die bislang unübertroffen sind.\\
    {\tiny Quelle: \href{https://de.wikipedia.org/w/index.php?title=Goldstandard_(Verfahren)&oldid=151270928}{de.wikipedia.org/w/index.php?title=Goldstandard\_(Verfahren)\&oldid=151270928}}}
\end{center}
\end{frame}

\begin{frame}[plain]{ImageNet / ILSVRC 2014}

ImageNet ist ein Datensatz mit
\begin{itemize}
    \item \num{14197122} Bildern und
    \item \num{21841} Klassen (non-empty synsets)
\end{itemize}



ILSVRC (Large Scale Visual Recognition Challenge) hatte 2014

\begin{itemize}
    \item \textbf{1000 Klassen}: abacus, abaya, academic gown, accordion,
    acorn, acorn squash, acoustic guitar, admiral, affenpinscher, Afghan hound,
    \dots
    \item 
\end{itemize}

Quellen: \href{http://image-net.org/about-stats}{image-net.org/about-stats},
O. Russakovsky, J. Deng et al.ImageNet Large Scale Visual Recognition Challenge. IJCV, 2015
\end{frame}

\end{document}
