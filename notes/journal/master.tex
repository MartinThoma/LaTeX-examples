\documentclass{article}

% Preamble stuff
% Multi language stuff
\usepackage{arabtex}
\usepackage[utf8]{inputenc}
\usepackage[T1]{fontenc}
\usepackage[icelandic,latin,dutch,spanish,arabic,english]{babel}

% \usepackage{cmbright}
\usepackage{stix}
\usepackage{textcomp}
\usepackage{amsmath, amssymb, amsthm}
\usepackage{mathtools}
\usepackage{tikz-cd}
\usepackage{titlesec}
\usepackage{pgfmath}
\usepackage{pgfcalendar}
\usepackage{float}
\usepackage{lastpage}

% page layout
\usepackage{geometry}
\geometry{
	a4paper,
	total={170mm,257mm},
	left=20mm,
	top=20mm,
}

% figure support
\usepackage{import}
\usepackage[shortlabels]{enumitem}
\usepackage{xifthen}
\pdfminorversion=7
\usepackage{pdfpages}
\usepackage{transparent}
\newcommand{\incfig}[2]{%
  \includepdf[pages=#2,pagecommand={},width=\textwidth]{#1}
}

\pdfsuppresswarningpagegroup=1

\usepackage{xcolor}
\usepackage{parskip}
\usepackage{soul}

\makeatletter

\def\@noteday{}
\newcommand{\noteday}[1]{
  \def\@noteday{#1}
}

% fancy headers
\usepackage{fancyhdr}
\pagestyle{fancy}

\renewcommand{\headrulewidth}{0pt}

\fancyhead{}
\fancyfoot[C]{\textit{\Large\@noteday}}
\fancyfoot[R]{\Large{\thepage\ of \pageref{LastPage}}}
\fancyfoot[L]{\Large{Hashem A. Damrah}}

\makeatother

\newcommand{\nothingtosay}{{\color{red}I don't feel like writing anything.}}

\usepackage{tcolorbox}
\usepackage{thmtools}
\usepackage[framemethod=TikZ]{mdframed}

\mdfsetup{
  skipabove=1em,
  skipbelow=0em,
  innertopmargin=5pt,
  innerbottommargin=6pt
}

\newmdenv[
	backgroundcolor = red!10,
	frametitle=Wrong
]{wrong}
\newmdenv[
	backgroundcolor = green!10,
	frametitle=Correct
]{correct}
\renewmdenv[
	bottomline=false,
	topline=false,
	rightline=false,
	fontcolor=black!70,
]{quote}
\newmdenv[
	backgroundcolor = blue!10,
	frametitle=Goals for Today
]{goals}
\newmdenv[
	backgroundcolor = blue!10,
	frametitle=Status for my Goals
]{status}
\newmdenv[
	backgroundcolor = blue!10,
	frametitle=The results of my Goals
]{results}

\declaretheoremstyle[headfont=\bfseries\sffamily, bodyfont=\normalfont, numbered=no, mdframed={ nobreak } ]{mainenv}

\declaretheorem[style=mainenv, name=Definition]{definition}
\declaretheorem[style=mainenv, name=Question]{question}
\declaretheorem[style=mainenv, name=Confusion]{confusion}
\declaretheorem[style=mainenv, name=Answer]{answer}
\declaretheorem[style=mainenv, name=Thought]{thought}
\declaretheorem[style=mainenv, name=Idea]{idea}
\declaretheorem[style=mainenv, name=Claim]{claim}
\declaretheorem[style=mainenv, name=Remark]{remark}
\declaretheorem[style=mainenv, name=Problem]{problem}
\declaretheorem[style=mainenv, name=Example]{example}
\declaretheorem[style=mainenv, name=Rant]{rant}
\declaretheorem[style=mainenv, name=Note]{note}
\declaretheorem[style=mainenv, name=TODO]{todo}

\usepackage{hyperref}
\hypersetup{hidelinks}
\usepackage{fontawesome}
\usepackage{xifthen}% provides \isempty test
\newcommand\pdfref[3]{%
  \href{phd://open-paper?id=#1&page=#2}{%
  \textup{[\textbf{\ifthenelse{\isempty{#3}}{here}{#3}}]}}%
}
\newcommand\urlref[2]{%
  \href{#1}{\raisebox{0.15ex}{\scriptsize \faLink}\:\textup{\textbf{#2}}}%
}
\newcommand\absolutefileref[2]{%
  \href{run:#1}{\raisebox{0.15ex}{\scriptsize \faFile}\:\textup{\textbf{#2}}}%
}

% this will contain the current date in yyyy-mm-dd format
\def\formatteddate{}
\newcommand\fileref[2]{
  \IfFileExists{./\formatteddate/#1}{
    \absolutefileref{./\formatteddate/#1}{#2}
  }{
    \textcolor{gray}{\absolutefileref{./\formatteddate/#1}{#2}}
  }
}
\newcommand{\xournal}{\fileref{note.xopp}{Handwritten notes}}%

\newcommand{\morning}{\subsection*{Morning}\vspace*{1em}}
\newcommand{\afternoon}{\subsection*{Afternoon}\vspace*{1em}}
\newcommand{\evening}{\subsection*{Evening}\vspace*{1em}}
\newcommand{\night}{\subsection*{Night}\vspace*{1em}}
\newcommand{\todostatus}[1]{
  \ifthenelse{\equal{#1}{\string 0}}
    {\hfill\textbf{Done}}{}
  \ifthenelse{\equal{#1}{\string 1}}
    {\hfill\textbf{In Progress}}{}
  \ifthenelse{\equal{#1}{\string 2}}
   {\hfill\textbf{To-Do}}{}
}

\newcommand{\contentment}[1]{
  \begin{flushright}
    \textbf{Feeling of contentment for the day:}\hfill\textbf{\underline{#1/10}}
  \end{flushright}
}

\renewcommand{\time}[1]{
  \begin{flushright}
    \hfill\textbf{Current logging time: \underline{#1}}
  \end{flushright}
}


% New command stuff
\let\d\pgfcalendarshorthand
\newcommand\formatdate[2]{\pgfcalendar{cal}{#1}{#1}{#2}}

\newcommand\firstdate{2022-01-01}
\newcommand\lastdate{\year-\month-\day}

\newcommand\grayrule{{\color{gray} \noindent\makebox[\linewidth]{\rule{\paperwidth}{0.4pt}}}}

\def\firstoftwo#1#2{#1}
\def\secondoftwo#1#2{#2}
\def\iffileempty#1{%
	\ifnum0\pdffilesize{#1}>0
		\expandafter\secondoftwo
	\else
		\expandafter\firstoftwo
	\fi
}
\newcommand\emptyentry[3]{%
	\iffileempty{#1}{
		\pgfcalendarifdate{#3}{equals=\year-\month-\day}{
			\phantomsection\addcontentsline{toc}{subsubsection}{#2 (Empty) (Today)}
		}{
			\phantomsection\addcontentsline{toc}{subsubsection}{#2 (Empty)}
		}
		\newline {\LARGE {\color{red}No journal entry for today!
			}}}{
		\pgfcalendarifdate{#3}{equals=\year-\month-\day}{
			\phantomsection\addcontentsline{toc}{subsubsection}{#2 (Today)}
		}{
			\phantomsection\addcontentsline{toc}{subsubsection}{#2}
		}
		\input{#1}
	}
}

\iffalse
	Different commands you can use with the \d command:

	d: day
	w: week
	m: month
	y: year

	Different kinds of arguments you can use with the \d command:

	-:    Numerical representation with no leading zeros;
	=:    Numerical representation with a leading space for single digit numbers;
	0:    Numerical representation with a leading zero for single digit numbers;
	t:    Textual representation;
	.:    Abbreviated textual representation.

	Examples:
	\d{d}-    ->    1, 2, 3, 4, 5, ...
	\d{d}=    ->     1,  2,  3,  4,  5, ...
	\d{d}0    ->    01, 02, 03, 04, 05, ...
	\d{d}t    ->    no output
	\d{d}.    ->    no output

	\d{w}-    ->    no output
	\d{w}=    ->    no output
	\d{w}0    ->    no output
	\d{w}t    ->    Monday, Tuesday, Wednesday, ...
	\d{w}.    ->    Mon, Tue, Wed, ...

	\d{m}-    ->    1, 2, 3, 4, 5, ...
	\d{m}=    ->     1,  2,  3,  4,  5, ...
	\d{m}0    ->    01, 02, 03, 04, 05, ...
	\d{m}t    ->    January, February, March, ...
	\d{m}.    ->    Jan, Feb, Mar, ...

	\d{y}-    ->    2022, 2023, 2024, 2025, ...
	\d{y}=    ->    2022, 2023, 2024, 2025, ...
	\d{y}0    ->    2022, 2023, 2024, 2025, ...
	\d{y}t    ->    no output
	\d{y}.    ->    no output
\fi

% Main document
\begin{document}
%%%%%%%%%%%%%%%%
%  Title Page  %
%%%%%%%%%%%%%%%%

\begin{center}
	\huge{Journal}\\[0.4em]
	\Large{Your Name}\\[0.2em]
	\emph{
		From August 1 2022 to
		\formatdate{\lastdate}{\d{m}t \d{d}- \d{y}-}
	}
\end{center}

%%%%%%%%%%%%%%%%%%%%%%%
%  Table of Contents  %
%%%%%%%%%%%%%%%%%%%%%%%

\tableofcontents
\newpage

%%%%%%%%%%%%%%%%%%%%%%%%%%%%%%%%%%%%%%%%%%%%%%%%%%%%%%%%%%%%%%%%%%%%
%  Iterate through all dates from the first date to the last date  %
%%%%%%%%%%%%%%%%%%%%%%%%%%%%%%%%%%%%%%%%%%%%%%%%%%%%%%%%%%%%%%%%%%%%

\pgfcalendar{cal}{\firstdate}{\lastdate}{
	% Set some variables so I don't keep re-typing stuff
	\def\formatteddate{\d{y}0/\d{m}0/\d{d}0} % EXM: 2022/06/10
	\def\todayformatted{\d{y}0-\d{m}0-\d{d}0} % EXM: 2022-06-10

	\def\firstdayofyear{2022-01-01} % EXM: 2022-01-01
	\def\yeargoalsformat{./\d{y}0/goals.tex} % EXM: 2022/goals.tex

	\def\firstdayofmonth{\d{y}0-\d{m}0-01} % EXM: 2022-08-01
	\def\monthgoalsformat{./\d{y}0/\d{m}0/goals.tex} % EXM: 2022/01/goals.tex

	%%%%%%%%%%%%%%%%%%%%%%%%%%%%%%%%%%%%%%%%%%%%%
	%  Check if it's the first day of the year  %
	%%%%%%%%%%%%%%%%%%%%%%%%%%%%%%%%%%%%%%%%%%%%%

	\pgfcalendarifdate{\todayformatted}{equals=\firstdayofyear}{
		\noteday{Year \d{y}0}
		% Add the year to the table of contents
		{\Huge \d{y}0}
		\phantomsection\addcontentsline{toc}{section}{\d{y}0}%
		% Check if there's a file EXM: 2022/goals.tex
		\IfFileExists{\yeargoalsformat}{
			\newline
			\newline
			% Add the goals to the table of contents
			{\LARGE Goals for \d{y}0}
			\phantomsection\addcontentsline{toc}{subsection}{Goals for \textbf{\d{y}0}}
			\newline
			\newline
			\input{\yeargoalsformat}
			\newpage
		}{}
	}{}

	%%%%%%%%%%%%%%%%%%%%%%%%%%%%%%%%%%%%%%%%%%%%%%
	%  Check if it's the first day of the month  %
	%%%%%%%%%%%%%%%%%%%%%%%%%%%%%%%%%%%%%%%%%%%%%%

	\ifdate{equals=\firstdayofmonth}{
		\noteday{Month \d{m}t}
		% Check if there's a file EXM: 2022/07/goals.tex
		\IfFileExists{\monthgoalsformat}{
			% Add the month to the table of contents
			% \section*{\d{m}t}
			{\LARGE \d{m}t}
			\phantomsection\addcontentsline{toc}{subsection}{\d{m}t}
			\newline
			\newline
			% Add the goals to the table of contents
			{\Large Goals for \d{m}t}
			\phantomsection\addcontentsline{toc}{subsubsection}{Goals for \textbf{\d{m}t}}
			\newline
			\newline
			\input{\monthgoalsformat}
			\newpage
		}{}
	}{}

	%%%%%%%%%%%%%%%%%%%%%%%%%%%%%%%%%%%%%%%%%%%%%%%%%%%%%%%%%%%%%%%%%%%%%%%%%
	%  Check if the note.tex file exists in the following example directory %
	%                       EXM: 2022/05/14/note.tex                        %
	%%%%%%%%%%%%%%%%%%%%%%%%%%%%%%%%%%%%%%%%%%%%%%%%%%%%%%%%%%%%%%%%%%%%%%%%%

	\IfFileExists{./\formatteddate/note.tex}{
		\def\dayout{}
		\def\dayouttotable{}

		% Checking if it's the first day of the month
		\ifthenelse{\equal{\d{d}0}{\string 01}}{
			\def\dayout{\d{w}t the \d{d}-st of \d{m}t, \d{y}0}
			\def\dayouttotable{\d{w}t the \d{d}0st of \d{m}t, \d{y}0}
		}{
			% Checking if it's the second day of the month
			\ifthenelse{\equal{\d{d}0}{\string 02}}{
				\def\dayout{\d{w}t the \d{d}-nd of \d{m}t, \d{y}0}
				\def\dayouttotable{\d{w}t the \d{d}0nd of \d{m}t, \d{y}0}
			}{
				% Checking if it's the third day of the month
				\ifthenelse{\equal{\d{d}0}{\string 03}}{
					\def\dayout{\d{w}t the \d{d}-rd of \d{m}t, \d{y}0}
					\def\dayouttotable{\d{w}t the \d{d}0rd of \d{m}t, \d{y}0}
				}{
					\def\dayout{\d{w}t the \d{d}-th of \d{m}t, \d{y}0}
					\def\dayouttotable{\d{w}t the \d{d}0th of \d{m}t, \d{y}0}
				}
			}
		}

		\noteday{\dayout}

		\vspace*{1em}
		{\Large \dayout}
		\emptyentry{./\formatteddate/note.tex}{\dayouttotable}{\todayformatted}
		\begin{flushright}
			\IfFileExists{./\formatteddate/note.xopp}{\hfill \xournal}{\hfill \\ }
		\end{flushright}
		\newpage
	}{}
}
\end{document}
