%!TEX root = ../booka4.tex
\section{Standards und Verordnungen}\label{ch:standards}
Für den Automobilbereich existieren viele Standards und Verordnungen. In diesem
Abschnitt wird eine Auswahl vorgestellt, die Fahrzeuge der Klassen M$_1$ und
N$_1$ betrifft. Das sind Fahrzeuge zur Personenbeförderung
\enquote{mit mindestens vier Rädern und höchstens acht Sitzplätzen außer dem Fahrersitz}
sowie \enquote{für die Güterbeförderung ausgelegte und gebaute Kraftfahrzeuge
mit einer zulässigen Gesamtmasse von
3,5~Tonnen}~\cite{Richtlinie70/156/EWG:Fahrzeugklassen}.

In der EU wurde mit~\cite{EUDirective98/69/EC} die OBD-Schnittstelle
verpflichtend für Fahrzeuge der Klasse M$_1$ und N$_1$ mit Fremdzündungsmotor
ab 1.~Januar 2004. Die EU-Direktive führt weiter die in der ISO~DIS~15031-6
Norm aufgeführten Fehlercodes als Minimalstandard ein. Diese müssen
\enquote{für genormte Diagnosegeräte \elide uneingeschränkt zugänglich sein}.
Außerdem muss die Schnittstelle im Auto so verbaut werden, dass sie
\enquote{für das Servicepersonal leicht zugänglich \elide ist}.

Der Software-Zugang ist durch J2534 der Society of Automotive Engineers
standardisiert~\cite{SAE2004}. Dieser Standard stellt sicher, dass unabhängig
vom OBD-Reader Diagnosen über das Auto erstellt und die ECUs umprogrammiert
bzw. mit Aktualisierungen versorgt werden können.

Um die Daten bereitzustellen, werden verschiedene elektronische Komponenten
über den CAN-Bus vernetzt. Dieser ist durch ISO~11898 genormt.

Weiterhin wurde in der EU mit \cite{EURegulation661/2009} beschlossen, dass ab
1.~November 2012 alle PKWs für Neuzulassungen ein System zur
Reifen\-druck\-über\-wachung (engl. \textit{tire pressure monitoring system}, kurz
\textit{TPMS}) besitzen müssen. Seit 1.~November 2014 müssen alle Neuwagen ein
solches System besitzen. Da sich die Räder schnell drehen ist eine
kabelgebundene Übertragung der Druckmesswerte nicht durchführbar. Daher sendet
jeder Reifen kabellos ein Signal, welches von einem oder mehreren Sensoren im
Auto aufgenommen wird.

Mit \cite{EURegulation2015/ecall} wird für Fahrzeuge, die ab dem 31.~März 2018
gebaut werden das eCall-System, ein elektronisches Notrufsystem, verpflichtend.
Dabei müssen dem eCall-System \enquote{präzise\mbox{[-]} und verlässliche\mbox{[-]}
Positionsdaten} zur Verfügung stehen, welche über das globales
Satelliten\-navigations\-system Galileo und dem Erweiterungssystem EGNOS geschehen
soll. eCall soll über öffentliche Mobilfunknetze eine \enquote{Tonverbindung
zwischen den Fahrzeug\-insassen und einer eCall-Notruf\-abfrage\-stelle} herstellen
können. Außerdem muss ein Mindestdatensatz übermittelt werden, welcher in
DIN EN 15722:2011 geregelt ist. Diese Funktionen müssen im Fall eines schweren
Unfalls automatisch durchgeführt werden können.

% TODO Marvin:
% Hier gutes Beispiel. Eventuell kannst du kurz die Auswirkung beschrieben, die eine standartieserung 
% der Telematikeinheit auf die Angreifbarkeit hat. A la: In Zukunft ist jedes Auto per über eine Einheitliche Schnittstelle angreifbar.
% Entdecken Angreifer einen Bug in dieser sind alle Autos gefährdet. 

