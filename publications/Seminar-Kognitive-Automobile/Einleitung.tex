%!TEX root = ../booka4.tex

\section{Einleitung}
Kognitive Automobile sind, im Gegensatz zu klassischen Automobilen, in der Lage
ihre Umwelt und sich selbst wahrzunehmen und dem Fahrer zu assistieren oder
auch teil- bzw. vollautonom zu fahren. Diese Systeme benötigen Zugriff auf
Sensoren und Aktoren, um ihre Aufgabe zu erfüllen. So benötigt ein Auto mit
Antiblockiersystem beispielsweise die Drehzahl an jedem Reifen und die
Möglichkeit die Bremsen zu beeinflussen; für Einparkhilfen werden Sensoren
benötigt, welche die Distanz zu Hindernissen wahrnehmen sowie Aktoren, die das
Auto lenken und beschleunigen können. Weitere dieser Systeme sind
Spurhalteassistenz, Spurwechselassistenz und Fernlichtassistenz.

% TODO: Marvin
%   
% Wenn du das ganze an das BKA etc. schicken möchtest,
% hast du mal nachgedacht einen Absatz über praktische relevantz zu machen? 
% Also so ein wenig Panikmache a la gezielter Mord ist möglich
% Extrimisten könnten gezielt angriffe machen etc. ? 
% Die reelle Gefahr die von den beschriebenen Angriffen ausgeht
% kann von nicht Fachpersonal leicht unterschätzt werden
%

% TODO: Marvin
%
% Allgemein finde ich, dass du recht wenig über Implikationen schreibst.
% Eventuell könntest du die gesammte Veröffentlichung noch damit Würzen,
% dass bei den verschiedenen Aspekten öfter praxisrelevante Beispiele beschreibst.
% Ich werde ein paar solcher Stellen mit TODOs markieren.
%

% TODO: Marvin
% Dieser Absatz (Absatz 2) gehört imho nicht in die Einleitung.   
% Würde das unter \cref{ch:standards} tun.


Als immer mehr elektronische Systeme in Autos verbaut wurden, die teilweise
sich überschneidende Aufgaben erledigt haben, wurde der CAN-Bus
entwickelt~\cite{Kiencke1986}. Über ihn kommunizieren elektronische
Steuergeräte, sog. \textit{ECUs} (engl. \textit{electronic control units}).
Diese werden beispielsweise für ABS und ESP eingesetzt.

% TODO: Marvin
%   
% Eventuell beginnen mit: In \cref{ch:standards} werden standards wie ... 
% Finde "der folgende Abschnitt" etwas unorthodox


Der folgende Abschnitt geht auf Standards wie den CAN-Bus und Verordnungen, die
in der Europäischen Union gültig sind, ein. In \cref{ch:attack} werden
Angriffsziele und Grundlagen zu den Angriffen erklärt, sodass in
\cref{ch:defense} mögliche Verteidigungsmaßnahmen erläutert werden können.
