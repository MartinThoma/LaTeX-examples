% used PGFPlots v1.14
% (inspired by Jake's answer given here
% <http://tex.stackexchange.com/a/340939/95441>)
\documentclass[border=5pt]{standalone}
\usepackage{pgfplots}
    \pgfplotsset{
        compat=1.3,
    }
    % create cycle lists that uses the style from OPs figure
    % <https://upload.wikimedia.org/wikipedia/en/d/df/TnormPDF.png>
    \pgfplotscreateplotcyclelist{line styles}{
        black,solid\\
        blue,dashed\\
        red,dotted\\
        orange,dashdotted\\
    }
    % define a command which stores all commands that are needed for every
    % `raw gnuplot' call
    \newcommand*\GnuplotDefs{
        % set number of samples
        set samples 50;
        %
        %%% from <https://en.wikipedia.org/wiki/Normal_distribution>
        % cumulative distribution function (CDF) of normal distribution
        cdfn(x,mu,sd) = 0.5 * ( 1 + erf( (x-mu)/sd/sqrt(2)) );
        % probability density function (PDF) of normal distribution
        pdfn(x,mu,sd) = 1/(sd*sqrt(2*pi)) * exp( -(x-mu)^2 / (2*sd^2) );
        % PDF of a truncated normal distribution
        tpdfn(x,mu,sd,a,b) = pdfn(x,mu,sd) / ( cdfn(b,mu,sd) - cdfn(a,mu,sd) );
    }
\begin{document}
    \begin{tikzpicture}
            % define macros which are needed for the axis limits as well as for
            % setting the domain of calculation
            \pgfmathsetmacro{\xmin}{0}
            \pgfmathsetmacro{\xmax}{1}
        \begin{axis}[
            xmin=\xmin,
            xmax=\xmax,
            ymin=0,
            %ymax=0.23,
            %ytick distance=0.05,
            %enlargelimits=0.05,
            no markers,
            smooth,
            % use the above created cycle list ...
            cycle list name=line styles,
            % ... and append the following style to all `\addplot' calls
            every axis plot post/.append style={
                very thick,
            },
            yticklabel style={
                /pgf/number format/.cd,
                    fixed,
                    fixed zerofill,
                    precision=2,
            },
            xlabel={x},
            ylabel={probability density},
        ]
            \addplot gnuplot [raw gnuplot] {
                % first call all the "common" definitions
                \GnuplotDefs
                % and then create the data tables
                % in GnuPlot `x` key is identical to PGFPlots `domain` key
                plot [x=\xmin:\xmax] tpdfn(x,0.8,0.3, 0, 1);
            };
            \addplot gnuplot [raw gnuplot] {
                \GnuplotDefs
                plot [x=\xmin:\xmax] tpdfn(x,0.7,0.1, 0, 1);
            };
        \end{axis}
    \end{tikzpicture}
\end{document}
